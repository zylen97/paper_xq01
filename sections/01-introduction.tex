\section{Introduction}
\label{sec:introduction}

Rapid urbanization and technological advancement have significantly expanded disaster impacts, creating unprecedented challenges for governmental emergency management (ref). Modern urban systems' interconnected nature demands more sophisticated emergency response mechanisms (ref). Inter-regional governmental collaboration has emerged as a critical solution for enhancing resource efficiency and response capabilities (ref). This represents a shift from hierarchical to flexible, network-based emergency management systems (ref).

At the policy level, the importance of inter-governmental collaboration has been explicitly recognized in national emergency planning frameworks. For instance, China's 14th Five-Year National Emergency System Plan explicitly mandates the establishment of robust regional collaborative response mechanisms (ref). Similar policy initiatives across various countries emphasize the need for horizontal governmental cooperation to address the increasingly trans-boundary nature of disaster impacts (ref). Despite this policy emphasis and theoretical recognition, the practical implementation of such collaborative mechanisms faces substantial challenges that significantly impede their effectiveness in real-world emergency scenarios.

Three primary obstacles systematically undermine the effectiveness of horizontal inter-governmental collaboration in emergency response. First, the absence of clear delineation of rights and responsibilities creates operational ambiguity that hampers decisive action during critical emergency periods (ref). Many collaborative efforts rely heavily on spontaneous cooperation between horizontal governments without effective constraints or guidance from higher-level vertical administrative structures, resulting in coordination failures when rapid response is most needed (ref). Second, information sharing barriers severely compromise collaborative efficiency, as information transmission suffers from both technical obstacles and institutional resistance, creating dangerous blind spots in emergency situational awareness (ref). The lack of standardized information sharing protocols and interoperable communication systems further exacerbates these challenges, leading to duplicated efforts and missed opportunities for resource optimization (ref). Third, benefit coordination difficulties arise from the dominance of administrative division-based management models, where local governments inherently prioritize their own jurisdictional interests over regional collective benefits (ref). This "every man for himself" mentality becomes particularly pronounced when collaborative benefit distribution mechanisms are inadequately designed or absent, drastically reducing cooperation incentives and undermining the potential synergies of joint emergency response efforts (ref).

The existing body of literature has extensively explored the theoretical foundations and practical implications of inter-governmental collaboration in emergency management from various perspectives. Scholars have investigated the fundamental necessity and influencing factors of governmental collaboration through both theoretical frameworks and empirical case studies (ref). For instance, research on humanitarian organizations has examined inventory cooperation mechanisms and resource sharing strategies that could inform governmental collaboration models (ref). Additionally, studies adopting macro-level perspectives have analyzed the game-theoretic relationships between central and local governments, providing insights into the strategic interactions that shape collaborative behaviors (ref). Evolutionary game theory has emerged as a particularly valuable analytical tool for modeling multi-agent coordination in emergency management contexts, offering dynamic perspectives on how cooperation patterns evolve over time under different institutional and environmental conditions (ref).

However, significant research gaps persist despite these valuable contributions to the field. First, there is a notable absence of rigorous modeling and analysis regarding information sharing platforms as specific solutions to collaboration challenges (ref). While information sharing barriers are widely recognized as critical obstacles to effective collaboration, few studies have employed mathematical models to quantitatively analyze how information sharing platforms might influence collaborative strategy evolution and emergency response outcomes (ref). Second, existing analytical approaches remain predominantly macro-level, focusing on aggregate benefits and losses without adequately capturing the micro-level practical factors that shape actual collaborative behaviors (ref). Critical operational details such as specific material coordination quantities, transportation costs, and the nonlinear characteristics of rescue benefits—which this paper models using S-shaped functions—have received insufficient attention in current research (ref). Third, the literature has largely overlooked the crucial role of benefit allocation and distribution mechanisms in horizontal governmental collaboration (ref). The design and implementation of specific benefit distribution schemes between horizontal governments, which are essential for overcoming local protectionism and sustaining long-term collaborative relationships, remain understudied despite their fundamental importance to collaborative success (ref).

These research gaps become particularly problematic when considering the practical implementation of emergency collaboration systems. The lack of quantitative models for information sharing platforms prevents policymakers from understanding the potential returns on investment in such infrastructure or optimizing their design for maximum collaborative benefit (ref). Similarly, the absence of micro-level analysis limits our understanding of how specific operational factors influence collaboration decisions, making it difficult to identify targeted interventions that could enhance cooperation likelihood (ref). Furthermore, without adequate attention to benefit distribution mechanisms, even well-intentioned collaborative initiatives may fail due to perceived inequities or misaligned incentives among participating governments (ref).

To address these critical gaps, this research pursues three primary objectives that collectively advance our understanding of horizontal governmental collaboration in emergency response. First, we construct a comprehensive evolutionary game model that incorporates both internal and external micro-level factors affecting resource sharing, including both material supplies and information exchange, in horizontal governmental emergency collaboration (ref). This model explicitly captures the complex interdependencies between disaster characteristics, regional positioning, cooperation efficiency, rescue benefits, and benefit coordination mechanisms that shape collaborative decisions in real-world emergency scenarios. Second, we conduct a comparative analysis of strategy evolution paths under two distinct scenarios: one with a vertical government-established information sharing platform and one without such infrastructure (ref). This comparison enables quantitative assessment of how information platforms influence the emergence and stability of cooperative equilibria, providing concrete evidence for the value of such investments. Third, we aim to provide theoretical foundations and policy recommendations for constructing effective inter-governmental collaboration mechanisms that can overcome the identified barriers to cooperation (ref).

The research questions guiding this investigation focus on understanding the micro-level determinants and macro-level interventions that shape collaborative behaviors in emergency response contexts. Specifically, we seek to identify the key micro-level factors—including disaster characteristics, regional location, cooperation efficiency, rescue benefits, and benefit coordination schemes—that influence horizontal local governments' strategic choices between cooperation and non-cooperation in disaster emergency response (ref). Additionally, we examine how information sharing platforms established by vertical governments can alter information efficiency and introduce incentive mechanisms to guide and influence the strategic evolution paths of horizontal governments toward more cooperative outcomes (ref). These questions are addressed through rigorous mathematical modeling and systematic analysis that bridges theoretical insights with practical implementation considerations.

This paper makes several significant contributions to the emergency management literature and practice. By developing a detailed evolutionary game model that captures previously overlooked micro-level factors, we provide a more nuanced understanding of the conditions under which horizontal governmental cooperation emerges and persists in emergency contexts. Our quantitative analysis of information sharing platforms offers concrete evidence for their value in promoting cooperation, informing investment and design decisions for emergency management infrastructure. Furthermore, our examination of benefit distribution mechanisms provides practical guidance for designing collaborative agreements that align individual governmental interests with collective emergency response objectives. These contributions collectively advance both theoretical understanding and practical implementation of inter-governmental collaboration in emergency management, offering valuable insights for researchers, policymakers, and emergency management practitioners seeking to enhance collaborative emergency response capabilities in an increasingly interconnected and disaster-prone world.