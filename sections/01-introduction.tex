\section{Introduction}
\label{sec:introduction}

The construction industry is undergoing a transformation, challenging the efficacy of construction enterprises' traditional, often fragmented, project-based business modes \citep{dang2025Assessing,zhao2024Using}. On one hand, conventional operational pressures, including rising material costs and skilled labor shortages, continue to affect construction enterprises \citep{aghimien2022Dynamic}. On the other hand, disruptive external demands for digitalization and sustainability are intensifying these pressures \citep{wen2025Gap}. For instance, digitalization from building information modeling (BIM) or robotics can help mitigate labor shortages; however, it requires high capital investment and complex implementation across projects, which in turn reduces enterprises' profit margins \citep{shao2025Competitive}. Sustainability mandates, such as net-zero regulations \citep{winch2023Projectinga}, introduce new compliance costs and operational complexities, requiring a more skilled workforce that is already in short supply \citep{wang2023Exploring}. These trends create a high-velocity environment where general advantages, such as a successful project bid or isolated technology adoption, are no longer sufficient to ensure long-term performance. Consequently, a practical problem facing construction enterprises' managers is how to cultivate \textit{sustainable competitive advantage} --- a set of valuable, rare, and difficult-to-imitate organizational capabilities \citep{adner2006Demandbased} --- to ensure growth within the constraints of project-based business.

Sustainable competitive advantage, coined by \citet{barney1991Firm}, denotes a firm's capacity to "\textit{implemente a value creating strategy not simultaneously being implemented by any current or potential competitors and when these other firms are unable to duplicate the benefits of this strategy}" \citep[p.~102]{barney1991Firm}. Developing sustainable competitive advantage is particularly critical for construction enterprises due to the inherent discontinuity and fragmentation of construction enterprises' operations \citep{betts1994Sustainablea}. Unlike continuous manufacturing processes, construction operations are project-based and temporary, often leading to the loss of knowledge and efficiency when project teams disband \citep{sydow2018Projects}. In this context, sustainable competitive advantage serves as a mechanism for organizational continuity, allowing firms to transfer technological innovations and management expertise across discrete projects , rather than treating each project as an isolated event \citep{malik2023Green}. Specifically, in a market historically dominated by fierce price competition and thin profit margins \citep{sharma2023Construction}, possessing rare and non-substitutable capabilities enables firms to differentiate themselves beyond mere cost leadership \citep{deng2014Developing,medina2014Project}. Consequently, clarifying the formation of sustainable competitive advantage is imperative for understanding how to ensure construction enterprises' long-term survival.

Existing construction engineering and management (CEM) literature has investigated the drivers of construction enterprises' competitive advantage, which are categorized into three aspects. First, research on resource-based drivers identifies critical success factors, such as resource amalgamation \citep{wang2024Strategic}, dynamic capability \citep{ning2022How} and human capital \citep{sarihi2020Multiskilled}, as the foundation of market position \citep{deng2014Developing,zhao2015Enterprise}. Second, studies on strategy-based drivers explore how specific business models, including internationalization \citep{jang2020Classifying}, networking \citep{lello2024Professional}, and projectization \citep{barbosa2024Multilevel} enable firms to leverage core competencies. Third, works on innovation-based drivers highlight how digital technologies like BIM \citep{shao2025Competitive}, digital transformation \citep{wang2025Digital}, and green innovation \citep{dang2025Assessing}, reshape competitive strategies by offering new avenues for differentiation. Collectively, these studies offer valuable insights into the specific determinants of competitiveness.

However, despite these advancements, two research gaps remain in the CEM literature that hinder a comprehensive understanding of the formation of construction enterprises' sustainable competitive advantage. \textit{First}, CEM literature often focuses on static or conventional competitive advantage rather than sustainable competitive advantage. Related studies often treat competitive advantage as a direct, often immediate outcome of achieving profitability levels \citep{li2025Impact}, possessing specific business resources \citep{wang2024Strategic}, or adopting new tools (e.g., data-driven techniques \citep{guo2023Measuring}). In contrast, sustainable competitive advantage emphasizes the formation of unique and difficult-to-imitate advantages through continuous resource accumulation and capability enhancement \citep{abdeen2025Developing,sabug2020Competitive}. Although some CEM studies have discussed sustainable competetive advantage, they failed to capture its core attributes. For instance, \citet{toor2010Positive} conceptualized sustainable competetive advantage primarily as a static derivative of leadership, while \citet{betts1994Sustainablea} equated it with critical success factors of construction enterprises. These studies overlooked the nature of sustainable competitive advantage, rendering it insufficient to guide construction enterprises' long-term survival. \textit{Second}, existing research on construction enterprises' competitive advantage is fragmented, predominantly examining antecedents from isolated perspectives. A substantial body of literature relies on a "net-effect" logic to assess the specific impact of single variables, such as the implementation of BIM \citep{shao2025Competitive}, the efficacy of project management tools and techniques \citep{li2025Impact}, or drivers in specific market segments like international high-speed rail \citep{niu2021Determinants,zhang2021Critical}. While these studies deepen our understanding of individual factors, they assume that variables operate independently, thereby neglecting the causal complexity of how advantages are built in a multifaceted environment. In reality, construction enterprises do not strictly respond to single environmental stimuli; rather, they "orchestrate" a complex combination of drivers to navigate challenges \citep{black1994Strategic}. CEM literature lacks an integrative, configurational perspective to systematically unravel how distinct factors interact and combine through multiple pathways (i.e., equifinality) to jointly generate sustainable competitive advantage. Therefore, \textbf{\textit{the first aim of this study is to investigate the configurations of factors that enable construction enterprises to achieve sustainable competitive advantage.}} 

While sustainable competitive advantage is crucial for sustained growth of construction enterprises, organizational resilience serves as the critical safeguard for survival and stability in turbulent environments \citep{yang2024What}. Organizational resilience refers to an enterprise's latent capacity to anticipate, absorb, and respond to contingencies, particularly crisis events \citep{shao2024Contradiction}. Unlike sustainable competitive advantage, which primarily targets superior market position and exceptional performance, organizational resilience prioritizes stable operational performance and continuity. Recent evidence from the COVID-19 pandemic underscores this distinction \citep{chih2022Resilience,lv2024Digital}. Despite examinations of organizational resilience in CEM literature, few studies have integrated resilience with sustainable competitive advantage. Sustainable competitive advantage and organizational resilience represent two distinct yet mutually reinforcing dimensions of long-term viability. Focusing solely on competitive superiority suggests profitability in stable times, but may create vulnerability to sudden crises. This disconnection hinders our understanding of how construction enterprises can balance the pursuit of market dominance with the necessity of shock absorption. Consequently, \textbf{\textit{the second aim of this study is to examine which configurations leading to sustainable competitive advantage simultaneously enable construction enterprises to maintain organizational resilience.}}

To bridge these knowledge gaps, we developed an integrated "Resource-Environment-Strategy" framework to investigate the configurations of factors across dimensions of organizational resource, external environment, and strategic orientation. This framework is derived from the literature on sustainable competitive advantage, and is further grounded in the body of CEM literature regarding competitive advantage. Under these three dimensions, we identified nine antecedent conditions: cost stickiness, firm size, corporate social responsibility, digital transformation, environmental dynamism, environmental munificence, diversification, differentiation, and cost leadership. Empirically, this study utilizes a panel dataset including 2,150 observations from 118 listed Chinese construction enterprises spanning 2014-2023. We employed necessary condition analysis and time-series qualitative comparative analysis to examine how configurations of these nine conditions relate to sustainable competitive advantage and organizational resilience. Furthermore, empirical results were validated through illustrative case studies of specific construction enterprises to ensure contextual validity. Therefore, \textbf{\textit{this study answers the following two research questions (RQs):}}

\textit{RQ1: What are the multidimensional configurations of resource, environmental, and strategic factors that enable construction enterprises to achieve high sustainable competitive advantage?}

\textit{RQ2: Which of these causal configurations simultaneously allow construction enterprises to maintain a high level of organizational resilience in a dynamic environment?}

The remainder of this paper is organized as follows. Section 2 reviews the related works and establishes the "Resource-Environment-Strategy" framework. Section 3 details the research methods. Section 4 presents the empirical results of the configurational analysis and the necessary condition analysis. Section 5 discusses the key findings, elaborating on the theoretical contributions and providing practical implications for managers. Finally, Section 6 concludes the study and outlines limitations for future research.