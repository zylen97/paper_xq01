\section{Introduction}
\label{sec:introduction}

The construction industry is undergoing a transformation, challenging the efficiency of construction enterprises' traditional, often fragmented, project-based business modes \citep{dang2025Assessing}. On the one hand, conventional operational pressures exist, including rising material costs and skilled labor shortages \citep{aghimien2022Dynamic}. On the other hand, disruptive external requirements, such as digitalization and sustainability, are intensifying these pressures \citep{wen2025Gap}. Digitalization from building information modeling (BIM) or robotics can mitigate labor shortages; however, it requires substantial capital investment and implementation across projects, which in turn reduces enterprises' short-term profit margins \citep{shao2025Competitive}. Sustainability mandates, such as net-zero regulations \citep{locatelli2025Social}, also introduce new compliance costs and operational complexities \citep{wang2023Exploring}. These trends create a high-velocity environment where conventional enterprises' advantages, such as a successful project bid or isolated technology adoption, are no longer sufficient to ensure long-term performance. A practical problem facing construction enterprises' managers is how to develop \textbf{\textit{sustainable competitive advantage}} --- a set of difficult-to-imitate edges \citep{adner2006Demandbased} --- to ensure growth within contexts of project-based business.

Sustainable competitive advantage, coined by \citet[p.~102]{barney1991Firm}, denotes a firm's capacity to "\textit{implement a value creating strategy not simultaneously being implemented by any current or potential competitors and when these other firms are unable to duplicate the benefits of this strategy}". Developing sustainable competitive advantage is critical for construction enterprises due to their inherent discontinuity and fragmentation of operations \citep{betts1994Sustainablea}. Unlike continuous manufacturing, construction operations are project-based and temporary, often leading to the loss of knowledge and efficiency when project teams disband \citep{sydow2018Projects}. In this context, sustainable competitive advantage improves organizational continuity, allowing enterprises to transfer technological innovations and management expertise across projects \citep{malik2023Green}. Specifically, the construction industry is historically dominated by fierce price competition and thin profit margins \citep{sharma2023Construction}. Rare and non-substitutable edges enable enterprises to differentiate themselves beyond low bid competition \citep{deng2014Developing}. Thus, clarifying the formation of construction enterprises' sustainable competitive advantage is imperative to the construction engineering and management (CEM) literature for understanding how to ensure long-term survival.

Existing CEM literature has largely investigated the antecedents of construction enterprises' competitive advantage in three aspects. First, research identifies critical resource-based factors, such as resource amalgamation \citep{wang2024Strategic}, dynamic capability \citep{ning2022How}, and human capital \citep{sarihi2020Multiskilled}, as the foundation of market position. Second, studies on strategy-based antecedents explore specific business modes, including internationalization \citep{jang2020Classifying}, networking \citep{lello2024Professional}, and projectization \citep{barbosa2024Multilevel}. Third, works on innovation-based antecedents highlight how digital technologies like BIM \citep{shao2025Competitive} and green innovation techniques \citep{dang2025Assessing} offer new avenues for sustainable development.

Despite these advancements, \textbf{\textit{two gaps remain in the CEM literature}} that hinder a comprehensive understanding of the formation of sustainable competitive advantage. \textbf{\textit{First}}, CEM literature pays most attention to static or conventional competitive advantage rather than sustainable competitive advantage. Related studies treated competitive advantage as a direct, immediate outcome of achieving profitability \citep{li2025Impact}, possessing specific business resources \citep{wang2024Strategic}, or adopting new tools. In contrast, sustainable competitive advantage emphasizes the formation of difficult-to-imitate advantages through continuous resource integration and capability enhancement \citep{abdeen2025Strategic}. Although some CEM studies have discussed sustainable competitive advantage, they failed to capture its core attributes. For instance, \citet{toor2010Positive} conceptualized sustainable competitive advantage as a derivative of leadership. \citet[p.~45]{betts1994Sustainablea} equated it with the critical success factors of construction enterprises. These studies overlooked the very nature of sustainable competitive advantage, rendering it insufficient to guide construction enterprises' managers. \textbf{\textit{Second}}, competitive advantage research is fragmented, mostly examining antecedents from isolated perspectives. A substantial body of literature relies on a "net-effect" logic to assess the specific impact of single variables, such as the implementation of BIM, project management tools and techniques \citep{li2025Impact}, or specific market segment drivers. In reality, construction enterprises do not strictly respond to single environmental stimuli; rather, they "orchestrate" antecedents to navigate challenges \citep{black1994Strategic}. While some recent studies have adopted a configurational perspective \citep{wang2024Strategic,shao2025Competitive}, they have not explicitly targeted competitive advantage as their subjects. CEM literature lacks an integrative, configurational perspective to systematically unravel how different antecedents interact and combine to jointly develop sustainable competitive advantage. Therefore, \textbf{\textit{the first aim of this study is to investigate the configurations of factors that shape construction enterprises' sustainable competitive advantage.}} 

While sustainable competitive advantage is crucial for sustained growth, organizational resilience serves as an important safeguard for survival and stability in turbulent environments \citep{yang2024What}. Organizational resilience refers to an enterprise's capacity to anticipate, absorb, and respond to contingencies \citep{zhang2024Deconstructing}. Despite systematic examinations of organizational resilience in CEM literature \citep{zhang2022Organizational}, few studies have integrated resilience with sustainable competitive advantage. Sustainable competitive advantage and organizational resilience represent two distinct yet mutually reinforcing dimensions of long-term viability. Sustainable competitive advantage primarily targets the sustainability of superior market position and inimitable performance, while organizational resilience prioritizes stable operations and continuity, specifically in crisis events \citep{shao2024Contradiction}. Focusing solely on competitive superiority may create vulnerability to crises, particularly in the highly uncertain post-pandemic era \citep{lv2024Digital}. This disconnection hinders our understanding of how construction enterprises can ensure long-term market dominance with the capability of addressing contingencies. Consequently, \textbf{\textit{the second aim of this study is to examine which configurations shaping sustainable competitive advantage simultaneously enable a high level of organizational resilience.}}

To bridge these knowledge gaps, we developed a "Resource-Environment-Strategy" framework to investigate configurations of nine antecedents across dimensions of organizational resource, external environment, and strategic orientation. This framework is derived from the literature on sustainable competitive advantage and is further grounded in the body of CEM research. This study utilizes a panel dataset including 1,061 observations from 115 listed Chinese construction enterprises spanning 2014-2023. We employed necessary condition analysis, time-series qualitative comparative analysis, and typical case tracing to examine how configurations of nine antecedents relate to sustainable competitive advantage and organizational resilience. Therefore, \textbf{\textit{this study answers two research questions (RQs):}}

\textit{RQ1: What are the configurations of factors across dimensions of organizational resource, external environment, and strategic orientation that shape construction enterprises' sustainable competitive advantage?}

\textit{RQ2: Which of these configurations simultaneously allow construction enterprises to maintain a high level of organizational resilience, specifically in crisis events?}

The remainder of this paper is organized as follows. Section 2 reviews the related works and establishes the "Resource-Environment-Strategy" framework. Section 3 details the research methods. Section 4 presents the empirical results. Section 5 discusses the key findings, elaborating on research implications. Section 6 concludes the study with limitations for future research.