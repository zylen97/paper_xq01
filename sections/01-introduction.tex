\section{Introduction}
\label{sec:introduction}

The architecture, engineering, and construction (AEC) industry is undergoing a transformation, challenging the efficacy of construction enterprises' traditional, often fragmented, project-based business modes \citep{dang2025Assessing,zhao2024Using}. On one hand, conventional operational pressures, including rising material costs and skilled labor shortages, continue to affect the industry \citep{aghimien2022Dynamic}. On the other hand, disruptive external demands for digitalization and sustainability are intensifying these pressures \citep{wen2025Gap}. For instance, digitalization from building information modeling or construction robotics can help mitigate labor shortages but requires high capital investment and complex implementation across projects, which in turn reduces profit margins \citep{shao2025Competitive}. Sustainability mandates, such as net-zero regulations \citep{winch2023Projectinga}, introduce new compliance costs and operational complexities, requiring a more skilled workforce that is already in short supply \citep{wang2023Exploring}. These trends create a high-velocity environment where temporary advantages—such as a single successful project bid or isolated technology adoption—are no longer sufficient to ensure long-term performance. Therefore, construction enterprises have to build a sustainable competitive advantage——a set of valuable, rare, and difficult-to-imitate organizational capabilities——to secure success in a project-based ecosystem.

Sustainable competitive advantage (SCA) denotes a firm's capacity to sustain superior performance relative to competitors through the deployment of unique resources and capabilities that exhibit temporal persistence and inimitability \citep{barney1991Firm}. Developing SCA is particularly critical for construction enterprises due to the inherent discontinuity and fragmentation of the industry. Unlike continuous manufacturing processes, construction operations are project-based and temporary, often leading to the loss of knowledge and efficiency when project teams disband \citep{dubois2002Construction}. In this context, SCA serves as a mechanism for organizational continuity, allowing firms to transfer technological innovations and management expertise across discrete projects, rather than treating each project as an isolated event. Furthermore, in a market historically dominated by fierce price competition and thin profit margins, possessing rare and non-substitutable capabilities enables firms to differentiate themselves beyond mere cost leadership, thereby securing resilience against market volatility and ensuring long-term survival \citep{hsu2022Risk}.