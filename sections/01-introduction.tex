\section{Introduction}
\label{sec:introduction}

Construction enterprises, characterized by project-based production, capital intensity, and geographical dispersion(A.S B et al. 2021), recently face three aspects of pressures that threaten their long-term viability. First, traditional operational challenges have intensified, including escalating material costs and acute skilled labor shortages (ref). Second, digital transformation imperatives require fundamental shifts from conventional project management to integrated digital ecosystems, demanding substantial investments in BIM, AI, and IoT technologies across fragmented project networks(Z. -S. Chen et al. 2024). Third, sustainability mandates pressure enterprises to adopt sustainable operations while maintaining strict project timelines and budget constraints(İbrahim Yitmen 2007). These challenges exhibit compounding effects: labor shortages accelerate automation needs(Zijun Z et al. 2025), sustainability requirements demand both technological upgrades and workforce retraining, while digital transformation costs strain already tight project margins.

In this context, developing sustainable competitive advantages becomes critical for construction enterprises' survival and growth. Sustainable competitive advantage denotes an enterprise's capacity to sustain superior performance relative to competitors through the deployment of unique resources and capabilities that exhibit temporal persistence and inimitability(6). Characterized by deep embeddedness in organizational capabilities, path dependency, and resistance to imitation(7), sustainable competitive advantages have become core strategic resources enabling construction enterprises to navigate structural industry transformations (e.g. XX) and cyclical market shocks (e.g. XX) (REF). Consequently, construction enterprises must construct sustainable competitive advantages; failure to do so will expose them to severe risk of marginalization amid intensifying market competition.

Existing Architecture, Engineering, and Construction (AEC) literature has examined construction enterprises' competitive strategies from perspectives including political relationships, social responsibility, Enterprise Resource Planning (ERP), digital transformation, and blockchain(H. Li, Y. Chang et al. 2024; Ram , J. et al. 2014;S. Wang et al. 2025;A. Kumar et al. 2025). However, existing research exhibits two significant limitations.

First, the majority of AEC literature focuses predominantly on traditional competitive advantages rather than sustainable competitive advantages, yet these concepts differ fundamentally. Traditional competitive advantages typically stem from external market opportunities(ref) or temporary resource allocations (ref). This exhibits replicability and temporal constraints that render them susceptible to displacement through competitor imitation or technological catch-up. They primarily emphasize static resource allocation efficiency. Rather, sustainable competitive advantages originate from enterprises' internal unique resource bundles and capability systems, possessing VRIN characteristics—valuable, rare, inimitable, and non-substitutable—enabling enterprises to generate superior profits over extended periods while resisting competitive pressures(ref). Moreover, sustainable competitive advantages emphasize the construction and evolution of dynamic capabilities—enterprises' capacity to sense environmental changes, seize market opportunities, and reconfigure resources. This fundamental distinction means that insights from traditional competitive advantage research may inadequately guide construction enterprises seeking long-term resilience and performance.需要去研究如何提升可持续竞争优势。。。我们去研究了竞争优势的前因变量。

Second, previous studies have predominantly analyzed the impact of individual factors on competitive advantage。 This literature fails to capture the complex, systemic nature of success in the construction industry. Competitive advantage in this sector rarely stems from a single “silver bullet”; rather, it emerges from the synergistic interplay among an enterprise's strategy, its environment, and its internal resources (Ref). This reality exposes the limitations of traditional linear models and necessitates a theoretical lens that can explain how different combinations of conditions lead to superior performance. The configurational school of strategic management provides such a lens, positing that an enterprise's advantage derives from the holistic fit of these multidimensional elements(DESS G G et al. 1993;MILLER D 1996;SHORT J C et al. 2008). By examining these combinations, or configurations, it becomes possible to understand why enterprises with similar resources in the same environment can exhibit vastly different outcomes. Therefore, this study adopts a configurational perspective to ask its first key question: From the multi-dimensional perspective of strategy, environment, and resources, what are the effective configurations for building a sustainable competitive advantage in the construction industry?

While sustainable competitive advantage determines construction enterprises' market position and profitability, organizational resilience ensures their survival and stability in turbulent environments. Organizational resilience refers to an enterprise's potential ability to anticipate, avoid, and respond to environmental shocks (especially crisis events)(SAJKO M et al. 2021), with its core characteristic being the stability and reliability of maintaining competitive advantage and business performance under crisis impacts. Unlike competitive advantage which focuses on exceptional operational performance, organizational resilience focuses on stable operational performance. Existing research demonstrates that organizationally resilient construction enterprises can better absorb shocks, maintain operational continuity, and recover more quickly from disruptions. For example, during the COVID-19 pandemic, construction enterprises with strong organizational resilience maintained project continuity through flexible resource reallocation and adaptive management practices, while less resilient competitors faced project suspensions and financial distress(Rufaidah A ,Amani Q B 2023). This divergence in crisis responses underscores the critical importance of organizational resilience in sustaining competitive advantage over time.

Existing research largely focuses separately on the singular dimensions of either organizational resilience or competitive advantage. For instance, Diamantopoulos et al. (2006) concentrated on functional resilience rather than systemic resilience(Diamantopoulos A et al. 2006); Burnard and Bhamra (2011), Williams et al. (2017) and other scholars provided organizational resilience frameworks(Burnard K et al. 2017), However, existing research largely focuses on either organizational resilience or competitive advantage as separate dimensions. The crucial interplay between them, particularly how resilience contributes to the sustainability of an advantage, remains underexplored. A truly sustainable advantage must be a resilient one; otherwise, it is merely a temporary high-performance state vulnerable to the next crisis.

The Resource-Based View (RBV) provides a theoretical foundation for understanding this relationship, positing that enterprises' sustained competitive advantage stems primarily from internal rather than external factors. From this perspective, organizational resilience capability can be viewed as a strategic internal resource that enhances sustainable competitive advantage. Organizational resilience enables enterprises to withstand pressure, continuously innovate, and rapidly adapt to changes, thereby protecting and reinforcing their competitive advantages during environmental turbulence. In the context of construction enterprises facing increasingly complex and volatile market environments characterized by supply chain disruptions, regulatory changes, and economic fluctuations, the interplay between organizational resilience and competitive advantage becomes particularly critical. Solely pursuing high competitive advantage without resilience leaves enterprises vulnerable to crisis-induced performance collapse, while focusing exclusively on resilience without competitive advantage results in stable mediocrity. Construction enterprises must therefore develop strategic configurations that simultaneously cultivate both organizational resilience and sustainable competitive advantage—enabling them to not only excel in stable periods but also maintain stability during crisis events and capitalize on post-crisis opportunities. Therefore, this study presents its second research question: How can enterprises simultaneously maintain organizational resilience while achieving high competitive advantage?

To address these two research questions, this study employs Time-Series Qualitative Comparative Analysis (TSQCA)(SAETRE A S et al. 2021). This configurational method was selected because traditional regression analysis, which focuses on the net effects of isolated variables, is ill-suited to explain how competitive advantage and resilience emerge from the complex, synergistic interplay of an enterprise's strategy, environment, and resources. TSQCA, in contrast, is designed to identify the multiple, distinct combinations of conditions that produce an outcome. By embracing the principles of conjunctural causation (outcomes stem from combinations of factors) and equifinality (multiple pathways lead to success), this approach provides a powerful tool for uncovering the different "recipes for success." In this study, TSQCA is used to determine which configurations of strategic, environmental, and organizational conditions are sufficient for achieving high sustainable competitive advantage and, subsequently, which of those configurations also ensure organizational resilience.

On this basis, this paper collected 2,150 data points from 118 listed construction enterprises in China from 2014 to 2023. It analyzes nine conditions across three major dimensions—strategic orientation (diversification strategy, differentiation strategy, cost leadership strategy), environmental characteristics (environmental dynamism, environmental richness), and organizational resources (cost stickiness, organizational size, social responsibility, digital transformation)—to determine which strategic configurations help construction enterprises gain sustainable competitive advantages and which of these can also help enterprises obtain organizational resilience in response to external uncertainties.
