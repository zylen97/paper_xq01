\section{Conclusion}
\label{sec:conclusion}
\subsection{Conclusion}
This study investigated the configurations of organizational resource, external environment, and strategic orientation that shape sustainable competitive advantage and organizational resilience in construction enterprises. By integrating NCA, TSQCA, and TCT within the "Resource-Environment-Strategy" framework, we analyzed the causal complexity underlying long-term value creation. The conclusions of this study are as follows:

First, the formation of sustainable competitive advantage follows the principle of equifinality, driven by eight distinct configurations rather than universal necessary conditions. These configurations reveal a polarized industry structure: the majority of enterprises adopt mainstream efficiency-driven pathways (e.g., Digital-Enabled Lean Scale or Diversification-Driven Cost Leadership) to navigate fierce price competition. In contrast, a minority of enterprises pursue niche pathways (e.g., Dual-Resource Driven Differentiation or Specialized Cost Leadership) by leveraging scarce resources such as superior ESG performance or high cost stickiness.

Second, there exists a pronounced trade-off between efficiency and organizational resilience. While mainstream efficiency-driven configurations maximize market coverage in stable and munificent environments, they are structurally vulnerable to exogenous shocks due to supply chain coupling and organizational inertia. Conversely, niche configurations, despite holding lower market shares, demonstrate high organizational resilience. Mechanisms such as the legitimacy conferred by ESG and the strategic slack provided by cost stickiness act as buffers, enabling these firms to maintain performance stability during crises like the COVID-19 pandemic.

\subsection{Limitations and future research}
This study has several limitations that pave the way for future research. First, the empirical setting is restricted to Chinese construction enterprises. Given the unique institutional characteristics of China's construction market, the generalizability of the "Resource-Environment-Strategy" framework to other cultural and economic contexts warrants validation. Future studies should replicate this configurational analysis in other emerging or developed economies to test cross-national applicability. Second, the sample primarily consists of listed enterprises, which tend to be large-scale and resource-abundant. Consequently, the identified configurations may not fully capture the survival logic of Small and Medium-sized Enterprises (SMEs), which operate under different resource constraints. Future research could expand the scope to include non-listed SMEs to explore distinct pathways for smaller players. Third, while this study examined nine key antecedents, other potential factors, such as top management team characteristics, open innovation ecosystems, or specific policy incentives, were not included due to data availability. Future scholars are encouraged to leverage other perspectives and include more antecedents to further refine the understanding of sustainable competitive advantage.