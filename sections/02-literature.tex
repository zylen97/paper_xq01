\section{Literature Review}
\label{sec:literature}

Two streams of literature are related to our research questions: (1) general management literature on sustainable competitive advantage, and (2) CEM literature regarding competitive advantage (CEM literatrue on sustainable competitive advantage is very limited \citep{toor2010Positive,betts1994Sustainablea}). This section reviews these studies to build our theoretical foundation. First, we synthesized the evolution of sustainable competitive advantage research to propose the "Resource-Environment-Strategy" framework as the overarching framework. Second, we reviewed the extant CEM literature regarding competitive advantage to contextualize this framework and identified nine potential antecedents of sustainable competitive advantage.

\subsection{"Resource-Environment-Strategy" framework from sustainable competitive advantage studies}

We derived the "Resource-Environment-Strategy" framework from a systematic review of the literature on sustainable competitive advantage. As shown in Table 1, related works have revealed a theoretical evolution. Scholars have progressively moved from static, single-factor explanations toward dynamic, multi-dimensional frameworks.

First, the theoretical discourse on sustainable competitive advantage originates from the resource-based view, which posits that the heterogeneous resources are the primary source of advantage \citep{abdeen2025Developing,zaman2025How}. However, the literature quickly evolved from focusing on tangible assets to more complex resource forms. For instance, \citet{hall1993FRAMEWORK} argued that sustainable advantage is derived primarily from intangible resources—such as reputation and employee know-how—because their causal ambiguity creates formidable barriers to imitation. Further, \citet{black1994STRATEGIC} critiqued the atomistic view of resources, introducing the concept of "resource networks". They posited that advantage arises not from isolated factors but from the complex configuration and complementary relationships between resources. As such, the literature has established organizational resources as the foundation for sustainable competitive advantage.

Second, building upon the resource-based view, several studies integrating institutional theory and market perspectives argues that resources cannot exist in a vacuum; rather, their value is contingent upon the external environment. \citet{oliver1997Sustainable} provided a seminal integration, demonstrating that a enterprise's advantage is shaped by the interaction between internal resources and external institutional pressures, where social legitimacy becomes a prerequisite for survival. Also, \citet{adner2006Demandbased} introduced a demand-side perspective, arguing that the sustainability of advantage is determined by market heterogeneity and consumer marginal utility rather than supply-side capabilities alone. Empirical work by \citet{mady2024Nexus} further confirms that external drivers, such as regulatory pressure and eco-friendly product demand, are critical forces that compel enterprises to adapt their resource base. Consequently, the literature identifies the external environment as the boundary condition.

Third, the scholarship also highlights that alignment between resources and the environment requires clear strategic orientation. Strategic orientation reflects the firm's proactive intent and the specific logic it employs to create and capture value \citep{sabug2020Competitive}. \citet{lado1992COMPETENCEBASED} were among the first to critique the static nature of resource possession, proposing a comprehensive model that prioritizes "managerial competencies" and "strategic focus". They argued that strategy acts as the engine that transforms input resources into competitive outputs, integrating environmental determinism with strategic choice. This view is supported by \citet{johannessen2003Knowledge}, who highlighted that sustainable competitive advantage is the result of conscious "strategic training" and management intervention. Moreover, \citet{malik2023Strategic} empirically demonstrated that strategic orientations—specifically Entrepreneurial Orientation and Market Orientation——act as essential mediators that leverage technological readiness to achieve sustainable competitive advantage, particularly in emerging markets. Additionally, \citet{pacheco-de-almeida2007Timing} framed strategy as a dynamic process of "timing," where the pace of resource development itself constitutes a critical strategic choice. Thus, the literature establishes strategic orientation as the dynamic mechanism that aligns internal resources with external opportunities.

While the literature has identified these three dimensions, prior studies have predominantly examined them in isolation or through linear "net-effect" models. \citet{rouse1999Rethinking} and \citet{levitas2002Rethinking} have debated the difficulties of isolating sources of advantage, pointing to a "black box" in understanding how these factors interact holistically. Current research on sustainable competitive advantage lacks a systematic analysis of how these three dimensions synergistically "fit" together to form high-performance configurations.

\subsection{Antecedents of sustainable competitive advantage in construction enterprises under "Resource-Environment-Strategy" framework}

Construction enterprises operate as complex, open systems characterized by project discontinuity, fragmented supply chains, and high sensitivity to institutional pressures \citep{ning2022How,wang2026Dynamic}. These distinctive industry attributes render the "Resource-Environment-Strategy" framework particularly pertinent for unraveling the causal complexity underlying the formation of sustainable competitive advantage. This section grounds the "Resource-Environment-Strategy" framework in the body of CEM literature regarding conventional competitive advantage. On this basis, we selected nine antecedents across the three dimensions to investigate the configurations leading to construction enterprises' sustainable competitive advantage.

\subsubsection{Organizational resource and sustainable competitive advantage}

In exploring the significance of organizational resources, the CEM literature regarding conventional competitive advantage has witnessed a theoretical evolution from resource-based view to dynamic capabilities view \citep{choi2018Dynamic}. The resource-based view posits that the possession and effective utilization of resources that are valuable, rare, inimitable, and non-substitutable are fundamental for enterprises to gain a competitive advantage \citep{barney2021Emergence,jugdev2019Mediated}. However, resource-based view primarily centers on the accumulation of internal assets, dynamic capabilities are formally defined as the potential to "integrate, build, and reconfigure internal and external competences to address rapidly changing environments" \citep{whang2024Multilevel}. Situated within these two theoretical lenses, this study identifies \textit{cost stickiness}, \textit{organizational size}, \textit{corporate social responsibility}, and \textit{digital transformation} as four critical resource-related antecedents.

\textbf{Cost stickiness.} Cost Stickness refers to the asymmetry where costs increase more rapidly with rising activity volume than they decrease during declines. In the project-based construction enterprises, cost stickiness reflects the deliberate retention of slack resources——such as skilled project managers and specialized technical equipment \citep{potgieter2016Maximizing}, acting as a resource investment from a long-term perspective \cite{luo2019Impacts}. Cost stickness enables construction enterprises to rapidly mobilize resources and stimulate innovation when new project opportunities emerge \citep{love2004Industrycentric}, which is specifically required for sustainable competitive advantage.

\textbf{Firm size.} Firm size functions as a critical indicator of resource endowment \citep{shao2025Competitive}. Large-scale enterprises typically possess abundant slack resources and lower financing costs, which provide a buffer against the high risks inherent in construction projects. Furthermore, consistent with \citet{maury2018Sustainable} who found that market share significantly predict profit persistence, large construction enterprises benefit from deep social embeddedness \citep{lello2024Professional}. Stakeholders, including governments and the social, tend to support these enterprises to ensure employment stability and infrastructure delivery, thereby enhancing their brand reputation. Thus, size determines the type of long-term resource advantages——stability versus flexibility——available to enterprises.

\textbf{Corporate social responsibility (CSR).} CSR is an intangible resource that transforms ethical behavior into assets \citep{wang2023Exploring}. Because construction projects have significant environmental and social footprints, CSR goes beyond compliance to become a mechanism for building trust and legitimacy. For construction enterprises, proactive CSR practices——such as ensuring site safety, engaging local communities, and utlizing environmental-friendly materials——can reduce friction with stakeholders and enhance corporate reputation \citep{hassan2016Organizational,locatelli2025Social}. As noted by \citet{mattera2022Sustainable}, commitment to sustainable business models and social responsibility contributes to a firm's ability to overcome crises and improve long-term financial performance.

\textbf{Digital transformation.} Digital transformation represents a "transformation-based competency" that fundamentally reconfigures a construction enterprise's operational resources \citep{wen2025Gap}. It involves integrating digital technologies (e.g., BIM, AI, IoT) into project lifecycles to enhance decision-making and efficiency \citep{simard2025Project}. \citet{probojakti2025Driving} found that digital transformation significantly improves organizational agility and resiliency, which are pivotal for sustaining competitive edges. Also, \citet{van2025Enhancing} emphasized that data-driven decision-making enabled by digital integration boosts organizational creativity and competitive advantage. By shifting from labor-intensive to technology-intensive processes, digital transformation allows construction enterprises to better sense environmental changes and seize new market opportunities, thereby securing a sustainable position.

\subsubsection{External environment and sustainable competitive advantage}

Construction enterprises operate as complex open systems where the sustainability of advantage is determined by how well internal capabilities align with external legitimacy requirements and market demands \citep{aghimien2023Dynamic,zhao2024Using}. Recent CEM studies suggested that the external environment is no longer static but characterized by rapid technological disruptions and fluctuating resource availability \citep{ning2022How}. Consequently, to capture the multidimensional nature of these external pressures and opportunities, this study identifies \textit{environmental dynamism} and \textit{environmental munificence} as the two critical environmental antecedents.

\textbf{Environmental dynamism.} Environmental dynamism refers to the rate and unpredictability of change in a firm's external environment, characterized by technological shifts, fluctuating market demands, and evolving regulations \citep{dess1984Dimensions}. In the construction industry, dynamism is currently driven by the "Fourth Industrial Revolution" and increasingly stringent sustainability mandates \citep{aghimien2023Dynamic}. High dynamism challenges the traditional static model of competitive advantage, as existing competencies can rapidly become obsolete. \citet{zhao2024Using} argue that in such transient competitive environments, advantages are easily eroded, compelling firms to continuously sense and seize new opportunities. Therefore, dynamism acts as a stressor that forces construction enterprises to shift from efficiency-based strategies to flexibility-based dynamic capabilities to sustain their market position \citep{ning2022How}.

\textbf{Environmental munificence.} Environmental munificence describes the extent to which an environment can support sustained growth, reflecting the abundance of critical resources and market opportunities \citep{chen2017Munificence}. For construction enterprises, this manifests as the availability of infrastructure projects, financial capital, and collaborative network support \citep{ma2025Unraveling}. A munificent environment provides necessary "slack resources," allowing firms to experiment with innovations and absorb failures without threatening survival. \citet{wang2024Strategic} suggest that firms in munificent environments (e.g., expanding international markets) can leverage diversified operations to capture emerging opportunities. Conversely, low munificence implies intense competition and resource scarcity. Thus, the level of munificence dictates the strategic "room for maneuver," influencing how construction enterprises acquire external support and social capital to construct sustainable advantages \citep{lello2024Professional,whang2024Understanding}.

\subsubsection{Strategic orientation and sustainable competitive advantage}

While \citet{porter1997COMPETITIVE}'s generic strategies have long served as a baseline, recent CEM literature suggests that sustainable competitive advantage emerges not from a single strategic posture but from the dynamic configuration of multiple orientations that match the firm's resource endowment with environmental demands \citep{shao2025Competitive}. To capture the diverse pathways through which construction enterprises position themselves, this study identifies \textit{diversification}, \textit{differentiation}, and \textit{cost leadership} as the three critical strategic antecedents.

\textbf{Diversification.} Diversification refers to the strategic expansion into new market segments or business lines to spread risks and capture emerging opportunities. For construction enterprises facing cyclical demand and intense local competition, diversification is a vital mechanism for survival and growth. \citet{wang2024Strategic} argued that "strategic resource amalgamation" is the driver of diversified operations, enabling contractors to leverage their operational and innovation capabilities across broader markets. By engaging in internationalization or expanding into related sectors (e.g., infrastructure investment or facility management), firms can buffer against environmental volatility and access new revenue streams. Thus, diversification represents a strategy of scope, allowing firms to exploit their existing resource base in novel environmental contexts to sustain competitive advantages \citep{whang2024Understanding}.

\textbf{Differentiation.} Differentiation involves creating a unique value proposition that distinguishes a firm’s products or services from competitors, thereby allowing for premium pricing or customer loyalty. In the CEM sector, differentiation is increasingly driven by "soft power" attributes such as corporate image, technical innovation, and brand reputation \citep{anjomshoa2024Key}. \citet{budayan2014Alignment} classify this into "quality and image-related differentiation," emphasizing that firms must align their project management processes with these strategic goals. Furthermore, in the era of digital transformation, differentiation is often achieved through the superior implementation of technologies like BIM. \citet{shao2025Competitive} found that image-oriented and quality-oriented competitive strategies are heavily emphasized by contractors to signal competence and secure legitimacy. Consequently, differentiation serves as a strategy of value, insulating firms from direct price competition.

\textbf{Cost Leadership.} Cost leadership remains a fundamental strategic orientation in the construction industry, characterized by the pursuit of the lowest operational costs to offer competitive pricing in bidding processes. While often viewed as a traditional approach, modern cost leadership transcends mere cost-cutting; it involves the rigorous pursuit of efficiency through lean management and technological integration \citep{li2024Lean}. \citet{sabug2020Competitive} highlighted that in competitive markets (e.g., rail sectors), a hybrid approach combining cost leadership with other strategies is often required for success. However, \citet{shao2025Competitive} note that cost-oriented strategies must be paired with effective learning strategies to drive innovation implementation (e.g., BIM) without compromising efficiency. Thus, cost leadership represents a strategy of efficiency, essential for maintaining the economic viability of projects in a low-margin industry \citep{li2025Impact}.