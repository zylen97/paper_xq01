\section{Discussion}
\label{sec:discussion}

\subsection{Discussion on research questions}
\subsubsection{Configurations that shape construction enterprises' sustainable competitive advantage (RQ1)}
Regarding RQ1, our findings reveal that the formation of sustainable competitive advantage is shaped by complex interactions of antecedents across dimensions of organizational resource, external environment, and strategic orientation. \textbf{\textit{Firstly}}, the NCA results indicate that no single condition constitutes a necessary condition for sustainable competitive advantage. This finding expands on some CEM literature that often elevated specific factors, such as BIM adoption or human capital, to the status of prerequisites for success \citep{probojakti2025Driving}. Instead, our results resonate with the principle of "equifinality", demonstrating that construction enterprises can achieve the same outcome through multiple, distinct pathways. For instance, while digital transformation is highly emphasized \citep{simard2025Project}, our NCA results suggest it is not necessary; enterprises can still achieve sustainable advantages through specialized cost leadership without heavy digital reliance (as seen in C4a), provided they possess other resources like cost stickiness.

\textbf{\textit{Secondly}}, TSQCA results indicate eight distinct configurations. The high consistency of these configurations validates the principle of "equifinality". There is no single "best way"; rather, there are multiple "orchestrations" of resources and strategies that match specific environments, echoing the configurational view that competitive advantage stems from complex resource networks \citep{black1994Strategic}. Further, POCOV results delineate the distribution of valid configurations. Configurations in Group C3 ("Diversification-Driven Cost Leadership") and Group C2 ("Digital-Enabled Lean Scale") exhibit the highest explanatory power. From the "Resource-Environment-Strategy" perspective, these high-coverage groups share a common strategic core of cost leadership, yet they achieve it through distinct mechanisms. Group C3 represents an "externally-oriented" efficiency model where enterprises, operating in munificent environments, leverage diversification strategies to integrate supply chains and internalize transaction costs, consistent with the logic of strategic resource amalgamation \citep{wang2024Strategic}. Conversely, Group C2 represents an "internally-oriented" efficiency model, where firms rely on massive organizational size and digital transformation to achieve economies of scale and optimize management efficiency. The dominance of these two groups suggests that efficiency-based cost leadership remains the mainstream logic for Chinese construction enterprises to build sustainable competitive advantage. This finding aligns with the industry's inherent characteristics of fierce price competition and thin profit margins \citep{das2021Developing,sharma2023Construction}. Consequently, for the majority of enterprises, the primary pathway to sustainability lies in leveraging resource endowments to ensure efficiency.

In contrast, Group C1 ("Dual-Resource Driven Differentiation") and Group C4 ("Specialized Cost Leadership") demonstrate significantly lower POCOV values. This disparity indicates that these configurations represent "elite" or "niche" pathways that are fewer in number but equally effective in outcome. Group C1 reflects a "high-end" pathway where enterprises combine superior ESG performance and digital capabilities to pursue differentiation. The lower coverage of this group suggests high barriers to entry, as it requires the accumulation of scarce, high-quality resources that average firms typically lack, particularly given the challenges of transforming ethical behaviors into strategic assets \citep{locatelli2025Social}. Similarly, Group C4 reflects a "specialized" pathway where enterprises reject diversification in favor of focusing on a niche market. The presence of cost stickiness in this configuration implies a reliance on asset specificity and long-term resource commitment, which acts as a strategic investment in specialized human capital \citep{luo2019Impacts}. The low coverage here indicates that this is a focused strategy suitable for "hidden champions" in specific technical sub-sectors, rather than a general model for the mass market. 

\subsubsection{High sustainable competitive advantage configurations that simultaneously maintain high organizational resilience (RQ2)}

Regarding RQ2, we identified which sustainable competitive advantage configurations can lead to high organizational resilience. Our investigation of temporal variations of BECOV in configurations during crisis reveals the answer: the configurations that are most dominant in stable times (Groups C2 and C3) appear the most vulnerable during crises. Specifically, the coverage of C2 and C3 declined sharply during shock periods. This finding empirically corroborates the "efficiency-resilience trade-off" discussed in recent literature, suggesting that the pursuit of absolute efficiency may inadvertently erode the capacity to withstand shocks \citep{shao2024Contradiction}. We argue that the very mechanisms driving the efficiency of these groups create structural rigidities. For Group C3, the reliance on complex, integrated supply chains—while reducing transaction costs in normal times—amplifies exposure to disruption risks. When external shocks cause widespread supply chain ruptures, the tight coupling between diversified segments transmits the shock across the entire enterprise, leading to a rapid loss of advantage \citep{zhang2024Deconstructing}. Similarly, for Group C2, the pursuit of economies of scale creates organizational inertia. While massive assets provide resource buffers, they also entail high fixed costs. In the face of sudden market contractions, these "heavy" enterprises struggle to pivot quickly, highlighting that focusing solely on competitive superiority may create relative vulnerability to contingencies \citep{lv2024Digital}.

On the contrary, Group C1 and Group C4 demonstrated remarkable stability, with minimal fluctuations in coverage during crises. This suggests that the pathways to resilience differ fundamentally from those to efficiency. For Group C1, the stability suggests that superior ESG performance acts as an "insurance mechanism". During crises, intangible assets such as corporate reputation and stakeholder trust become critical buffers. Enterprises in this group leverage their commitment to social responsibility to maintain legitimacy and reduce friction with stakeholders, thereby preserving their market position even when external conditions deteriorate \citep{mattera2022Sustainable}. Furthermore, the resilience of Group C4 offers a novel theoretical insight into the role of cost stickiness. While conventional accounting wisdom often views cost stickiness as a sign of inefficiency, our findings suggest that for specialized firms, high stickiness represents the deliberate retention of slack resources—such as skilled project teams and specialized equipment \citep{potgieter2016Maximizing}. Instead of laying off staff to cut costs during downturns, these enterprises retain their core human capital. This "resource redundancy" allows them to rapidly mobilize capabilities and recover functions as soon as the shock subsides \citep{love2004Industrycentric}.

\subsection{Theoretical implications}
This study suggests four theoretical implications for the literature. First, this study contributes to the CEM literature by systematically conceptualizing and empirically investigating sustainable competitive advantage in construction enterprises. While prior studies have extensively examined static competitiveness or isolated factors, this research represents a pioneering effort to unravel the "sustainable" dimension of advantage—specifically the durability of value creation. By adopting a configurational perspective, we demonstrate that sustainable competitive advantage in the project-based construction industry is not driven by single factors but by the "orchestration" of resources, environments, and strategies. This shift provides a more holistic perspective for understanding how construction firms maintain long-term survival amidst discontinuity.

Second, this study bridges the theoretical divide between sustainable competitive advantage and organizational resilience in the CEM domain. Existing literature often treats these two constructs in isolation. By integrating them, our study uncovers a critical "efficiency-resilience trade-off" specific to the construction sector. We provide theoretical nuance by revealing that the efficiency-driven configurations that dominate stable periods are structurally fragile to exogenous shocks, whereas niche configurations exhibit superior resilience. This insight challenges the assumption that high performance automatically equates to high resilience, enriching the theoretical understanding of organizational viability in turbulent environments.

Third, this study validates and contextualizes the "Resource-Environment-Strategy" framework within the unique setting of the construction industry. While this framework was derived from general management literature, its application in CEM has been limited. We extend its applicability by demonstrating that construction enterprises are complex open systems where internal resource endowments must be dynamically aligned with external environmental conditions through strategic orientations. By confirming the explanatory power of this framework in the construction industry, we provide a robust theoretical scaffold for future research on strategic management in project-based organizations.

Fourth, this study makes a methodological contribution by integrating TSQCA with TCT. While QCA is gaining traction in CEM research, it is often criticized for remaining a "black box" regarding causal mechanisms \citep{frateur2025How}. By employing TCT, we moved beyond merely identifying "what" certain configurations work to explaining "how" they work through the detailed reconstruction of evidence chains in typical enterprises. Our mixed-method approach offers a rigorous template for future empirical studies addressing causal complexity in the construction management field.

\subsection{Practical implications}

We provide practical insights for construction enterprise managers aiming to achieve sustainable competitive advantage and maintain organizational resilience as follows.

First, managers must recognize that there is no "one-size-fits-all" formula for long-term success; rather, they should tailor their managerial patterns to align with their specific resource endowments and environmental contexts. For leaders of large-scale enterprises, the "Digital-Enabled Lean Scale" pathway (as seen in Group C2) offers a guideline. Managers in such enterprises should prioritize digital transformation not merely as a technical upgrade, but as a strategic enabler to optimize cost leadership. For example, implementing centralized digital procurement platforms can leverage massive purchasing power to negotiate lower material prices, converting sheer size into genuine efficiency. Conversely, for specialized or smaller firms, blindly imitating the diversification strategies of giants is ill-advised. Instead, they should adopt the "Specialized Cost Leadership" model (Group C4). Managers should focus resources on a specific niche—such as tunnel engineering or green building technology—to build technical barriers that generalist competitors cannot penetrate.

Second, managers are required to make a strategic choice between maximizing efficiency in stable times and ensuring resilience during crises, as our results highlight a distinct trade-off. For enterprises pursuing the "mainstream" efficiency strategies (Groups C2 and C3), managers must be acutely aware of their structural fragility. While integrated supply chains drive profits in munificent environments, they become vulnerabilities during shocks. Therefore, managers adopting these strategies should establish proactive risk-addressing mechanisms, such as maintaining flexible backup suppliers rather than relying solely on lean, just-in-time delivery, to mitigate the losses due to contingencies.

Third, for managers seeking high organizational resilience, our findings regarding "niche" configurations (Groups C1 and C4) offer advice regarding resource allocation. Specifically, managers should rethink the value of ESG and Cost Stickiness. Regarding ESG, managers should view investments in social responsibility not as compliance burdens but as intangible assets. Proactive community engagement and environmental compliance can build a reservoir of goodwill that protects the firm's legitimacy when market sentiments turn negative. Regarding cost stickiness, managers of specialized firms should resist the pressure to immediately cut costs by laying off core technical staff during industry downturns. By retaining skilled project managers and R&D teams during recessions, managers effectively preserve the organization's recovery capacity, enabling the firm to seize opportunities faster than competitors once the market rebounds.