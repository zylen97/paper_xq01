\section{Results}
\label{sec:results}

\subsection{Necessary conditions analysis results}

Table \ref{tab:nca} presents the NCA results using both ceiling regression and ceiling envelopment techniques \citep{dul2016Necessary}. According to established NCA standards, a condition is considered necessary only if it meets two criteria simultaneously: the effect size is not less than 0.1 ($d \ge 0.1$), and the Monte Carlo simulation permutation test shows that the effect size is statistically significant ($p < 0.05$) \citep{dul2016Necessary}. The empirical data reveal that the effect sizes for eight out of the nine antecedent conditions are 0.000, indicating absolutely no bottleneck effect. The only exception is cost Stickiness, which exhibits a minor effect size ($d_{CE} = 0.077$, $d_{CR} = 0.039$). However, this value remains below the 0.1 threshold, and the permutation tests yield non-significant results for all conditions (p-values = 1.000). These statistics confirm that no single factor constitutes a necessary condition for achieving sustainable competitive advantage.

\begin{table}[!htbp]
    \centering
    \captionsetup{font=normalsize, labelsep=period}
    \setlength{\abovecaptionskip}{5pt}
    \setlength{\belowcaptionskip}{0pt}
    \caption{Results of Necessary Condition Analysis}
    \label{tab:nca}
    \small
    \begin{threeparttable}
    \begin{tabular*}{0.9\textwidth}{@{\extracolsep{\fill}}lcccccc}
    \toprule
    \textbf{Condition} & \textbf{Method} & \textbf{Consistency} & \textbf{Ceiling Zone} & \textbf{Coverage} & \textbf{Effect Size} & \textbf{p-value} \\
    \midrule
    \textit{diver} & CR & 1.000 & 0.000 & 1.000 & 0.000 & 1.000 \\
                   & CE & 1.000 & 0.000 & 1.000 & 0.000 & 1.000 \\
    \textit{diff}  & CR & 1.000 & 0.000 & 1.000 & 0.000 & 1.000 \\
                   & CE & 1.000 & 0.000 & 1.000 & 0.000 & 1.000 \\
    \textit{lead}  & CR & 1.000 & 0.000 & 1.000 & 0.000 & 1.000 \\
                   & CE & 1.000 & 0.000 & 1.000 & 0.000 & 1.000 \\
    \textit{dynam} & CR & 1.000 & 0.000 & 1.000 & 0.000 & 1.000 \\
                   & CE & 1.000 & 0.000 & 1.000 & 0.000 & 1.000 \\
    \textit{munif} & CR & 1.000 & 0.000 & 1.000 & 0.000 & 1.000 \\
                   & CE & 1.000 & 0.000 & 1.000 & 0.000 & 1.000 \\
    \textit{stick} & CR & 1.000 & 0.031 & 0.950 & 0.039 & 1.000 \\
                   & CE & 1.000 & 0.063 & 0.950 & 0.077 & 1.000 \\
    \textit{size}  & CR & 1.000 & 0.000 & 1.000 & 0.000 & 1.000 \\
                   & CE & 1.000 & 0.000 & 1.000 & 0.000 & 1.000 \\
    \textit{esg}   & CR & 1.000 & 0.000 & 1.000 & 0.000 & 1.000 \\
                   & CE & 1.000 & 0.000 & 1.000 & 0.000 & 1.000 \\
    \textit{digit} & CR & 1.000 & 0.000 & 1.000 & 0.000 & 1.000 \\
                   & CE & 1.000 & 0.000 & 1.000 & 0.000 & 1.000 \\
    \bottomrule
    \end{tabular*}
    \begin{tablenotes}[flushleft]
    \small\linespread{1}\selectfont
    \item \textit{Note}: CR = Ceiling Regression; CE = Ceiling Envelopment. Consistency values of 1.000 indicate perfect consistency. Effect sizes below 0.1 suggest negligible necessity.
    \end{tablenotes}
    \end{threeparttable}
    \end{table}
    \vspace{-15pt}

Furthermore, the bottleneck level analysis (see Table S1 in Supplementary Materials) corroborates these findings. While \textit{stick} shows a localized bottleneck level of 27.2\% strictly at the highest tier of performance (70\%--100\%), for all other performance levels and conditions, the bottleneck requirement is consistently "Not Necessary" (NN). This suggests that while maintaining slack resources (cost stickiness) may act as a threshold condition for achieving elite status, it is not a prerequisite for sustainable competitive advantage.

\subsection{Sufficient configuration results}
Standard QCA procedures yield three types of solutions: complex, parsimonious, and intermediate. Complex solutions often lead to many configurations that are difficult to interpret. Parsimonious solutions incorporate all logical remainders to maximize simplicity, potentially producing results that are detached from empirical reality. To address these limitations, this study adopts the intermediate solution. The intermediate solution strikes an optimal balance between the complexity of the complex solution and the over-simplification of the parsimonious solution \citep{ragin2006Set}. This approach ensures that the findings are both theoretically grounded and empirically interpretable, a practice particularly recommended for maintaining robustness in panel data QCA \citep{guedes2016UK}. Consequently, the analysis that follows focuses on the 8 configurations derived from the intermediate solution, as detailed in Table \ref{tab:tsqca}.

\subsubsection{Consistency Analysis}
Consistency serves as the primary criterion for evaluating configurations' validity. TSQCA assesses consistency through three metrics: pooled consistency (POCONS), between-case consistency (BECONS), and within-case consistency (WICONS) \citep{castro2016General}. We selected POCONS and BECONS for our analysis because our aim is to investigate the general patterns and temporal evolution trends across construction enterprises, rather than focusing on trajectories of individual case. 

POCONS assesses the strength of the sufficiency relationship between each configuration and high sustainable competitive advantage across the study period and all cases. As reported in Table \ref{tab:tsqca}, the POCONS values for all 8 configurations range from 0.893 to 0.950, exceeding the recommended threshold of 0.80 \citep{ragin2009Redesigning}. This indicates that all identified configurations act as valid and reliable sufficient conditions for achieving sustainable competitive advantage.

BECONS considers temporal effects by calculating the degree of sufficiency of condition combinations for the outcome variable based on data at each specific time point \citep{guedes2016UK}. Table \ref{tab:tsqca} suggests that annual BECONS values exhibit notable temporal variation throughout the study period. For instance, configurations C1b and C2a experience pronounced fluctuations, with BECONS values dropping to as low as 0.678 and 0.745 in 2017, before recovering in subsequent years. To quantify this volatility, we further computed the BECONS distance, which measures the deviation of annual consistency from the pooled consistency. Higher BECONS distances indicate more substantial temporal fluctuations. BECONS distances for all configurations exceed the 0.004 threshold \citep{castro2016General}, confirming significant temporal volatility ang the necessity of TSQCA. 

\begin{table}[!htbp]
\centering
\captionsetup{font=normalsize, labelsep=period}
\setlength{\abovecaptionskip}{5pt}
\setlength{\belowcaptionskip}{0pt}
\caption{TSQCA results}
\label{tab:tsqca}
\footnotesize
\begin{threeparttable}
\begin{tabular*}{\textwidth}{@{\extracolsep{\fill}}lcccccccc}
\toprule
\textbf{Conditions} & \textbf{C1a} & \textbf{C1b} & \textbf{C2a} & \textbf{C2b} & \textbf{C3a} & \textbf{C3b} & \textbf{C4a} & \textbf{C4b} \\
\midrule
\textit{stick} & {\Large $\otimes$} & {\Large $\otimes$} & {\Large $\otimes$} & & {\Large $\otimes$} & {\Large $\otimes$} & {\huge $\bullet$} & {\huge $\bullet$} \\
\textit{size} & & {\huge $\bullet$} & {\huge $\bullet$} & {\huge $\bullet$} & & {\Large $\otimes$} & & {\huge $\bullet$} \\
\textit{esg} & {\huge $\bullet$} & {\huge $\bullet$} & & {\Large $\otimes$} & {\huge $\bullet$} & & {\Large $\otimes$} & {\Large $\otimes$} \\
\textit{digit} & {\huge $\bullet$} & {\huge $\bullet$} & {\huge $\bullet$} & {\huge $\bullet$} & & & & {\huge $\bullet$} \\
\textit{dynam} & {\huge $\bullet$} & & {\huge $\bullet$} & {\Large $\otimes$} & {\Large $\otimes$} & {\huge $\bullet$} & {\huge $\bullet$} & {\Large $\otimes$} \\
\textit{munif} & & {\Large $\otimes$} & & {\huge $\bullet$} & {\huge $\bullet$} & {\huge $\bullet$} & & \\
\textit{diver} & {\huge $\bullet$} & {\Large $\otimes$} & {\Large $\otimes$} & & {\huge $\bullet$} & {\huge $\bullet$} & {\Large $\otimes$} & {\Large $\otimes$} \\
\textit{diff} & {\huge $\bullet$} & {\huge $\bullet$} & {\huge $\bullet$} & {\Large $\otimes$} & & {\Large $\otimes$} & {\huge $\bullet$} & \\
\textit{lead} & & & {\huge $\bullet$} & {\huge $\bullet$} & {\huge $\bullet$} & {\huge $\bullet$} & {\huge $\bullet$} & {\huge $\bullet$} \\
\midrule
POCONS & 0.893 & 0.873 & 0.910 & 0.925 & 0.884 & 0.905 & 0.918 & 0.875 \\
BECONS 2014 & 0.876 & 0.851 & 0.867 & 0.891 & 0.855 & 0.879 & 0.833 & 0.851 \\
BECONS 2015 & 0.813 & 0.832 & 0.846 & 0.803 & 0.811 & 0.826 & 0.834 & 0.821 \\
BECONS 2016 & 0.830 & 0.789 & 0.758 & 0.805 & 0.910 & 0.861 & 0.806 & 0.805 \\
BECONS 2017 & 0.857 & 0.828 & 0.845 & 0.876 & 0.857 & 0.893 & 0.878 & 0.876 \\
BECONS 2018 & 0.837 & 0.868 & 0.883 & 0.842 & 0.876 & 0.883 & 0.859 & 0.842 \\
BECONS 2019 & 0.919 & 0.938 & 0.944 & 0.937 & 0.942 & 0.937 & 0.941 & 0.927 \\
BECONS 2020 & 0.868 & 0.876 & 0.928 & 0.871 & 0.886 & 0.901 & 0.872 & 0.897 \\
BECONS 2021 & 0.931 & 0.908 & 0.924 & 0.939 & 0.922 & 0.919 & 0.929 & 0.895 \\
BECONS 2022 & 0.942 & 0.829 & 0.972 & 0.951 & 0.951 & 0.935 & 0.896 & 0.921 \\
BECONS 2023 & 0.962 & 0.969 & 0.952 & 0.960 & 0.961 & 0.949 & 0.946 & 0.952 \\
BECONS distance & 0.050 & 0.053 & 0.064 & 0.067 & 0.048 & 0.037 & 0.060 & 0.046 \\
\midrule
POCOV & 0.074 & 0.092 & 0.139 & 0.146 & 0.202 & 0.182 & 0.067 & 0.059 \\
BECOV SD & 0.022 & 0.025 & 0.037 & 0.034 & 0.058 & 0.052 & 0.024 & 0.015 \\
\bottomrule
\end{tabular*}
\begin{tablenotes}[flushleft]
\footnotesize\linespread{1}\selectfont
\item \textit{Note}: {\huge $\bullet$} = condition present; {\Large $\otimes$} = condition absent. POCONS = Pooled Consistency; BECONS = Between-Case Consistency; POCOV = Pooled Coverage; BECOV = Between-Case Coverage; SD = Standard Deviation. See Table S2 in Supplementary Materials for annual BECOV details.
\end{tablenotes}
\end{threeparttable}
\end{table}
\vspace{-15pt}

\subsubsection{Coverage Analysis}
Coverage assesses the empirical relevance and explanatory power of each configuration, indicating the proportion of the outcome set explained by a specific configuration. Variations in coverage values reflect changes in the explanatory strength of strategic configurations for achieving high sustainable competitive advantage. Specifically, coverage in panel data QCA comprises three metrics: pooled coverage (POCOV), between-case coverage (BECOV), and within-case coverage (WICOV). Following the analytical approach adopted in the consistency analysis, we first examine POCOV in this section and subsequently explore BECOV in the "Further Analysis" section.

As shown in Table \ref{tab:tsqca}, POCOV values vary across configurations. Notably, configurations C4b (0.237), C3a (0.232), C5a (0.231), and C2a (0.229) exhibit the highest coverage, suggesting they represent the most dominant patterns in the industry. A deeper inspection of these high-coverage configurations reveals a compelling commonality: they predominantly feature a combination of differentiation and cost leadership strategies (i.e., a hybrid strategy) synergized with key organizational resources like ESG performance and digital transformation. This underscores a critical trend in the contemporary Chinese construction market: pure strategies alone have limited explanatory power. Instead, the paradigm for success has shifted towards the integration of dual business advantages supplemented by modern organizational capabilities.

\subsection{Typical case tracing results and elaboration of configurations}
To further illustrate the identified configurations, we conducted typical case tracing and configuration explanation in three steps. First, we employed typical case tracing to identify typical cases representing configurations \cite{schneider2013Combining}. Second, we consolidated and labeled configurations into broader patterns based on the similarity and dissimilarity of their antecedent conditions, identifying four distinct configuration groups. Third, we conducted qualitative content analysis of the "Business Overview" and "Management Discussion and Analysis" sections from the annual reports of typical construction enterprises within the five consolidated patterns, developing theoretical propositions accordingly.

\subsubsection{Innovation-Responsibility Dual-Driven Configuration (Configurations C1a, C1b)}

C1a ($\sim \textit{stick} * \textit{esg} * \textit{digit} * \textit{dynam} * \textit{diver} * \textit{diff}$) and C1b ($\textit{size} * \textit{esg} * \textit{digit} * \sim \textit{munif} * \sim \textit{diver} * \textit{diff}$) represent a group of configurations we term the "Innovation-Responsibility Dual-Driven Configurations". The characteristic of these configurations is the simultaneous presence of differentiation strategy ($\textit{diff}$), superior ESG performance ($\textit{esg}$), and deep digital transformation ($\textit{digit}$). This configuration depicts an industry leader that transcends traditional low-cost competition by positioning technological innovation and social responsibility as differentiating advantages. 

According to the "principle of maximum set membership," \textbf{China State Construction Engineering Corporation (CSCEC)} is identified as the typical case for these configurations. Specificially, in terms of digital differentiation, CSCEC has developed proprietary "C-Smart" management platforms and independently controlled digital techiniques that define industry standards, such as the comprehensive application of digital twin technology in complex landmark structures \citep{cscec2024China}. Regarding ESG performance, CSCEC completed the construction of the Huoshenshan and Leishenshan hospitals within days, a feat made by the integration of prefabricated modular construction and an unwavering commitment to public health \citep{tan2021Integrated}. Furthermore, its aggressive promotion of green buildings and zero-carbon industrial parks serves as a tangible response to national "dual carbon" goals. 

Based on the "Resource-Environment-Strategy" framework, the formation mechanism of sustainable competitive advantage for this configuration follows a cumulative "Resource-Strategy" mutual reinforcement logic. Specifically, CSCEC integrates tangible digital capabilities with intangible legitimacy resources. This integration is not merely additive but synergistic: proprietary digital technologies (e.g., BIM, smart sites) provide precise data support for ESG management, while high-standard ESG mandates (e.g., "dual carbon" goals) conversely drive the demand for deep technological innovation. Building upon this dual-resource foundation, the enterprise successfully deployed a differentiation strategy to navigate environmental dynamism. Fig. 1 illustrates the configurational pathways of C1a and C1b.

\begin{center}[Insert Fig. 1 here]\end{center}

\subsubsection{Digital-Enabled Lean Scale Configurations (Configurations C2a, C2b)}
We term C2a ($\sim \textit{diver} * \textit{diff} * \textit{lead} * \textit{dynam} * \textit{size} * \textit{digit}$) and C2b ($\sim \textit{diff} * \textit{lead} * \textit{munif} * \textit{size} * \textit{digit}$) "Digital-Enabled Lean configurations". Case enterprise in these configurations leverages its immense organizational size ($\textit{size}$) as a foundation, executes a cost leadership strategy ($\textit{lead}$), and deploys deep digital transformation ($\textit{digit}$) as a critical catalyst. Unlike traditional contractors who rely on labor-intensive scale, these configurations utilize digital tools to convert static scale advantages into dynamic, inimitable efficiency advantages.

\textbf{China Communications Construction Company (CCCC)} stands out as the typical case. CCCC is an engineering contractor in ultra-large infrastructure (e.g., ports, roads, bridges), facing extreme pressure to control costs while managing vast assets. This configuration accurately reflects its strategic pivot: moving from "extensive expansion" to "lean management" via digitalization. Specifically, CCCC has successfully implemented this model through its digital supply chain and smart engineering. To support its cost leadership strategy, CCCC established a centralized digital procurement platform that leverages its massive scale ($\textit{size}$) to negotiate lower material prices. Furthermore, in landmark projects like the \textit{Hong Kong-Zhuhai-Macao Bridge}, CCCC utilized BIM in manufacturing of steel structures \citep{cccc2024China}. This digital approach minimized construction errors and rework, thereby achieving "lean construction". By integrating the Beidou satellite system into dual-carbon service platform \citep[p.~42]{cccc2024China}, CCCC further optimized operational efficiency under complex environmental conditions.

The formation of sustainable competitive advantage for this group follows a logic of "Scale Digitization $\rightarrow$ Efficiency Activation $\rightarrow$ Cost Barrier." Specifically, by embedding digital capabilities into a massive organizational size, the enterprise mitigates potential organizational inertia and enhances resource orchestration efficiency. This digital foundation facilitates the precise execution of a cost leadership strategy amidst environmental growth or volatility. Consequently, the synergistic integration of scale and digital precision identifies and eliminates operational waste, constructing a formidable and inimitable cost barrier that secures market share and sustains long-term performance. 

\begin{center}[Insert Fig. 2 here]\end{center}

\subsubsection{Diversification-Driven Cost Leadership Configurations (Configurations C3a, C3b)}

C3a ($\textit{diver} * \textit{diff} * \textit{lead} * \textit{munif} * \textit{esg}$) and C3b ($\textit{diver} * \sim \textit{diff} * \textit{lead} * \textit{munif} * \textit{dynam}$) depict configurations we named the "Diversification-Driven Cost Leadership Configurations". Enterprises in these configurations leverage abundant external opportunities ($\textit{munif}$) to construct a highly \textit{diversified} business mode ($\textit{diver}$) and accordingly reinforce a cost leadership advantage ($\textit{lead}$). Unlike simple conglomerate expansion, this group treats diversification as a strategic instrument to realize economies of scope and reduce transaction costs. 

\textbf{China Railway Group Limited (CREC)} serves as the case for these configurations. CREC's bueinsses span over railway, highway, bridge, building, industrial, ecological, and other construction. Its growth trajectory is intertwined with China's massive national infrastructure investment \citep{tan2019Rise}. CREC exemplifies the power of "diversification for efficiency" by establishing a full-chain layout encompassing survey, design, construction, and industrial manufacturing. A prominent example of this synergy is its industrial manufacturing sector: CREC is not only a global engineering contractor but also a leading manufacturer of high-value equipment, such as Tunnel Boring Machines (TBMs) and railway turnouts \citep{chinadaily2024BRI}. By incorporating these critical upstream manufacturing sectors into its diversified portfolio, CREC effectively internalizes high procurement costs and mitigates supply chain risks. This integration allows the enterprise to utilize proprietary, lower-cost, and reliable equipment in its construction projects.

This group of configurations follows a logic of "environmental support + diversified integration + cost barrier". Enterprises capitalize on environmental munificence, characterized by sustained large-scale national infrastructure investment, as a fertile foundation. Upon this basis, the enterprise pursues a diversification strategy not merely for revenue growth, but as a mechanism for vertical integration across the upstream and downstream supply chain. This strategic orchestration internalizes external market transactions (e.g., equipment manufacturing, project financing), thereby building a structural cost leadership advantage. Fig. 3 illustrates these configurations' logic.

\begin{center}[Insert Fig. 3 here]\end{center}

\subsubsection{Specialized Cost Leadership Archetype (Configurations C4a, C4b)}

We named C4a ($\sim \textit{diver} * \textit{diff} * \textit{lead} * \textit{dynam} * \textit{stick} * \sim \textit{munif} * \sim \textit{esg}$) and C4b ($\sim \textit{diver} * \textit{lead} * \textit{size} * \textit{stick} * \textit{digit} * \sim \textit{dynam} * \sim \textit{esg}$) as \textbf{"Specialized Cost Leadership Configurations".} Under these configurations, enterprises forgo diversification ($\sim \textit{diver}$) in favor of concentration on a specific niche market to pursue ultimate operational efficiency ($\textit{lead}$). A critical feature is the presence of cost stickiness ($\textit{stick}$). In this specific context, stickiness implies high "asset specificity" --- the deliberate retention of specialized human capital and R&D capabilities. This sustained resource commitment serves as foundations for technical dominance in a focused domain.

\textbf{Sinoma International Engineering Co., Ltd.} serves as a typical case. Unlike diversified conglomerates, Sinoma exhibits high strategic focus ($\sim \textit{diver}$), consistently channeling resources into its core business of cement technology and equipment. This focus allows the company to build a global moat \citep{sinafinance2025Sinoma}. Its high cost stickiness ($\textit{stick}$) reflects long-term investments in specialized assets and a highly professional talent pool. Even during industry downturns, the retention of these core technical teams creates "sunk costs" that competitors cannot easily replicate. Having constructed hundreds of cement production lines worldwide \citep{sinafinance2025Sinoma}, Sinoma has accumulated unmatched process data and achieved a steep "learning curve." This enables it to offer proprietary technologies—such as low-energy clinker calcination—that simultaneously deliver the industry's lowest operating costs and highest technical standards, effectively unifying technical differentiation with cost leadership.

Sustainable competitive advantage for this configuration follows a logic of "Asset Specificity + Niche Focus + Technical Moat." Specifically, cost stickiness is seen as a form of strategic resource accumulation. By maintaining high levels of investment in specialized R&D and personnel, the firm builds tacit knowledge. This resource base supports a focused strategy that targets specific market segments, allowing enterprises to achieve cost leadership through the economies of specialization rather than economies of scale. While competitors may enter the general market, they cannot replicate the efficiency and technological sophistication derived from the incumbent's long-term, specific asset accumulation in the niche. The logic of C4a and C4b is illustrated in Fig. 4.

\begin{center}[Insert Fig. 4 here]\end{center}

\subsection{High Sustainable Competitive Advantage Configurations with High Organizational Resilience}
Existing quantitative literature primarily operationalizes organizational resilience by measuring the capacity of an enterprise to maintain performance levels or minimize volatility following a crisis shock \citep{zhang2022Organizational,yao2025Clear}. Building upon this logic, and drawing on the set-theoretic discussions \citep{garciacastro2016General, ragin2006Set}, this study proposes a configurational approach to measure resilience. Specifically, we assess resilience by calculating the temporal variations in \textit{consistency} (validity) and \textit{coverage} (empirical relevance) of each configuration before and after specific crisis. A smaller fluctuation in coverage implies that the strategic configuration retains its empirical explanatory power and applicability despite external turbulence. Conversely, a sharp decline in coverage suggests that the configuration is fragile, leading cases to "drop out" of the high-performance set when facing disruptions. 

To empirically test this, we identified two specific crisis shocks and focused on the observation windows of 2014--2015 and 2019--2020. The first shock was the 2015 real state structural crisis. In 2015, the Chinese construction industry faced a significant cyclical crisis driven by structural adjustment of the real estate market. The newly started floor area of building construction plummeted to 1,066.51 million square meters, a year-on-year decrease of 14.6\% \citep{chinesenationalbureauofstatistics2016National}. Given the period's context of high-speed economic ascent, this contraction represented a severe exogenous shock. The second shock was the 2020 COVID-19 pandemic. This "black swan" event imposed a more abrupt test than the 2015 structural adjustment, causing widespread project suspensions, supply chain ruptures, and labor shortages due to lockdowns. 

We subsequently examined the changes in BECONS and BECOV for all configurations across these two crisis intervals. Regarding BECONS, the analysis reveals that the values for all configurations remained consistently above the 0.80 threshold during both crisis periods. This indicates that the core logic of these strategies did not fail; they remained valid sufficiency pathways for achieving competitive advantage even during crises. Regarding BECOV, however, the empirical relevance exhibited varying degrees of volatility. Figures 5 and 6 illustrate the BECOV trends for the two crisis periods, respectively. Specifically, four configurations in C2 and C3 exhibited a sharp decline in coverage during both crises. This suggests that strategies relying heavily on scale rigidity and complex diversified supply chains are more vulnerable to external shocks. In contrast, the configurations within C1 and C4 demonstrated relative stability, with significantly smaller declines in coverage. This conclusion is further corroborated by the BECOV standard deviations for Groups C2 and C3 that are markedly higher than those for Groups C1 and C4. This statistical evidence confirms that the innovation-driven and specialized configurations possess superior organizational resilience. 

\begin{center}[Insert Fig. 5 and Fig. 6 here]\end{center}

