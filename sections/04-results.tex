\section{Results}
\label{sec:results}

\subsection{Necessary conditions analysis results}

Table \ref{tab:nca} presents the NCA results using both ceiling regression and ceiling envelopment techniques \citep{dul2016Necessary}. The empirical data reveal that the effect sizes for all nine antecedent conditions fall below the threshold of 0.1. Moreover, the permutation tests yield non-significant results for all conditions (p-values = 1.00), indicating that no single factor constitutes a necessary condition for achieving sustainable competitive advantage. Furthermore, the bottleneck level analysis in Table 3 corroborates these findings, demonstrating that achieving any specific percentile of the outcome (ranging from 10\% to 100\%) does not require any antecedent condition to reach a specific minimum threshold.

\begin{table}[!htbp]
\centering
\captionsetup{font=normalsize, labelsep=period}
\setlength{\abovecaptionskip}{5pt}
\setlength{\belowcaptionskip}{0pt}
\caption{Results of Necessary Condition Analysis}
\label{tab:nca}
\small
\begin{threeparttable}
\begin{tabular*}{0.9\textwidth}{@{\extracolsep{\fill}}lcccccc}
\toprule
\textbf{Condition} & \textbf{Method} & \textbf{Consistency} & \textbf{Ceiling Zone} & \textbf{Coverage} & \textbf{Effect Size} & \textbf{p-value} \\
\midrule
\textit{diver} & CR & 1.000 & 0.000 & 1.000 & 0.000 & 1.000 \\
               & CE & 1.000 & 0.000 & 1.000 & 0.000 & 1.000 \\
\textit{diff}  & CR & 1.000 & 0.000 & 1.000 & 0.000 & 1.000 \\
               & CE & 1.000 & 0.000 & 1.000 & 0.000 & 1.000 \\
\textit{lead}  & CR & 1.000 & 0.000 & 1.000 & 0.000 & 1.000 \\
               & CE & 1.000 & 0.000 & 1.000 & 0.000 & 1.000 \\
\textit{dynam} & CR & 1.000 & 0.000 & 1.000 & 0.000 & 1.000 \\
               & CE & 1.000 & 0.000 & 1.000 & 0.000 & 1.000 \\
\textit{munif} & CR & 1.000 & 0.000 & 1.000 & 0.000 & 1.000 \\
               & CE & 1.000 & 0.000 & 1.000 & 0.000 & 1.000 \\
\textit{stick} & CR & 1.000 & 0.031 & 0.950 & 0.039 & 1.000 \\
               & CE & 1.000 & 0.063 & 0.950 & 0.077 & 1.000 \\
\textit{size}  & CR & 1.000 & 0.000 & 1.000 & 0.000 & 1.000 \\
               & CE & 1.000 & 0.000 & 1.000 & 0.000 & 1.000 \\
\textit{esg}   & CR & 1.000 & 0.000 & 1.000 & 0.000 & 1.000 \\
               & CE & 1.000 & 0.000 & 1.000 & 0.000 & 1.000 \\
\textit{digit} & CR & 1.000 & 0.000 & 1.000 & 0.000 & 1.000 \\
               & CE & 1.000 & 0.000 & 1.000 & 0.000 & 1.000 \\
\bottomrule
\end{tabular*}
\begin{tablenotes}[flushleft]
\small\linespread{1}\selectfont
\item \textit{Note}: CR = Ceiling Regression; CE = Ceiling Envelopment. Consistency values of 1.000 indicate perfect consistency. Effect sizes below 0.1 suggest negligible necessity.
\end{tablenotes}
\end{threeparttable}
\end{table}
\vspace{-15pt}

These findings provide a critical theoretical insight: they effectively refute "monocausal" explanations for success in the Chinese construction industry. The results demonstrate that achieving a high level of sustainable competitive advantage does not depend on any single attribute—neither a specific strategy (e.g., diversification) nor a particular resource (e.g., digitalization) individually serves as a prerequisite. This absence of universal necessary conditions validates the core premise of this study: success is not driven by isolated factors but by the synergistic orchestration of multiple ingredients. Consequently, these results provide a robust empirical foundation for the subsequent sufficiency analysis, justifying the focus on configurational pathways rather than net-effect relationships.