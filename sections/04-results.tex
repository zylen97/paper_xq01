\section{Results}
\label{sec:results}

\subsection{Necessary conditions analysis results}

Table \ref{tab:nca} presents the NCA results using both Ceiling Regression (CR) and Ceiling Envelopment (CE) techniques. The empirical data reveal that the effect sizes for all nine antecedent conditions—spanning strategic orientations, environmental characteristics, and organizational resources—fall substantially below the threshold of 0.1. Moreover, the permutation tests yield non-significant results for all conditions (p-values = 1.00), indicating that no single factor constitutes a necessary condition for achieving sustainable competitive advantage. Furthermore, the bottleneck level analysis in Table 3 corroborates these findings, demonstrating that achieving any specific percentile of the outcome (ranging from 10\% to 100\%) does not require any antecedent condition to reach a specific minimum threshold.

These findings provide a critical theoretical insight: they effectively refute "monocausal" explanations for success in the Chinese construction industry. The results demonstrate that achieving a high level of sustainable competitive advantage does not depend on any single attribute—neither a specific strategy (e.g., diversification) nor a particular resource (e.g., digitalization) individually serves as a prerequisite. This absence of universal necessary conditions validates the core premise of this study: success is not driven by isolated factors but by the synergistic orchestration of multiple ingredients. Consequently, these results provide a robust empirical foundation for the subsequent sufficiency analysis, justifying the focus on configurational pathways rather than net-effect relationships.