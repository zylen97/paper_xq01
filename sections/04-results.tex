\section{Results}
\label{sec:results}

\subsection{Necessary conditions analysis results}
Table \ref{tab:nca} presents the NCA results using ceiling regression and ceiling envelopment techniques \citep{dul2016Necessary}. A condition is considered necessary only if it meets two criteria simultaneously: the effect size is not less than 0.1 ($d \ge 0.1$), and the Monte Carlo simulation permutation test shows that the effect size is statistically significant ($p < 0.05$) \citep{dul2016Necessary}. The empirical data reveal that the effect sizes for eight out of the nine antecedent conditions are 0.000. Cost stickiness exhibits a minor effect size (0.039, 0.077); however, the value remains below the 0.1 threshold, and the permutation tests yield non-significant results for all conditions (p-values = 1.000). These statistics confirm that no single factor constitutes a necessary condition for sustainable competitive advantage. Furthermore, the bottleneck level analysis (see Table S1 in Supplementary Materials) corroborates these findings. While \textit{stick} shows a localized bottleneck level of 27.2\% strictly at the highest tier of performance (70\%--100\%), the bottleneck requirement is consistently "Not Necessary" (NN) for all other levels and conditions. This suggests that while cost stickiness acts as a threshold condition for achieving elite status, it is not a prerequisite for sustainable competitive advantage.

\begin{table}[!t]
\centering
\captionsetup{font=normalsize, labelsep=period}
\setlength{\abovecaptionskip}{2pt}
\setlength{\belowcaptionskip}{0pt}
\caption{Results of necessary condition analysis}
\label{tab:nca}
\small
\begin{threeparttable}
\begin{tabular*}{0.85\textwidth}{@{\extracolsep{\fill}}lcccccc}
\toprule
\textbf{Condition} & \textbf{Method} & \textbf{Consistency} & \textbf{Ceiling Zone} & \textbf{Coverage} & \textbf{Effect Size} & \textbf{p-value} \\
\midrule
\textit{diver} & CR & 1.000 & 0.000 & 1.000 & 0.000 & 1.000 \\
                & CE & 1.000 & 0.000 & 1.000 & 0.000 & 1.000 \\
\textit{diff}  & CR & 1.000 & 0.000 & 1.000 & 0.000 & 1.000 \\
                & CE & 1.000 & 0.000 & 1.000 & 0.000 & 1.000 \\
\textit{lead}  & CR & 1.000 & 0.000 & 1.000 & 0.000 & 1.000 \\
                & CE & 1.000 & 0.000 & 1.000 & 0.000 & 1.000 \\
\textit{dynam} & CR & 1.000 & 0.000 & 1.000 & 0.000 & 1.000 \\
                & CE & 1.000 & 0.000 & 1.000 & 0.000 & 1.000 \\
\textit{munif} & CR & 1.000 & 0.000 & 1.000 & 0.000 & 1.000 \\
                & CE & 1.000 & 0.000 & 1.000 & 0.000 & 1.000 \\
\textit{stick} & CR & 1.000 & 0.031 & 0.810 & 0.039 & 1.000 \\
                & CE & 1.000 & 0.063 & 0.810 & 0.077 & 1.000 \\
\textit{size}  & CR & 1.000 & 0.000 & 1.000 & 0.000 & 1.000 \\
                & CE & 1.000 & 0.000 & 1.000 & 0.000 & 1.000 \\
\textit{esg}   & CR & 1.000 & 0.000 & 1.000 & 0.000 & 1.000 \\
                & CE & 1.000 & 0.000 & 1.000 & 0.000 & 1.000 \\
\textit{digit} & CR & 1.000 & 0.000 & 1.000 & 0.000 & 1.000 \\
                & CE & 1.000 & 0.000 & 1.000 & 0.000 & 1.000 \\
\bottomrule
\end{tabular*}
\begin{tablenotes}[flushleft]
\small\linespread{1}\selectfont
\item \textit{Note}: CR = Ceiling Regression; CE = Ceiling Envelopment. Consistency values of 1.000 indicate perfect consistency. Effect sizes below 0.1 suggest negligible necessity.
\end{tablenotes}
\end{threeparttable}
\end{table}
\vspace{-15pt}

\subsection{Configuration analysis results}
QCA yields three types of solutions: complex, parsimonious, and intermediate. Complex solutions often lead to many configurations that are difficult to interpret. Parsimonious solutions incorporate all logical remainders to maximize simplicity, potentially producing results that are detached from empirical reality. To address these limitations, this study adopts the intermediate solution. The intermediate solution strikes an optimal balance between the complex solution and the parsimonious solution \citep{ragin2006Set}. This approach ensures that the findings are both theoretically grounded and empirically interpretable, a practice particularly recommended for maintaining robustness in panel data QCA \citep{guedes2016UK}. Consequently, eight configurations were identified from the intermediate solution, as detailed in Table \ref{tab:tsqca}.

\subsubsection{Consistency analysis}
Consistency serves to evaluate configurations' validity. TSQCA assesses consistency through three metrics: pooled consistency (POCON), between-case consistency (BECON), and within-case consistency (WICON) \citep{castro2016General}. We selected POCON and BECON for our analysis because we aim to investigate the general patterns and temporal evolution trends across construction enterprises, rather than focusing on trajectories of individual cases. 

POCON assesses the strength of the sufficiency relationship between each configuration and high sustainable competitive advantage across the study period. As reported in Table \ref{tab:tsqca}, the POCON values for all 8 configurations range from 0.873 to 0.925, exceeding the recommended threshold of 0.80 \citep{ragin2009Redesigning}. This indicates that all identified configurations act as valid and reliable sufficient conditions for achieving sustainable competitive advantage.

BECON considers temporal effects by calculating the degree of sufficiency of condition combinations for the outcome variable based on data at each specific time point \citep{guedes2016UK}. Table \ref{tab:tsqca} suggests that annual BECON values exhibit notable temporal variation. To quantify this volatility, we further computed the BECON distance, which measures the deviation of annual consistency from the pooled consistency. BECON distances for all configurations exceed the 0.004 threshold \citep{castro2016General}, confirming significant temporal volatility and the necessity of TSQCA. 

\begin{table}[!htb]
\centering
\captionsetup{font=normalsize, labelsep=period}
\setlength{\abovecaptionskip}{0pt}
\setlength{\belowcaptionskip}{0pt}
\caption{TSQCA results}
\label{tab:tsqca}
\small
\begin{threeparttable}
\begin{tabular*}{\textwidth}{@{\extracolsep{\fill}}lcccccccc}
\toprule
\textbf{Conditions} & \textbf{C1a} & \textbf{C1b} & \textbf{C2a} & \textbf{C2b} & \textbf{C3a} & \textbf{C3b} & \textbf{C4a} & \textbf{C4b} \\
\midrule
\textit{stick} & {\Large $\otimes$} & & {\Large $\otimes$} & & {\Large $\otimes$} & {\Large $\otimes$} & {\huge $\bullet$} & {\huge $\bullet$} \\
\textit{size} & & {\huge $\bullet$} & {\huge $\bullet$} & {\huge $\bullet$} & & {\Large $\otimes$} & & {\huge $\bullet$} \\
\textit{esg} & {\huge $\bullet$} & {\huge $\bullet$} & & {\Large $\otimes$} & {\huge $\bullet$} & & {\Large $\otimes$} & {\Large $\otimes$} \\
\textit{digit} & {\huge $\bullet$} & {\huge $\bullet$} & {\huge $\bullet$} & {\huge $\bullet$} & & & & {\huge $\bullet$} \\
\textit{dynam} & {\huge $\bullet$} & & {\huge $\bullet$} & {\Large $\otimes$} & {\Large $\otimes$} & {\huge $\bullet$} & {\huge $\bullet$} & {\Large $\otimes$} \\
\textit{munif} & & {\Large $\otimes$} & & {\huge $\bullet$} & {\huge $\bullet$} & {\huge $\bullet$} & & \\
\textit{diver} & {\huge $\bullet$} & {\Large $\otimes$} & {\Large $\otimes$} & & {\huge $\bullet$} & {\huge $\bullet$} & {\Large $\otimes$} & {\Large $\otimes$} \\
\textit{diff} & {\huge $\bullet$} & {\huge $\bullet$} & {\huge $\bullet$} & {\Large $\otimes$} & & {\Large $\otimes$} & {\huge $\bullet$} & \\
\textit{lead} & & & {\huge $\bullet$} & {\huge $\bullet$} & {\huge $\bullet$} & {\huge $\bullet$} & {\huge $\bullet$} & {\huge $\bullet$} \\
\midrule
POCON & 0.893 & 0.873 & 0.910 & 0.925 & 0.884 & 0.905 & 0.918 & 0.875 \\
BECON 2014 & 0.876 & 0.851 & 0.867 & 0.891 & 0.855 & 0.879 & 0.833 & 0.851 \\
BECON 2015 & 0.813 & 0.832 & 0.846 & 0.803 & 0.811 & 0.826 & 0.834 & 0.821 \\
BECON 2016 & 0.830 & 0.789 & 0.758 & 0.805 & 0.910 & 0.861 & 0.806 & 0.805 \\
BECON 2017 & 0.857 & 0.828 & 0.845 & 0.876 & 0.857 & 0.893 & 0.878 & 0.876 \\
BECON 2018 & 0.837 & 0.868 & 0.883 & 0.842 & 0.876 & 0.883 & 0.859 & 0.842 \\
BECON 2019 & 0.919 & 0.938 & 0.944 & 0.937 & 0.942 & 0.937 & 0.941 & 0.927 \\
BECON 2020 & 0.868 & 0.876 & 0.928 & 0.871 & 0.886 & 0.901 & 0.872 & 0.897 \\
BECON 2021 & 0.931 & 0.908 & 0.924 & 0.939 & 0.922 & 0.919 & 0.929 & 0.895 \\
BECON 2022 & 0.942 & 0.829 & 0.972 & 0.951 & 0.951 & 0.935 & 0.896 & 0.921 \\
BECON 2023 & 0.962 & 0.969 & 0.952 & 0.960 & 0.961 & 0.949 & 0.946 & 0.952 \\
BECON distance & 0.050 & 0.053 & 0.064 & 0.067 & 0.048 & 0.037 & 0.060 & 0.046 \\
\midrule
POCOV & 0.074 & 0.092 & 0.139 & 0.146 & 0.202 & 0.182 & 0.067 & 0.059 \\
BECOV SD & 0.022 & 0.025 & 0.037 & 0.034 & 0.058 & 0.052 & 0.024 & 0.015 \\
\bottomrule
\end{tabular*}
\begin{tablenotes}[flushleft]
\small\linespread{1}\selectfont
\item \textit{Note}: $\bullet$ = condition present; $\otimes$ = condition absent. POCON = Pooled Consistency; BECON = Between-Case Consistency; POCOV = Pooled Coverage; BECOV = Between-Case Coverage; SD = Standard Deviation. See Table S2 in Supplementary Materials for annual BECOV details.
\end{tablenotes}
\end{threeparttable}
\end{table}
\vspace{-20pt}

\vspace{15pt}
\subsubsection{Coverage analysis}
Coverage assesses the explanatory power of each configuration, indicating the proportion of the outcome set explained by a specific configuration. Variations in coverage values reflect changes in the explanatory strength of configurations across the study period. Specifically, coverage in TSQCA comprises three metrics: pooled coverage (POCOV), between-case coverage (BECOV), and within-case coverage (WICOV) \citep{castro2016General}. Following the analytical approach adopted in the consistency analysis, we examined POCOV and BECOV. As shown in Table \ref{tab:tsqca}, POCOV values vary across configurations. Notably, configurations C3a (0.202), C3b (0.182), and C2a (0.146) exhibit the highest coverage, suggesting that they represent the most dominant patterns to shape sustainable competitive advantage. The results of BECOV are shown in Section \ref{subsec:resilience}. 

\subsection{Typical case tracing results and elaboration of configurations}
We employed three steps to elaborate on the identified configurations. First, we consolidated and labeled configurations into broader patterns based on their similarity and dissimilarity. Four distinct configuration groups were established. Second, we utilized TCT to trace typical cases representing configurations \citep{schneider2013Combining}. Third, we developed theoretical pathways based on four groups of configurations and their typical enterprise cases.

\subsubsection{Dual-resource driven differentiation configurations (C1a, C1b)}
C1a ($\sim \textit{stick} * \textit{esg} * \textit{digit} * \textit{dynam} * \textit{diver} * \textit{diff}$) and C1b ($\textit{size} * \textit{esg} * \textit{digit} * \sim \textit{munif} * \sim \textit{diver} * \textit{diff}$) represent a group of configurations we term the "Dual-Resource Driven Differentiation Configurations". The characteristic of these configurations is the simultaneous presence of differentiation strategy ($\textit{diff}$), superior ESG performance ($\textit{esg}$), and deep digital transformation ($\textit{digit}$). This configuration depicts an industry leader that transcends traditional low-cost competition by positioning technological innovation and social responsibility as differentiating advantages. 

According to the "principle of maximum set membership," \textbf{China State Construction Engineering Corporation (CSCEC)} is identified as the typical case for these configurations. Specifically, CSCEC has developed proprietary "C-Smart" management platforms and independently controlled digital techniques that define industry standards, such as the comprehensive application of digital twin technology in complex landmark structures \citep{cscec2024China}. Regarding ESG performance, CSCEC delivered Huoshenshan and Leishenshan hospitals within days, a feat made possible by the commitment to public health \citep{tan2021Integrated}. Furthermore, its promotion of green buildings and zero-carbon industrial parks serves as a tangible response to national "dual carbon" goals. 

The formation mechanism of sustainable competitive advantage for this configuration follows a cumulative "Resource-Strategy" mutual reinforcement logic. Specifically, CSCEC integrates tangible digital capabilities with intangible legitimacy resources. Proprietary digital technologies provide data support for ESG management, while high-standard ESG requirements conversely drive the demand for technological innovation. On this basis, the enterprise deployed a differentiation strategy to navigate environmental dynamism. Fig. 3 illustrates these configurations' pathways.

\begin{center}[Insert Fig. 3 here]\end{center}

\subsubsection{Digital-enabled lean scale configurations (C2a, C2b)}
We term C2a ($\sim \textit{stick} * \textit{size} * \textit{digit} * \textit{dynam} * \sim \textit{diver} * \textit{diff} * \textit{lead}$) and C2b ($\textit{size} * \sim \textit{esg} * \textit{digit} * \sim \textit{dynam} * \textit{munif} * \sim \textit{diff} * \textit{lead}$) "Digital-Enabled Lean configurations". The case enterprise in these configurations leverages its immense organizational size ($\textit{size}$) as a foundation, executes a cost leadership strategy ($\textit{lead}$), and deploys digital transformation ($\textit{digit}$). Enterprises in these configurations utilize digital tools to convert static scale advantages into dynamic, inimitable efficiency advantages.

\textbf{China Communications Construction Company (CCCC)} stands out as the typical case. CCCC is an engineering contractor in ultra-large infrastructure, facing extreme pressure to control costs while managing vast assets. This configuration accurately reflects its strategic pivot: moving from "extensive expansion" to "lean management" via digitalization. Specifically, CCCC has successfully implemented digitalization through its digital supply chain and smart engineering. To support its cost leadership strategy, CCCC established a centralized digital procurement platform that leverages its massive scale ($\textit{size}$) to negotiate lower material prices. Furthermore, in landmark projects like the \textit{Hong Kong-Zhuhai-Macao Bridge}, CCCC utilized BIM in manufacturing of steel structures \citep{cccc2024China}. By integrating the Beidou satellite system into dual-carbon service platform \citep[p.~42]{cccc2024China}, CCCC optimized operational efficiency under complex environmental conditions.

The formation of sustainable competitive advantage for this group follows a logic of "Scale Digitization $\rightarrow$ Efficiency Activation $\rightarrow$ Cost Barrier." Specifically, by embedding digital capabilities into a massive organizational size, the enterprise enhances resource orchestration efficiency. Consequently, the synergistic integration of scale and digital precision constructs an inimitable cost barrier that sustains long-term performance. Fig. 4 illustrates the configurational pathways.

\begin{center}[Insert Fig. 4 here]\end{center}

\subsubsection{Diversification-driven cost leadership configurations (C3a, C3b)}
C3a ($\sim \textit{stick} * \textit{esg} * \sim \textit{dynam} * \textit{munif} * \textit{diver} * \textit{lead}$) and C3b ($\sim \textit{stick} * \sim \textit{size} * \textit{dynam} * \textit{munif} * \textit{diver} * \sim \textit{diff} * \textit{lead}$) depict configurations we named the "Diversification-Driven Cost Leadership Configurations". Enterprises leverage abundant external opportunities ($\textit{munif}$) to construct a highly \textit{diversified} business mode ($\textit{diver}$) and accordingly reinforce a cost leadership advantage ($\textit{lead}$). Unlike simple conglomerate expansion, this group treats diversification as a strategic instrument to realize economies of scope and reduce transaction costs. 

\textbf{China Railway Group Limited (CREC)} serves as the case for these configurations. The growth trajectory of CREC is intertwined with China's massive national infrastructure investment \citep{tan2019Rise}. CREC exemplifies the power of "diversification for efficiency" by establishing a full-chain layout encompassing survey, design, construction, and industrial manufacturing. For instance, CREC is not only a global engineering contractor but also a leading manufacturer of high-value equipment, such as Tunnel Boring Machines (TBMs) and railway turnouts \citep{chinadaily2024BRI}. By incorporating these critical upstream manufacturing sectors into its diversified portfolio, CREC effectively internalizes high procurement costs and mitigates supply chain risks. 

This group follows a logic of "environmental support + diversified integration + cost barrier". CREC capitalizes on environmental munificence, characterized by sustained large-scale national infrastructure investment, as a fertile foundation. Upon this basis, the enterprise pursues a diversification strategy not merely for revenue growth, but as a mechanism for vertical integration across the upstream and downstream supply chain. This strategic orchestration internalizes external market transactions, thereby building a cost advantage. Fig. 5 illustrates these configurations' logic.

\begin{center}[Insert Fig. 5 here]\end{center}

\subsubsection{Specialized cost leadership configurations (C4a, C4b)}
We named C4a ($\textit{stick} * \sim \textit{esg} * \textit{dynam} * \sim \textit{diver} * \textit{diff} * \textit{lead}$) and C4b ($\textit{stick} * \textit{size} * \sim \textit{esg} * \textit{digit} * \sim \textit{dynam} * \sim \textit{diver} * \textit{lead}$) as "Specialized Cost Leadership Configurations". Under these configurations, enterprises forgo diversification ($\sim \textit{diver}$) in favor of a specific niche market to pursue operational efficiency ($\textit{lead}$). Cost stickiness ($\textit{stick}$) here implies high "asset specificity" --- the deliberate retention of specialized human capital and R\&D capabilities. This sustained resource commitment serves as foundations for technical dominance in a focused domain.

\textbf{Sinoma International Engineering Co., Ltd.} serves as a typical case. Unlike diversified conglomerates, Sinoma exhibits high strategic focus ($\sim \textit{diver}$), consistently channeling resources into its core business of cement technology \citep{sinafinance2025Sinoma}. Its high cost stickiness ($\textit{stick}$) reflects long-term investments in specialized assets. Even during industry downturns, the retention of these core technical teams creates "sunk costs" that competitors cannot easily replicate. Hundreds of cement production lines worldwide \citep{sinafinance2025Sinoma} enable Sinoma to offer proprietary technologies, such as low-energy clinker calcination, for simultaneously achieving the industry's lowest operating costs and highest technical standards, effectively unifying technical differentiation with cost leadership.

Sustainable competitive advantage for this configuration follows a logic of "Asset Specificity + Niche Focus + Technical Moat." By maintaining high levels of investment in specialized R&D and personnel, the firm builds tacit knowledge and the moat. These bases support a focused strategy that targets specific market segments, allowing enterprises to achieve cost leadership through the economies of specialization rather than economies of scale. While competitors may enter the general market, they cannot replicate the efficiency and technological sophistication. The logic of C4a and C4b is illustrated in Fig. 6.

\begin{center}[Insert Fig. 6 here]\end{center}

\subsection{High sustainable competitive advantage configurations with organizational resilience}
\label{subsec:resilience}
Existing literature often operationalizes organizational resilience by measuring the capacity of an enterprise to maintain performance levels or minimize volatility following a crisis shock \citep{zhang2022Organizational,yao2025Clear}. Building upon this logic, and drawing on the set-theoretic discussions \citep{castro2016General, ragin2006Set}, this study proposes a configurational approach to measure resilience. Specifically, we assessed resilience by investigating the temporal variations in \textit{consistency} and \textit{coverage} before and after specific crises. A smaller fluctuation in coverage implies that the strategic configuration retains its empirical explanatory power and applicability despite external turbulence. 

To empirically test this, we identified two specific crisis shocks and focused on the observation windows of 2014--2015 and 2019--2020. The first shock was the 2015 real estate structural crisis. In 2015, the Chinese construction industry faced a cyclical crisis driven by adjustment of the real estate market. The newly started floor area of building construction plummeted to 1,066.51 million square meters, a year-on-year decrease of 14.6\% \citep{chinesenationalbureauofstatistics2016National}. Given the period's context of high-speed economic ascent, this contraction represented a severe exogenous shock. The second shock was the 2020 COVID-19 pandemic. This "black swan" event imposed a more abrupt test than the 2015 structural adjustment, causing widespread project suspensions, supply chain ruptures, and labor shortages due to lockdowns \citep{zhang2024Deconstructing}. 

We examined the changes in BECON and BECOV for all configurations across these two crises. Regarding BECON, the analysis reveals that BECON values for all configurations remained consistently above 0.80 during both crisis periods. They remained valid sufficiency pathways for achieving competitive advantage even during crises. Regarding BECOV, configurations exhibited varying degrees of volatility (See Table S2 in Supplementary Materials). Figures 7 and 8 illustrate the BECOV trends for the two crisis periods, respectively. Four configurations in C2 and C3 exhibited a sharp decline in coverage during both crises, suggesting that configurations relying heavily on cost leadership and complex diversified supply chains are more vulnerable to external shocks. In contrast, the configurations within C1 and C4 demonstrated relative stability. This conclusion is further corroborated by the BECOV standard deviations. BECOV standard deviations for Groups C2 and C3 are markedly higher than those for Groups C1 and C4. 

\begin{center}[Insert Figs. 7 and 8 here]\end{center}
