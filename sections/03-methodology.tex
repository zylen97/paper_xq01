\section{Model Building}

\subsection{The Baseline Two-Player Evolutionary Game Model}


We first establish a baseline evolutionary game model to analyze the strategic interactions between two horizontal governments during a disaster emergency. The model focuses on the decision-making process regarding cooperation on resource sharing, which includes both relief supplies and critical information. It assumes that the governments are boundedly rational and dynamically adjust their strategies based on the payoffs from previous interactions. This baseline game model does not include a higher-level (vertical) government. The baseline model is built upon the following key assumptions:

\textbf{Assumption 1}. The game involves two players, i.e., two governments at the same administrative level. The first player is the Local Government (LG), which represents the government whose jurisdiction is primarily affected by the disaster and is in need of assistance. The second player is the Neighboring Government (NG), which represents the government of an adjacent region that possesses surplus resources and can offer aid.

\textbf{Assumption 2}. Each player has a strategy set of \{Cooperate (C), Not Cooperate (NC)\}. Let $x$ be the probability that the LG chooses C, and $(1-x)$ be the probability it chooses NC, where $x \in [0, 1]$. Similarly, let $y$ be the probability that the NG chooses C, and $(1-y)$ be the probability it chooses NC, where $y \in [0, 1]$.

\textbf{Assumption 3}. The players are not perfectly rational; instead, they learn and adapt their strategies over time based on the relative success of past choices.

\textbf{Assumption 4}. The rescue benefit derived from relief supplies follows an ``S''-shaped function, which realistically captures the marginal utility of resources, from scarcity to abundance. The function is defined as:
\begin{equation}
    F(\theta) = \frac{c}{1+e^{-a\theta+b}}
\end{equation}
where $\theta = X/D$ represents the material satisfaction rate (the ratio of allocated supplies $X$ to demand $D$), and $a$, $b$, $c$ are benefit coefficients.

\textbf{Assumption 5} (Local Government's Strategic Considerations). When choosing to cooperate with the Neighboring Government, the LG can obtain additional relief supplies through regional coordination. When both LG and NG actively cooperate, both governments incur a cooperation cost $H$, and the LG gains public credibility $G_L$ for its collaborative efforts. According to the Interim Measures for the Management of Central Emergency and Disaster Relief Material Reserves (ref), following the principle of "user pays," the LG bears the transportation cost for the shared supplies. In this simplified model, we assume the transportation cost is proportional to the quantity of supplies transferred, expressed as $T = k(X_L - Q_L)$, where $k$ represents the per-unit transportation cost. Through supply sharing, the LG's per-capita rescue benefit $F_L$ exceeds what would be achieved without cooperation. Considering benefit distribution, the LG compensates the NG at a per-unit market price $m$, resulting in a coordination payment of $m(X_L - Q_L)$. Cooperation also involves information sharing, where the NG shares disaster situation data and resource information at a certain sharing rate, helping the LG improve emergency prediction and pre-deployment, thereby reducing potential costs and generating benefit $P_L$. When only the LG is willing to cooperate, it still incurs a unilateral cooperation cost $H_L$. When only the NG cooperates, the NG proactively shares information at rate $\alpha_N$, allowing the LG to obtain corresponding benefits.

\textbf{Assumption 6} (Neighboring Government's Strategic Considerations). The Neighboring Government's strategy space similarly consists of \{Cooperate, Not Cooperate\}. This analysis focuses on scenarios where the NG's disaster demand $D_N$ does not exceed its emergency reserve $Q_N$, meaning it has surplus supplies available to assist the LG. Given this surplus capacity, the NG must evaluate multiple factors including cooperation benefits, costs, and potential risks when making its decision. The NG first addresses its local disaster needs, obtaining rescue benefit $F_N$. Through cooperation, the NG receives coordination compensation $m(X_L - Q_L)$, information sharing benefit $\alpha P_N$, and public credibility $G_N$. However, it must also bear cooperation costs and consider potential losses from providing aid to the LG, which is primarily related to the quantity of coordinated supplies $(X_L - Q_L)$ and the per-unit potential loss $W$. When only the NG is willing to cooperate, it incurs a unilateral cooperation cost $H_N$. When only the LG cooperates, the LG shares information at rate $\alpha_L$.

\textbf{Parameters and Variables}

The parameters used in the baseline model are defined as follows:

\begin{table}[h]
\centering
\begin{tabular}{ll}
\hline
\textbf{Symbol} & \textbf{Definition} \\
\hline
\multicolumn{2}{l}{\textit{Government-Specific}} \\
$D_L$, $D_N$ & Demand for relief supplies for LG and NG, respectively \\
$Q_L$, $Q_N$ & Quantity of relief supplies initially possessed by LG and NG, respectively \\
$X_L$ & Total quantity of supplies available to LG after receiving aid from NG \\
      & The amount of aid is $(X_L - Q_L)$ \\
$G_L$, $G_N$ & The gain in public credibility for LG and NG from cooperative actions \\
\multicolumn{2}{l}{\textit{Costs}} \\
$H$ & Cost incurred by each government when both choose C \\
$H_L$, $H_N$ & Cost incurred by the willing party in a unilateral cooperation scenario \\
$T$ & Total transportation cost for the relief supplies, borne by the LG \\
$k$ & Per-unit transportation cost \\
$W$ & Per-unit potential loss for the NG for sharing its supplies \\
    & (e.g., risk of facing its own subsequent shortages) \\
\multicolumn{2}{l}{\textit{Benefits \& Payoffs}} \\
$F_L(\cdot)$, $F_N(\cdot)$ & The S-shaped benefit function for rescue effectiveness for LG and NG \\
$m$ & The per-unit compensation benefit paid by LG to NG for the provided supplies \\
$P_L$, $P_N$ & The benefit generated from information sharing for LG and NG, respectively \\
$\alpha$ & The information sharing rate when both governments choose C \\
$\alpha_L$, $\alpha_N$ & The information sharing rate when only LG or NG is willing to cooperate, respectively \\
\hline
\end{tabular}
\end{table}

\textbf{Payoff Matrix}

Based on the parameters above, the payoff matrix for the two-player game is constructed as follows:

\begin{table}[h]
\centering
\begin{tabular}{lcc}
\hline
 & \multicolumn{2}{c}{\textbf{Neighboring Government (NG)}} \\
\textbf{Local Government (LG)} & \textbf{C ($y$)} & \textbf{NC ($1-y$)} \\
\hline
\textbf{C ($x$)} & 
\begin{tabular}{@{}c@{}}
$D_L F_L\left(\frac{X_L}{D_L}\right) + \alpha P_L + G_L$ \\
$- (X_L - Q_L)(k+m) - H$, \\
$D_N F_N(1) + \alpha P_N + G_N$ \\
$+ (m-W)(X_L - Q_L) - H$
\end{tabular} & 
\begin{tabular}{@{}c@{}}
$D_L F_L\left(\frac{Q_L}{D_L}\right) + G_L - H_L$, \\
$D_N F_N(1) + \alpha_L P_N$
\end{tabular} \\
\textbf{NC ($1-x$)} & 
\begin{tabular}{@{}c@{}}
$D_L F_L\left(\frac{Q_L}{D_L}\right) + \alpha_N P_L$, \\
$D_N F_N(1) + G_N - H_N$
\end{tabular} & 
\begin{tabular}{@{}c@{}}
$D_L F_L\left(\frac{Q_L}{D_L}\right)$, \\
$D_N F_N(1)$
\end{tabular} \\
\hline
\end{tabular}
\end{table}

\noindent\textit{Note}: In each cell, the first entry is the payoff for the Local Government (LG), and the second is the payoff for the Neighboring Government (NG).

\textbf{Replicator Dynamics Equations}

The evolution of the strategies within the LG and NG populations is modeled by the following replicator dynamics equations:

\textbf{Replicator Dynamics Equation for the Local Government (LG):}
\begin{equation}
    F_L(x,y) = \frac{dx}{dt} = x(1-x)(E_x - E_{1-x})
\end{equation}
\begin{multline}
    = x(1-x)\Bigl(G_L - H_L + y\bigl(D_L F_L\left(\frac{X_L}{D_L}\right) - D_L F_L\left(\frac{Q_L}{D_L}\right) \\
    + (\alpha-\alpha_N)P_L - (X_L-Q_L)(k+m) - H + H_L\bigr)\Bigr)
\end{multline}

\textbf{Replicator Dynamics Equation for the Neighboring Government (NG):}
\begin{equation}
    F_N(x,y) = \frac{dy}{dt} = y(1-y)(E_y - E_{1-y})
\end{equation}
\begin{equation}
    = y(1-y)\left(G_N - H_N + x\left((\alpha-\alpha_L)P_N + (m-W)(X_L-Q_L) - H + H_N\right)\right)
\end{equation}

These equations describe the rate of change of the proportion of players adopting the C strategy in each population, forming the basis for analyzing the system's evolutionary stable strategies (ESS).