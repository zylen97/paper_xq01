% !TEX program = xelatex
% 这行告诉LaTeX编辑器使用XeLaTeX引擎编译(支持更好的字体处理)

% Elsarticle文档类声明:review格式,12pt字体,Harvard引用风格
\documentclass[authoryear,preprint,review,times,12pt]{elsarticle}

% ===== 宏包导入部分 =====
% 数学相关宏包(必备!工程管理论文经常用到数学公式)
\usepackage{amsmath}      % 强大的数学环境,如align、equation等
\usepackage{amssymb}      % 数学符号,如∈、⊆、∀、∃等
\usepackage{amsfonts}     % 数学字体,如粗体向量、花体字母等

% 基础功能宏包(elsarticle已包含graphicx)
\usepackage{booktabs}     % 制作专业表格(\toprule、\midrule、\bottomrule)
\usepackage{enumitem}     % 增强列表功能(itemize、enumerate)
\usepackage{lineno}       % 行号宏包(review格式需要)

% 页边距调整(草稿阶段使用,投稿时可注释掉)
\usepackage{geometry}
\geometry{
    left=2.5cm,    % 左边距(原约3.5cm)
    right=2.5cm,   % 右边距(原约3.5cm)
    top=2.5cm,     % 上边距(原约3cm)
    bottom=2.5cm   % 下边距(原约3cm)
}

\usepackage{hyperref}     % 超链接和PDF书签(应该放在最后导入)

% 指定目标期刊(根据实际投稿期刊修改)
\journal{Engineering Management Journal}

\begin{document}

\begin{frontmatter}

% 论文标题
\title{Emergency response strategy and simulation analysis considering inter-government coordination and information sharing}

% 作者信息(使用elsarticle格式)
\author[label1]{dongjie}
\ead{zylenw97@usts.edu.cn}

\author[label1]{Yitong Chen}
\ead{yitongchen@usts.edu.cn}

\author[label2]{Zhe Zhang\texorpdfstring{\corref{cor1}}{*}}
\ead{2431220@tongji.edu.cn}

\author[label2]{Qinghua He}
\ead{heqinghua@tongji.edu.cn}


\cortext[cor1]{Corresponding author}

\affiliation[label1]{organization={School of Civil Engineering, Suzhou University of Science and Technology},
            addressline={}, 
            city={Suzhou},
            postcode={215011}, 
            state={},
            country={China}}

\affiliation[label2]{organization={School of Economics and Management, Tongji University},
            addressline={}, 
            city={Shanghai},
            postcode={200092}, 
            state={},
            country={China}}

% 摘要
\begin{abstract}
This paper investigates the application of game-theoretic approaches to engineering management problems, focusing on multi-agent optimization in complex systems. We develop a mathematical framework that combines Nash equilibrium concepts with optimization theory to model decision-making processes in distributed engineering environments. Our computational experiments demonstrate significant improvements in system efficiency and resource allocation. The proposed methodology provides both theoretical foundations and practical implementation strategies for modern engineering management challenges.
\end{abstract}

% 关键词(使用elsarticle格式)
\begin{keyword}
Game theory \sep Multi-agent systems \sep Engineering management \sep Optimization \sep Nash equilibrium
\end{keyword}

\end{frontmatter}

% 启用行号(review格式特有)
\linenumbers

% 目录页(期刊论文通常不需要)
% \tableofcontents
% \newpage

% ===== 正文各章节 =====
% 使用 \input 命令导入各章节文件

\section{Introduction}
\label{sec:introduction}

The construction industry is undergoing a transformation, challenging the efficiency of construction enterprises' traditional, often fragmented, project-based business modes \citep{dang2025Assessing}. On the one hand, conventional operational pressures exist, including rising material costs and skilled labor shortages \citep{aghimien2022Dynamic}. On the other hand, disruptive external requirements, such as digitalization and sustainability, are intensifying these pressures \citep{wen2025Gap}. Digitalization from building information modeling (BIM) or robotics can mitigate labor shortages; however, it requires substantial capital investment and implementation across projects, which in turn reduces enterprises' short-term profit margins \citep{shao2025Competitive}. Sustainability mandates, such as net-zero regulations \citep{locatelli2025Social}, also introduce new compliance costs and operational complexities \citep{wang2023Exploring}. These trends create a high-velocity environment where conventional enterprises' advantages, such as a successful project bid or isolated technology adoption, are no longer sufficient to ensure long-term performance. A practical problem facing construction enterprises' managers is how to develop \textbf{\textit{sustainable competitive advantage}} --- a set of difficult-to-imitate edges \citep{adner2006Demandbased} --- to ensure growth within contexts of project-based business.

Sustainable competitive advantage, coined by \citet[p.~102]{barney1991Firm}, denotes a firm's capacity to "\textit{implement a value creating strategy not simultaneously being implemented by any current or potential competitors and when these other firms are unable to duplicate the benefits of this strategy}". Developing sustainable competitive advantage is critical for construction enterprises due to their inherent discontinuity and fragmentation of operations \citep{betts1994Sustainablea}. Unlike continuous manufacturing, construction operations are project-based and temporary, often leading to the loss of knowledge and efficiency when project teams disband \citep{sydow2018Projects}. In this context, sustainable competitive advantage improves organizational continuity, allowing enterprises to transfer technological innovations and management expertise across projects \citep{malik2023Green}. Specifically, the construction industry is historically dominated by fierce price competition and thin profit margins \citep{sharma2023Construction}. Rare and non-substitutable edges enable enterprises to differentiate themselves beyond low bid competition \citep{deng2014Developing}. Thus, clarifying the formation of construction enterprises' sustainable competitive advantage is imperative to the construction engineering and management (CEM) literature for understanding how to ensure long-term survival.

Existing CEM literature has largely investigated the antecedents of construction enterprises' competitive advantage in three aspects. First, research identifies critical resource-based factors, such as resource amalgamation \citep{wang2024Strategic}, dynamic capability \citep{ning2022How}, and human capital \citep{sarihi2020Multiskilled}, as the foundation of market position. Second, studies on strategy-based antecedents explore specific business modes, including internationalization \citep{jang2020Classifying}, networking \citep{lello2024Professional}, and projectization \citep{barbosa2024Multilevel}. Third, works on innovation-based antecedents highlight how digital technologies like BIM \citep{shao2025Competitive} and green innovation techniques \citep{dang2025Assessing} offer new avenues for sustainable development.

Despite these advancements, \textbf{\textit{two gaps remain in the CEM literature}} that hinder a comprehensive understanding of the formation of sustainable competitive advantage. \textbf{\textit{First}}, CEM literature pays most attention to static or conventional competitive advantage rather than sustainable competitive advantage. Related studies treated competitive advantage as a direct, immediate outcome of achieving profitability \citep{li2025Impact}, possessing specific business resources \citep{wang2024Strategic}, or adopting new tools. In contrast, sustainable competitive advantage emphasizes the formation of difficult-to-imitate advantages through continuous resource integration and capability enhancement \citep{abdeen2025Strategic}. Although some CEM studies have discussed sustainable competitive advantage, they failed to capture its core attributes. For instance, \citet{toor2010Positive} conceptualized sustainable competitive advantage as a derivative of leadership. \citet[p.~45]{betts1994Sustainablea} equated it with the critical success factors of construction enterprises. These studies overlooked the very nature of sustainable competitive advantage, rendering it insufficient to guide construction enterprises' managers. \textbf{\textit{Second}}, competitive advantage research is fragmented, mostly examining antecedents from isolated perspectives. A substantial body of literature relies on a "net-effect" logic to assess the specific impact of single variables, such as the implementation of BIM, project management tools and techniques \citep{li2025Impact}, or specific market segment drivers. In reality, construction enterprises do not strictly respond to single environmental stimuli; rather, they "orchestrate" antecedents to navigate challenges \citep{black1994Strategic}. While some recent studies have adopted a configurational perspective \citep{wang2024Strategic,shao2025Competitive}, they have not explicitly targeted competitive advantage as their subjects. CEM literature lacks an integrative, configurational perspective to systematically unravel how different antecedents interact and combine to jointly develop sustainable competitive advantage. Therefore, \textbf{\textit{the first aim of this study is to investigate the configurations of factors that shape construction enterprises' sustainable competitive advantage.}} 

While sustainable competitive advantage is crucial for sustained growth, organizational resilience serves as an important safeguard for survival and stability in turbulent environments \citep{yang2024What}. Organizational resilience refers to an enterprise's capacity to anticipate, absorb, and respond to contingencies \citep{zhang2024Deconstructing}. Despite systematic examinations of organizational resilience in CEM literature \citep{zhang2022Organizational}, few studies have integrated resilience with sustainable competitive advantage. Sustainable competitive advantage and organizational resilience represent two distinct yet mutually reinforcing dimensions of long-term viability. Sustainable competitive advantage primarily targets the sustainability of superior market position and inimitable performance, while organizational resilience prioritizes stable operations and continuity, specifically in crisis events \citep{shao2024Contradiction}. Focusing solely on competitive superiority may create vulnerability to crises, particularly in the highly uncertain post-pandemic era \citep{lv2024Digital}. This disconnection hinders our understanding of how construction enterprises can ensure long-term market dominance with the capability of addressing contingencies. Consequently, \textbf{\textit{the second aim of this study is to examine which configurations shaping sustainable competitive advantage simultaneously enable a high level of organizational resilience.}}

To bridge these knowledge gaps, we developed a "Resource-Environment-Strategy" framework to investigate configurations of nine antecedents across dimensions of organizational resource, external environment, and strategic orientation. This framework is derived from the literature on sustainable competitive advantage and is further grounded in the body of CEM research. This study utilizes a panel dataset including 1,061 observations from 115 listed Chinese construction enterprises spanning 2014-2023. We employed necessary condition analysis, time-series qualitative comparative analysis, and typical case tracing to examine how configurations of nine antecedents relate to sustainable competitive advantage and organizational resilience. Therefore, \textbf{\textit{this study answers two research questions (RQs):}}

\textit{RQ1: What are the configurations of factors across dimensions of organizational resource, external environment, and strategic orientation that shape construction enterprises' sustainable competitive advantage?}

\textit{RQ2: Which of these configurations simultaneously allow construction enterprises to maintain a high level of organizational resilience, specifically in crisis events?}

The remainder of this paper is organized as follows. Section 2 reviews the related works and establishes the "Resource-Environment-Strategy" framework. Section 3 details the research methods. Section 4 presents the empirical results. Section 5 discusses the key findings, elaborating on research implications. Section 6 concludes the study with limitations for future research.
\section{Literature review and theoretical foundation}
\label{sec:literature}

Three streams of literature are related to our research questions: (1) works on sustainable competitive advantage, (2) CEM literature regarding competitive advantage, and (3) CEM literature on organizational resilience. This section reviews these studies and builds the theoretical foundation. First, we synthesized the evolution of sustainable competitive advantage research to propose the "Resource-Environment-Strategy" framework. Second, we reviewed extant CEM literature regarding competitive advantage to contextualize the framework and identified nine potential antecedents. Third, we reviewed organizational resilience studies to clarify its relationship with sustainable competitive advantage.

\subsection{"Resource-Environment-Strategy" framework from sustainable competitive advantage studies}

We derived the "Resource-Environment-Strategy" framework from a review of the literature on sustainable competitive advantage. Distinct from conventional competitive advantage, sustainable competitive advantage emphasizes the \textit{durability} of superior performance and the \textit{inimitability} of value-creating capabilities against competitive duplication \citep{oliver1997Sustainable,lado1992CompetencyBased}. Scholars have progressively investigated the antecedents from resource, environment, and strategy perspectives.

Sustainable competitive advantage originates from the resource-based view, which posits that the heterogeneous resources are the primary source of advantage \citep{abdeen2025Strategic}. However, the literature evolved from focusing on tangible assets to more complex resource forms. For instance, \citet{hall1993FRAMEWORK} argued that sustainable advantage is derived primarily from intangible resources—such as reputation and employee know-how—because their causal ambiguity creates formidable barriers to imitation. Further, \citet{black1994Strategic} critiqued the atomistic view of resources, introducing the concept of "resource networks". They posited that advantage arises not from isolated factors but from the complementary relationships between resources.

Building upon the resource-based view, several studies integrating institutional theory and market perspectives argue that resources cannot exist in a vacuum; rather, their value is contingent upon the external environment. \citet{oliver1997Sustainable} provided a seminal integration, demonstrating that an enterprise's advantage is shaped by the interaction between internal resources and external institutional pressures, where social legitimacy becomes a prerequisite for survival. Also, \citet{adner2006Demandbased} indicated that the sustainability of advantage is determined by market heterogeneity and consumer marginal utility rather than supply-side capabilities alone. Empirical work by \citet{mady2024Nexus} further confirmed that external drivers, such as regulatory pressure and eco-friendly product demand, are critical forces that compel enterprises to adapt their resource base.

The scholarship also highlights that alignment between resources and the environment requires clear strategic orientation. Strategic orientation reflects the firm's proactive intent and the specific logic it employs to create and capture value \citep{sabug2020Competitive}. \citet{lado1992CompetencyBased} were among the first to propose a comprehensive model that prioritizes "managerial competencies" and "strategic focus". They argued that strategy acts as the "engine" that transforms input resources into competitive outputs, integrating environmental determinism with strategic choice. This view is supported by \citet{johannessen2003Knowledge}, who highlighted that sustainable competitive advantage is the result of conscious "strategic training" and management intervention. Moreover, \citet{malik2023Green} empirically demonstrated that strategic orientations act as essential mediators that leverage technological readiness to achieve sustainable competitive advantage, particularly in emerging markets.

While the literature has identified these three dimensions, prior studies have predominantly examined them in isolation. \citet{rouse1999Rethinking} and \citet{levitas2002Rethinking} debated the difficulties of isolating sources of advantage, pointing to a "black box" in understanding how these factors interact holistically. On this basis, we propose the "Resource-Environment-Strategy" framework to investigate the configurations of construction enterprises' sustainable competitive advantage.

\subsection{Antecedents of sustainable competitive advantage in construction enterprises}
Construction enterprises are characterized by project discontinuity, fragmented supply chains, and high sensitivity to institutional pressures \citep{ning2022How}. These attributes render the "Resource-Environment-Strategy" framework pertinent for unraveling the causal complexity of sustainable competitive advantage. This section grounds the framework in the CEM literature regarding conventional competitive advantage and identifies nine antecedents accordingly.

\subsubsection{Organizational resource and sustainable competitive advantage}
CEM literature regarding conventional competitive advantage has witnessed a theoretical evolution from the resource-based view to the dynamic capabilities view \citep{choi2018Dynamic}. Resource-based view primarily centers on the accumulation of internal resources that are valuable, rare, inimitable, and non-substitutable \citep{barney2021Emergence}; rather, dynamic capabilities are defined as the potential to "\textit{integrate, build, and reconfigure internal and external competences to address rapidly changing environments}" \citep{whang2024Understanding}. Situated within these two theoretical lenses, this study identifies \textit{cost stickiness}, \textit{organizational size}, \textit{ESG}, and \textit{digital transformation} as four resource-related antecedents.

\textbf{Cost stickiness.} Cost stickiness refers to the asymmetry where costs increase more rapidly with rising activity volume than they decrease during declines. Cost stickiness in construction enterprises reflects the deliberate retention of slack resources, such as skilled project managers and specialized technical equipment \citep{potgieter2016Maximizing}, acting as a resource investment from a long-term perspective \citep{luo2019Impacts}. Cost stickiness enables construction enterprises to rapidly mobilize resources and stimulate innovation when new project opportunities emerge \citep{love2004Industrycentric}, which is specifically required for sustainable competitive advantage.

\textbf{Enterprise size.} Size functions as a critical indicator of resource endowment \citep{shao2025Competitive}. Large-scale enterprises typically possess abundant slack resources and lower financing costs, which provide a buffer against the high risks inherent in construction projects. Furthermore, consistent with \citet{maury2018Sustainable}, who found that market share significantly predicts profit persistence, large construction enterprises benefit from deep social embeddedness \citep{lello2024Professional}. 

\textbf{Environmental, social, and governance (ESG).} ESG performance recently represents a critical intangible resource that transforms ethical behavior and sustainable practices into strategic assets \citep{wang2023Exploring}. Given that construction projects possess significant environmental footprints and social implications, superior ESG performance goes beyond mere compliance to become a mechanism for building trust and legitimacy. For construction enterprises, proactive ESG practices --- ranging from utilizing eco-friendly materials and ensuring site safety to maintaining transparent management --- can reduce friction with stakeholders and enhance corporate reputation \citep{locatelli2025Social}. As noted by \citet{mattera2022Sustainable}, a strong commitment to sustainable business models, as reflected in high ESG ratings, contributes to a firm's ability to improve long-term financial performance.

\textbf{Digital transformation.} Digital transformation represents a competency that fundamentally reconfigures an enterprise's operational resources \citep{wen2025Gap}. It involves integrating digital technologies (e.g., BIM, IoT) into project lifecycles to enhance decision-making and efficiency \citep{simard2025Project}. \citet{probojakti2025Driving} found that digital transformation significantly improves organizational agility and resiliency, which are pivotal for sustaining competitive edges. Also, \citet{van2025Green} emphasized that data-driven decision-making enabled by digital integration boosts organizational creativity and competitive advantage. Digital transformation allows construction enterprises to better sense environmental changes and seize new market opportunities, thereby securing a sustainable position.

\subsubsection{External environment and sustainable competitive advantage}
Construction enterprises operate as complex open systems where the sustainability of advantage is determined by how well internal capabilities align with external demands \citep{zhao2024Using}. Recent CEM studies suggested that the external environment is no longer static but characterized by rapid technological disruptions and fluctuating resource availability \citep{ning2022How}. This study identifies \textit{environmental dynamism} and \textit{environmental munificence} as the two critical environmental antecedents.

\textbf{Environmental dynamism.} Environmental dynamism refers to the rate and unpredictability of change in a firm's external environment \citep{dess1984Dimensions}. In the construction industry, dynamism is currently driven by the "Fourth Industrial Revolution" and increasingly stringent sustainability mandates \citep{aghimien2023Dynamic}. High dynamism challenges the traditional static model of competitive advantage, as existing competencies can rapidly become obsolete. \citet{zhao2024Using} argued that in transient competitive environments, advantages are easily eroded, compelling firms to continuously sense and seize new opportunities. Therefore, dynamism acts as a stressor that shifts from efficiency-based strategies to flexibility-based dynamic capabilities \citep{ning2022How}.

\textbf{Environmental munificence.} Environmental munificence describes the extent to which an environment can support sustained growth, reflecting the abundance of critical resources and market opportunities \citep{chen2017Munificence}. For construction enterprises, this manifests as the availability of infrastructure projects, financial capital, and network support \citep{ma2018Unraveling}. A munificent environment provides necessary "slack resources", allowing firms to experiment with innovations and absorb failures without threatening survival. \citet{wang2024Strategic} suggested that firms in munificent environments can leverage diversified operations to capture emerging opportunities. Thus, environmental munificence dictates the "room for maneuver".

\subsubsection{Strategic orientation and sustainable competitive advantage}
While \citet{porter1997COMPETITIVE}'s generic strategies have long served as a baseline, recent CEM literature suggested that sustainable competitive advantage emerges not from a single strategic posture but from the dynamic configuration of multiple orientations that match the firm's resource endowment with environmental demands \citep{shao2025Competitive}. This study identifies \textit{diversification}, \textit{differentiation}, and \textit{cost leadership} as three strategy-related antecedents.

\textbf{Diversification.} Diversification refers to the strategic expansion into new market segments or business lines to spread risks and capture emerging opportunities. For construction enterprises facing cyclical demand and intense local competition, diversification is a vital mechanism for survival and growth. \citet{wang2024Strategic} argued that "strategic resource amalgamation" is the driver of diversified operations, enabling contractors to leverage their operational and innovation capabilities across broader markets. Thus, diversification represents a strategy of scope, allowing firms to exploit their existing resource base.

\textbf{Differentiation.} Differentiation involves creating a unique value proposition, thereby allowing for premium pricing or customer loyalty. In the construction industry, differentiation is increasingly driven by "soft power" attributes such as corporate image, technical innovation, and brand reputation \citep{anjomshoa2024Key}. \citet{budayan2014Alignment} classified this into "quality and image-related differentiation," emphasizing that firms must align their project management processes with these strategic goals. Furthermore, differentiation is often achieved through the superior implementation of technologies like BIM. \citet{shao2025Competitive} found that image-oriented and quality-oriented competitive strategies signal competence and secure legitimacy. Differentiation serves as a strategy of value, insulating firms from direct price competition.

\textbf{Cost leadership.} Cost leadership is characterized by the pursuit of the lowest operational costs to offer competitive delivery services. While often viewed as a traditional strategic orientation, modern cost leadership transcends mere cost-cutting; it involves the rigorous pursuit of efficiency through lean management and technological integration \citep{li2024Lean}. \citet{sabug2020Competitive} highlighted that in competitive markets, a hybrid approach combining cost leadership with other strategies is often required for success. Thus, cost leadership represents a strategy of efficiency, essential for maintaining the economic viability in the low-margin construction industry \citep{li2025Impact}.

\subsection{Organizational resilience and its relationship with sustainable competitive advantage}

Resilience represents the fundamental capacity of construction enterprises to survive amidst contingencies \citep{zhang2022Organizational}. In the CEM literature, organizational resilience is defined as the dynamic capability of an enterprise to anticipate, absorb, recover from, and adapt to unexpected disruptions and shocks \citep{zhang2024Deconstructing}. The relevance of organizational resilience in the construction industry stems from the inherent complexity and uncertainty of project delivery \citep{yao2025Clear}. Construction enterprises frequently face high-impact, low-probability events—ranging from supply chain ruptures to sudden policy shifts—that threaten their viability \citep{wang2026Dynamic}. Unlike traditional risk management, which focuses on identifying specific risks, resilience emphasizes a generalized capacity to cope with the "unknown unknowns" \citep{lv2024Digital}.

Sustainable competitive advantage and organizational resilience represent two distinct yet complementary dimensions of construction enterprises' long-term viability. Sustainable competitive advantage focuses on \textit{market superiority}: it emphasizes the capacity to outperform competitors and secure persistent economic returns, largely under normal market conditions \citep{adner2006Demandbased}. In contrast, organizational resilience focuses on \textit{operational robustness}: it emphasizes the capacity to withstand shocks, absorb disruptions, and recover functions during crises or high-uncertainty events \citep{zhang2022Organizational,yao2025Clear}. While sustainable competitive advantage answers the question of "how to thrive and lead," organizational resilience answers "how to survive and persist". 

Studies on competitive advantage rarely account for how high-performance configurations work under crisis. It remains unclear whether the pathways leading to sustainable competitive advantage naturally encompass the attributes required for resilience, or if they leave enterprises vulnerable to disruptions.

In summary, the research framework of this study based on the "Resource-Environment-Strategy" framework is suggested in Fig. 1.

\begin{center}[Insert Fig. 1 here]\end{center}
\section{Model Building}

\subsection{The Baseline Two-Player Evolutionary Game Model}


We first establish a baseline evolutionary game model to analyze the strategic interactions between two horizontal governments during a disaster emergency. The model focuses on the decision-making process regarding cooperation on resource sharing, which includes both relief supplies and critical information. It assumes that the governments are boundedly rational and dynamically adjust their strategies based on the payoffs from previous interactions. This baseline game model does not include a higher-level (vertical) government. The baseline model is built upon the following key assumptions:

\textbf{Assumption 1}. The game involves two players, i.e., two governments at the same administrative level. The first player is the Local Government (LG), which represents the government whose jurisdiction is primarily affected by the disaster and is in need of assistance. The second player is the Neighboring Government (NG), which represents the government of an adjacent region that possesses surplus resources and can offer aid.

\textbf{Assumption 2}. Each player has a strategy set of \{Cooperate (C), Not Cooperate (NC)\}. Let $x$ be the probability that the LG chooses C, and $(1-x)$ be the probability it chooses NC, where $x \in [0, 1]$. Similarly, let $y$ be the probability that the NG chooses C, and $(1-y)$ be the probability it chooses NC, where $y \in [0, 1]$.

\textbf{Assumption 3}. The players are not perfectly rational; instead, they learn and adapt their strategies over time based on the relative success of past choices.

\textbf{Assumption 4}. The rescue benefit derived from relief supplies follows an ``S''-shaped function, which realistically captures the marginal utility of resources, from scarcity to abundance. The function is defined as:
\begin{equation}
    F(\theta) = \frac{c}{1+e^{-a\theta+b}}
\end{equation}
where $\theta = X/D$ represents the material satisfaction rate (the ratio of allocated supplies $X$ to demand $D$), and $a$, $b$, $c$ are benefit coefficients.

\textbf{Assumption 5} (Local Government's Strategic Considerations). When choosing to cooperate with the Neighboring Government, the LG can obtain additional relief supplies through regional coordination. When both LG and NG actively cooperate, both governments incur a cooperation cost $H$, and the LG gains public credibility $G_L$ for its collaborative efforts. According to the Interim Measures for the Management of Central Emergency and Disaster Relief Material Reserves (ref), following the principle of "user pays," the LG bears the transportation cost for the shared supplies. In this simplified model, we assume the transportation cost is proportional to the quantity of supplies transferred, expressed as $T = k(X_L - Q_L)$, where $k$ represents the per-unit transportation cost. Through supply sharing, the LG's per-capita rescue benefit $F_L$ exceeds what would be achieved without cooperation. Considering benefit distribution, the LG compensates the NG at a per-unit market price $m$, resulting in a coordination payment of $m(X_L - Q_L)$. Cooperation also involves information sharing, where the NG shares disaster situation data and resource information at a certain sharing rate, helping the LG improve emergency prediction and pre-deployment, thereby reducing potential costs and generating benefit $P_L$. When only the LG is willing to cooperate, it still incurs a unilateral cooperation cost $H_L$. When only the NG cooperates, the NG proactively shares information at rate $\alpha_N$, allowing the LG to obtain corresponding benefits.

\textbf{Assumption 6} (Neighboring Government's Strategic Considerations). The Neighboring Government's strategy space similarly consists of \{Cooperate, Not Cooperate\}. This analysis focuses on scenarios where the NG's disaster demand $D_N$ does not exceed its emergency reserve $Q_N$, meaning it has surplus supplies available to assist the LG. Given this surplus capacity, the NG must evaluate multiple factors including cooperation benefits, costs, and potential risks when making its decision. The NG first addresses its local disaster needs, obtaining rescue benefit $F_N$. Through cooperation, the NG receives coordination compensation $m(X_L - Q_L)$, information sharing benefit $\alpha P_N$, and public credibility $G_N$. However, it must also bear cooperation costs and consider potential losses from providing aid to the LG, which is primarily related to the quantity of coordinated supplies $(X_L - Q_L)$ and the per-unit potential loss $W$. When only the NG is willing to cooperate, it incurs a unilateral cooperation cost $H_N$. When only the LG cooperates, the LG shares information at rate $\alpha_L$.

\textbf{Parameters and Variables}

The parameters used in the baseline model are defined as follows:

\begin{table}[h]
\centering
\begin{tabular}{ll}
\hline
\textbf{Symbol} & \textbf{Definition} \\
\hline
\multicolumn{2}{l}{\textit{Government-Specific}} \\
$D_L$, $D_N$ & Demand for relief supplies for LG and NG, respectively \\
$Q_L$, $Q_N$ & Quantity of relief supplies initially possessed by LG and NG, respectively \\
$X_L$ & Total quantity of supplies available to LG after receiving aid from NG \\
      & The amount of aid is $(X_L - Q_L)$ \\
$G_L$, $G_N$ & The gain in public credibility for LG and NG from cooperative actions \\
\multicolumn{2}{l}{\textit{Costs}} \\
$H$ & Cost incurred by each government when both choose C \\
$H_L$, $H_N$ & Cost incurred by the willing party in a unilateral cooperation scenario \\
$T$ & Total transportation cost for the relief supplies, borne by the LG \\
$k$ & Per-unit transportation cost \\
$W$ & Per-unit potential loss for the NG for sharing its supplies \\
    & (e.g., risk of facing its own subsequent shortages) \\
\multicolumn{2}{l}{\textit{Benefits \& Payoffs}} \\
$F_L(\cdot)$, $F_N(\cdot)$ & The S-shaped benefit function for rescue effectiveness for LG and NG \\
$m$ & The per-unit compensation benefit paid by LG to NG for the provided supplies \\
$P_L$, $P_N$ & The benefit generated from information sharing for LG and NG, respectively \\
$\alpha$ & The information sharing rate when both governments choose C \\
$\alpha_L$, $\alpha_N$ & The information sharing rate when only LG or NG is willing to cooperate, respectively \\
\hline
\end{tabular}
\end{table}

\textbf{Payoff Matrix}

Based on the parameters above, the payoff matrix for the two-player game is constructed as follows:

\begin{table}[h]
\centering
\begin{tabular}{lcc}
\hline
 & \multicolumn{2}{c}{\textbf{Neighboring Government (NG)}} \\
\textbf{Local Government (LG)} & \textbf{C ($y$)} & \textbf{NC ($1-y$)} \\
\hline
\textbf{C ($x$)} & 
\begin{tabular}{@{}c@{}}
$D_L F_L\left(\frac{X_L}{D_L}\right) + \alpha P_L + G_L$ \\
$- (X_L - Q_L)(k+m) - H$, \\
$D_N F_N(1) + \alpha P_N + G_N$ \\
$+ (m-W)(X_L - Q_L) - H$
\end{tabular} & 
\begin{tabular}{@{}c@{}}
$D_L F_L\left(\frac{Q_L}{D_L}\right) + G_L - H_L$, \\
$D_N F_N(1) + \alpha_L P_N$
\end{tabular} \\
\textbf{NC ($1-x$)} & 
\begin{tabular}{@{}c@{}}
$D_L F_L\left(\frac{Q_L}{D_L}\right) + \alpha_N P_L$, \\
$D_N F_N(1) + G_N - H_N$
\end{tabular} & 
\begin{tabular}{@{}c@{}}
$D_L F_L\left(\frac{Q_L}{D_L}\right)$, \\
$D_N F_N(1)$
\end{tabular} \\
\hline
\end{tabular}
\end{table}

\noindent\textit{Note}: In each cell, the first entry is the payoff for the Local Government (LG), and the second is the payoff for the Neighboring Government (NG).

\textbf{Replicator Dynamics Equations}

The evolution of the strategies within the LG and NG populations is modeled by the following replicator dynamics equations:

\textbf{Replicator Dynamics Equation for the Local Government (LG):}
\begin{equation}
    F_L(x,y) = \frac{dx}{dt} = x(1-x)(E_x - E_{1-x})
\end{equation}
\begin{multline}
    = x(1-x)\Bigl(G_L - H_L + y\bigl(D_L F_L\left(\frac{X_L}{D_L}\right) - D_L F_L\left(\frac{Q_L}{D_L}\right) \\
    + (\alpha-\alpha_N)P_L - (X_L-Q_L)(k+m) - H + H_L\bigr)\Bigr)
\end{multline}

\textbf{Replicator Dynamics Equation for the Neighboring Government (NG):}
\begin{equation}
    F_N(x,y) = \frac{dy}{dt} = y(1-y)(E_y - E_{1-y})
\end{equation}
\begin{equation}
    = y(1-y)\left(G_N - H_N + x\left((\alpha-\alpha_L)P_N + (m-W)(X_L-Q_L) - H + H_N\right)\right)
\end{equation}

These equations describe the rate of change of the proportion of players adopting the C strategy in each population, forming the basis for analyzing the system's evolutionary stable strategies (ESS).
\section{Computational Experiments}
\label{sec:experiments}

\subsection{Experimental Setup}

We implemented our algorithm in MATLAB and conducted experiments with the parameters shown in Table \ref{tab:parameters}.

% 专业表格示例
\begin{table}[!htbp]
\centering
\caption{Experimental Parameters and Their Values}
\label{tab:parameters}
\begin{tabular}{lcc}
\toprule
Parameter & Symbol & Value \\
\midrule
Number of agents & $n$ & 5 \\
Cost coefficient & $\alpha$ & 0.1 \\
Discount factor & $\beta$ & 0.95 \\
Convergence tolerance & $\epsilon$ & $10^{-6}$ \\
Maximum iterations & $T$ & 1000 \\
\bottomrule
\end{tabular}
\end{table}

\subsection{Performance Metrics}

We evaluate our approach using the following metrics:
\begin{itemize}
    \item System-wide efficiency improvement
    \item Convergence speed (iterations to equilibrium)
    \item Solution stability under parameter variations
\end{itemize}
\section{Results and Discussion}
\label{sec:results}

Our computational experiments demonstrate the effectiveness of the proposed approach. The algorithm consistently converges to Nash equilibrium within 50 iterations across all test scenarios.

The utility function defined in Equation \ref{eq:utility} provides a robust framework for modeling agent interactions, while the equilibrium conditions in Equations \ref{eq:nash_condition} and \ref{eq:non_negativity} ensure solution stability.
\section{Conclusion and Future Work}
\label{sec:conclusion}

This study successfully demonstrates the application of game-theoretic approaches to multi-agent engineering management problems. Our key contributions include:

\begin{enumerate}
    \item A novel mathematical framework combining game theory with optimization
    \item Computational algorithms that efficiently solve large-scale problems
    \item Empirical validation showing significant performance improvements
\end{enumerate}

Future research directions include extending the model to dynamic environments and incorporating uncertainty in agent behaviors.

% ===== 参考文献 =====
% 使用BibTeX生成参考文献列表(Harvard格式)
\bibliographystyle{elsarticle-harv}
\bibliography{dj01}

\end{document}

% ===== Harvard 引用格式说明 =====
% 1. \citep{key} → (Author, Year) - 括号内引用
% 2. \citet{key} → Author (Year) - 文本内引用
% 3. 参考文献按作者姓氏字母排序
% 4. 适合管理学、社会科学期刊投稿