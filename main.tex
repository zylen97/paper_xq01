% !TEX program = xelatex
% 这行告诉LaTeX编辑器使用XeLaTeX引擎编译(支持更好的字体处理)

% Elsarticle文档类声明:review格式,10pt字体(5号字),author-year引用风格
\documentclass[authoryear,preprint,review,times,10pt]{elsarticle}

% ===== 宏包导入部分 =====
% 数学相关宏包(必备!工程管理论文经常用到数学公式)
\usepackage{amsmath}      % 强大的数学环境,如align、equation等
\usepackage{amssymb}      % 数学符号,如∈、⊆、∀、∃等
\usepackage{amsfonts}     % 数学字体,如粗体向量、花体字母等
\usepackage{bm}           % 加粗数学符号

% 基础功能宏包(elsarticle已包含graphicx)
\usepackage{booktabs}     % 制作专业表格(\toprule、\midrule、\bottomrule)
\usepackage{threeparttable} % 支持表格注释
\usepackage{caption}      % caption格式控制
\usepackage{adjustbox}    % 自动调整表格大小
\usepackage{enumitem}     % 增强列表功能(itemize、enumerate)
\usepackage{lineno}       % 行号宏包(review格式需要)

% 页边距调整(草稿阶段使用,投稿时可注释掉)
\usepackage{geometry}
\geometry{
    left=2cm,    % 左边距(原约3.5cm)
    right=2cm,   % 右边距(原约3.5cm)
    top=1.2cm,     % 上边距(原约3cm)
    bottom=1.5cm,  % 下边距(调整后,页码不会太靠下)
    footskip=0.5cm % 文本底部到页脚的距离
}

\usepackage{hyperref}     % 超链接和PDF书签(应该放在最后导入)

% 指定目标期刊(根据实际投稿期刊修改)
\journal{Engineering Management Journal}

\begin{document}

% ASCE要求:双倍行距
\renewcommand{\baselinestretch}{2.0}
\normalsize  % 激活新的行距设置

\linenumbers  % ASCE要求:摘要也需要行号
\begin{frontmatter}

% 论文标题
\title{Unraveling Sustainable Competitive Advantage and Resilience of Construction Enterprises: "Resource-Environment-Strategy" Framework}

% 作者信息(使用elsarticle格式)
\author[label1]{Zilun Wang}
\ead{zylenw97@usts.edu.cn}

\author[label1]{Ying Tang}
\ead{19552089006@163.com}

\author[label2]{Dongjie Zheng\texorpdfstring{\corref{cor1}}{*}}
\ead{2531180@tongji.edu.cn}

\author[label2]{Qinghua He}
\ead{heqinghua@tongji.edu.cn}

\cortext[cor1]{Corresponding author}

\affiliation[label1]{organization={School of Civil Engineering, Suzhou University of Science and Technology},
            city={Suzhou},
            postcode={215011},
            country={China}}

\affiliation[label2]{organization={School of Economics and Management, Tongji University},
            city={Shanghai},
            postcode={200092},
            country={China}}

% 摘要
\begin{abstract}
Developing sustainable competitive advantage is imperative for construction enterprises to navigate the high-velocity environment driven by digitalization and sustainability mandates. However, existing construction engineering and management literature has yet to systematically explore sustainable competitive advantage, largely focusing on static advantages or the net effects of isolated variables. To bridge this gap, this study investigates the configurations shaping sustainable competitive advantage by employing a "Resource-Environment-Strategy" framework and analyzing 1,061 observations from 115 listed Chinese construction enterprises (2014--2023) through necessary condition analysis, time-series qualitative comparative analysis, and typical case tracing. Additionally, we examine which high sustainable competitive advantage configurations enable organizational resilience. The results indicate that no single condition is necessary for sustainable competitive advantage; instead, eight equifinal configurations emerge, revealing a dual market structure: majority of configurations relying on efficiency-driven pathways (scale and diversification), and a minority leveraging differentiation and specialization. Also, a distinct efficiency-resilience trade-off is identified, where mainstream configurations exhibit fragility during crises, while the minority of configurations demonstrate superior resilience through mechanisms of legitimacy and resource slack. This study contributes theoretically by clarifying the configurational nature of sustainable competitive advantage and offering practical guidance for orchestrating resources to ensure long-term viability.
\end{abstract}

% 关键词(使用elsarticle格式)
\begin{keyword}
Sustainable competitive advantage \sep Organizational resilience \sep Construction enterprises \sep Time-series qualitative comparative analysis \sep Typical case tracing
\end{keyword}

\end{frontmatter}
\clearpage  % 强制换页,避免跨页导致双横线
% 行号已在frontmatter之前启用

% 目录页(期刊论文通常不需要)
% \tableofcontents
% \newpage

% ===== 正文各章节 =====
% 使用 \input 命令导入各章节文件

\section{Introduction}
\label{sec:introduction}

The construction industry is undergoing a transformation, challenging the efficiency of construction enterprises' traditional, often fragmented, project-based business modes \citep{dang2025Assessing}. On the one hand, conventional operational pressures exist, including rising material costs and skilled labor shortages \citep{aghimien2022Dynamic}. On the other hand, disruptive external requirements, such as digitalization and sustainability, are intensifying these pressures \citep{wen2025Gap}. Digitalization from building information modeling (BIM) or robotics can mitigate labor shortages; however, it requires substantial capital investment and implementation across projects, which in turn reduces enterprises' short-term profit margins \citep{shao2025Competitive}. Sustainability mandates, such as net-zero regulations \citep{locatelli2025Social}, also introduce new compliance costs and operational complexities \citep{wang2023Exploring}. These trends create a high-velocity environment where conventional enterprises' advantages, such as a successful project bid or isolated technology adoption, are no longer sufficient to ensure long-term performance. A practical problem facing construction enterprises' managers is how to develop \textbf{\textit{sustainable competitive advantage}} --- a set of difficult-to-imitate edges \citep{adner2006Demandbased} --- to ensure growth within contexts of project-based business.

Sustainable competitive advantage, coined by \citet[p.~102]{barney1991Firm}, denotes a firm's capacity to "\textit{implement a value creating strategy not simultaneously being implemented by any current or potential competitors and when these other firms are unable to duplicate the benefits of this strategy}". Developing sustainable competitive advantage is critical for construction enterprises due to their inherent discontinuity and fragmentation of operations \citep{betts1994Sustainablea}. Unlike continuous manufacturing, construction operations are project-based and temporary, often leading to the loss of knowledge and efficiency when project teams disband \citep{sydow2018Projects}. In this context, sustainable competitive advantage improves organizational continuity, allowing enterprises to transfer technological innovations and management expertise across projects \citep{malik2023Green}. Specifically, the construction industry is historically dominated by fierce price competition and thin profit margins \citep{sharma2023Construction}. Rare and non-substitutable edges enable enterprises to differentiate themselves beyond low bid competition \citep{deng2014Developing}. Thus, clarifying the formation of construction enterprises' sustainable competitive advantage is imperative to the construction engineering and management (CEM) literature for understanding how to ensure long-term survival.

Existing CEM literature has largely investigated the antecedents of construction enterprises' competitive advantage in three aspects. First, research identifies critical resource-based factors, such as resource amalgamation \citep{wang2024Strategic}, dynamic capability \citep{ning2022How}, and human capital \citep{sarihi2020Multiskilled}, as the foundation of market position. Second, studies on strategy-based antecedents explore specific business modes, including internationalization \citep{jang2020Classifying}, networking \citep{lello2024Professional}, and projectization \citep{barbosa2024Multilevel}. Third, works on innovation-based antecedents highlight how digital technologies like BIM \citep{shao2025Competitive} and green innovation techniques \citep{dang2025Assessing} offer new avenues for sustainable development.

Despite these advancements, \textbf{\textit{two gaps remain in the CEM literature}} that hinder a comprehensive understanding of the formation of sustainable competitive advantage. \textbf{\textit{First}}, CEM literature pays most attention to static or conventional competitive advantage rather than sustainable competitive advantage. Related studies treated competitive advantage as a direct, immediate outcome of achieving profitability \citep{li2025Impact}, possessing specific business resources \citep{wang2024Strategic}, or adopting new tools. In contrast, sustainable competitive advantage emphasizes the formation of difficult-to-imitate advantages through continuous resource integration and capability enhancement \citep{abdeen2025Strategic}. Although some CEM studies have discussed sustainable competitive advantage, they failed to capture its core attributes. For instance, \citet{toor2010Positive} conceptualized sustainable competitive advantage as a derivative of leadership. \citet[p.~45]{betts1994Sustainablea} equated it with the critical success factors of construction enterprises. These studies overlooked the very nature of sustainable competitive advantage, rendering it insufficient to guide construction enterprises' managers. \textbf{\textit{Second}}, competitive advantage research is fragmented, mostly examining antecedents from isolated perspectives. A substantial body of literature relies on a "net-effect" logic to assess the specific impact of single variables, such as the implementation of BIM, project management tools and techniques \citep{li2025Impact}, or specific market segment drivers. In reality, construction enterprises do not strictly respond to single environmental stimuli; rather, they "orchestrate" antecedents to navigate challenges \citep{black1994Strategic}. While some recent studies have adopted a configurational perspective \citep{wang2024Strategic,shao2025Competitive}, they have not explicitly targeted competitive advantage as their subjects. CEM literature lacks an integrative, configurational perspective to systematically unravel how different antecedents interact and combine to jointly develop sustainable competitive advantage. Therefore, \textbf{\textit{the first aim of this study is to investigate the configurations of factors that shape construction enterprises' sustainable competitive advantage.}} 

While sustainable competitive advantage is crucial for sustained growth, organizational resilience serves as an important safeguard for survival and stability in turbulent environments \citep{yang2024What}. Organizational resilience refers to an enterprise's capacity to anticipate, absorb, and respond to contingencies \citep{zhang2024Deconstructing}. Despite systematic examinations of organizational resilience in CEM literature \citep{zhang2022Organizational}, few studies have integrated resilience with sustainable competitive advantage. Sustainable competitive advantage and organizational resilience represent two distinct yet mutually reinforcing dimensions of long-term viability. Sustainable competitive advantage primarily targets the sustainability of superior market position and inimitable performance, while organizational resilience prioritizes stable operations and continuity, specifically in crisis events \citep{shao2024Contradiction}. Focusing solely on competitive superiority may create vulnerability to crises, particularly in the highly uncertain post-pandemic era \citep{lv2024Digital}. This disconnection hinders our understanding of how construction enterprises can ensure long-term market dominance with the capability of addressing contingencies. Consequently, \textbf{\textit{the second aim of this study is to examine which configurations shaping sustainable competitive advantage simultaneously enable a high level of organizational resilience.}}

To bridge these knowledge gaps, we developed a "Resource-Environment-Strategy" framework to investigate configurations of nine antecedents across dimensions of organizational resource, external environment, and strategic orientation. This framework is derived from the literature on sustainable competitive advantage and is further grounded in the body of CEM research. This study utilizes a panel dataset including 1,061 observations from 115 listed Chinese construction enterprises spanning 2014-2023. We employed necessary condition analysis, time-series qualitative comparative analysis, and typical case tracing to examine how configurations of nine antecedents relate to sustainable competitive advantage and organizational resilience. Therefore, \textbf{\textit{this study answers two research questions (RQs):}}

\textit{RQ1: What are the configurations of factors across dimensions of organizational resource, external environment, and strategic orientation that shape construction enterprises' sustainable competitive advantage?}

\textit{RQ2: Which of these configurations simultaneously allow construction enterprises to maintain a high level of organizational resilience, specifically in crisis events?}

The remainder of this paper is organized as follows. Section 2 reviews the related works and establishes the "Resource-Environment-Strategy" framework. Section 3 details the research methods. Section 4 presents the empirical results. Section 5 discusses the key findings, elaborating on research implications. Section 6 concludes the study with limitations for future research.
\section{Literature review and theoretical foundation}
\label{sec:literature}

Three streams of literature are related to our research questions: (1) works on sustainable competitive advantage, (2) CEM literature regarding competitive advantage, and (3) CEM literature on organizational resilience. This section reviews these studies and builds the theoretical foundation. First, we synthesized the evolution of sustainable competitive advantage research to propose the "Resource-Environment-Strategy" framework. Second, we reviewed extant CEM literature regarding competitive advantage to contextualize the framework and identified nine potential antecedents. Third, we reviewed organizational resilience studies to clarify its relationship with sustainable competitive advantage.

\subsection{"Resource-Environment-Strategy" framework from sustainable competitive advantage studies}

We derived the "Resource-Environment-Strategy" framework from a review of the literature on sustainable competitive advantage. Distinct from conventional competitive advantage, sustainable competitive advantage emphasizes the \textit{durability} of superior performance and the \textit{inimitability} of value-creating capabilities against competitive duplication \citep{oliver1997Sustainable,lado1992CompetencyBased}. Scholars have progressively investigated the antecedents from resource, environment, and strategy perspectives.

Sustainable competitive advantage originates from the resource-based view, which posits that the heterogeneous resources are the primary source of advantage \citep{abdeen2025Strategic}. However, the literature evolved from focusing on tangible assets to more complex resource forms. For instance, \citet{hall1993FRAMEWORK} argued that sustainable advantage is derived primarily from intangible resources—such as reputation and employee know-how—because their causal ambiguity creates formidable barriers to imitation. Further, \citet{black1994Strategic} critiqued the atomistic view of resources, introducing the concept of "resource networks". They posited that advantage arises not from isolated factors but from the complementary relationships between resources.

Building upon the resource-based view, several studies integrating institutional theory and market perspectives argue that resources cannot exist in a vacuum; rather, their value is contingent upon the external environment. \citet{oliver1997Sustainable} provided a seminal integration, demonstrating that an enterprise's advantage is shaped by the interaction between internal resources and external institutional pressures, where social legitimacy becomes a prerequisite for survival. Also, \citet{adner2006Demandbased} indicated that the sustainability of advantage is determined by market heterogeneity and consumer marginal utility rather than supply-side capabilities alone. Empirical work by \citet{mady2024Nexus} further confirmed that external drivers, such as regulatory pressure and eco-friendly product demand, are critical forces that compel enterprises to adapt their resource base.

The scholarship also highlights that alignment between resources and the environment requires clear strategic orientation. Strategic orientation reflects the firm's proactive intent and the specific logic it employs to create and capture value \citep{sabug2020Competitive}. \citet{lado1992CompetencyBased} were among the first to propose a comprehensive model that prioritizes "managerial competencies" and "strategic focus". They argued that strategy acts as the "engine" that transforms input resources into competitive outputs, integrating environmental determinism with strategic choice. This view is supported by \citet{johannessen2003Knowledge}, who highlighted that sustainable competitive advantage is the result of conscious "strategic training" and management intervention. Moreover, \citet{malik2023Green} empirically demonstrated that strategic orientations act as essential mediators that leverage technological readiness to achieve sustainable competitive advantage, particularly in emerging markets.

While the literature has identified these three dimensions, prior studies have predominantly examined them in isolation. \citet{rouse1999Rethinking} and \citet{levitas2002Rethinking} debated the difficulties of isolating sources of advantage, pointing to a "black box" in understanding how these factors interact holistically. On this basis, we propose the "Resource-Environment-Strategy" framework to investigate the configurations of construction enterprises' sustainable competitive advantage.

\subsection{Antecedents of sustainable competitive advantage in construction enterprises}
Construction enterprises are characterized by project discontinuity, fragmented supply chains, and high sensitivity to institutional pressures \citep{ning2022How}. These attributes render the "Resource-Environment-Strategy" framework pertinent for unraveling the causal complexity of sustainable competitive advantage. This section grounds the framework in the CEM literature regarding conventional competitive advantage and identifies nine antecedents accordingly.

\subsubsection{Organizational resource and sustainable competitive advantage}
CEM literature regarding conventional competitive advantage has witnessed a theoretical evolution from the resource-based view to the dynamic capabilities view \citep{choi2018Dynamic}. Resource-based view primarily centers on the accumulation of internal resources that are valuable, rare, inimitable, and non-substitutable \citep{barney2021Emergence}; rather, dynamic capabilities are defined as the potential to "\textit{integrate, build, and reconfigure internal and external competences to address rapidly changing environments}" \citep{whang2024Understanding}. Situated within these two theoretical lenses, this study identifies \textit{cost stickiness}, \textit{organizational size}, \textit{ESG}, and \textit{digital transformation} as four resource-related antecedents.

\textbf{Cost stickiness.} Cost stickiness refers to the asymmetry where costs increase more rapidly with rising activity volume than they decrease during declines. Cost stickiness in construction enterprises reflects the deliberate retention of slack resources, such as skilled project managers and specialized technical equipment \citep{potgieter2016Maximizing}, acting as a resource investment from a long-term perspective \citep{luo2019Impacts}. Cost stickiness enables construction enterprises to rapidly mobilize resources and stimulate innovation when new project opportunities emerge \citep{love2004Industrycentric}, which is specifically required for sustainable competitive advantage.

\textbf{Enterprise size.} Size functions as a critical indicator of resource endowment \citep{shao2025Competitive}. Large-scale enterprises typically possess abundant slack resources and lower financing costs, which provide a buffer against the high risks inherent in construction projects. Furthermore, consistent with \citet{maury2018Sustainable}, who found that market share significantly predicts profit persistence, large construction enterprises benefit from deep social embeddedness \citep{lello2024Professional}. 

\textbf{Environmental, social, and governance (ESG).} ESG performance recently represents a critical intangible resource that transforms ethical behavior and sustainable practices into strategic assets \citep{wang2023Exploring}. Given that construction projects possess significant environmental footprints and social implications, superior ESG performance goes beyond mere compliance to become a mechanism for building trust and legitimacy. For construction enterprises, proactive ESG practices --- ranging from utilizing eco-friendly materials and ensuring site safety to maintaining transparent management --- can reduce friction with stakeholders and enhance corporate reputation \citep{locatelli2025Social}. As noted by \citet{mattera2022Sustainable}, a strong commitment to sustainable business models, as reflected in high ESG ratings, contributes to a firm's ability to improve long-term financial performance.

\textbf{Digital transformation.} Digital transformation represents a competency that fundamentally reconfigures an enterprise's operational resources \citep{wen2025Gap}. It involves integrating digital technologies (e.g., BIM, IoT) into project lifecycles to enhance decision-making and efficiency \citep{simard2025Project}. \citet{probojakti2025Driving} found that digital transformation significantly improves organizational agility and resiliency, which are pivotal for sustaining competitive edges. Also, \citet{van2025Green} emphasized that data-driven decision-making enabled by digital integration boosts organizational creativity and competitive advantage. Digital transformation allows construction enterprises to better sense environmental changes and seize new market opportunities, thereby securing a sustainable position.

\subsubsection{External environment and sustainable competitive advantage}
Construction enterprises operate as complex open systems where the sustainability of advantage is determined by how well internal capabilities align with external demands \citep{zhao2024Using}. Recent CEM studies suggested that the external environment is no longer static but characterized by rapid technological disruptions and fluctuating resource availability \citep{ning2022How}. This study identifies \textit{environmental dynamism} and \textit{environmental munificence} as the two critical environmental antecedents.

\textbf{Environmental dynamism.} Environmental dynamism refers to the rate and unpredictability of change in a firm's external environment \citep{dess1984Dimensions}. In the construction industry, dynamism is currently driven by the "Fourth Industrial Revolution" and increasingly stringent sustainability mandates \citep{aghimien2023Dynamic}. High dynamism challenges the traditional static model of competitive advantage, as existing competencies can rapidly become obsolete. \citet{zhao2024Using} argued that in transient competitive environments, advantages are easily eroded, compelling firms to continuously sense and seize new opportunities. Therefore, dynamism acts as a stressor that shifts from efficiency-based strategies to flexibility-based dynamic capabilities \citep{ning2022How}.

\textbf{Environmental munificence.} Environmental munificence describes the extent to which an environment can support sustained growth, reflecting the abundance of critical resources and market opportunities \citep{chen2017Munificence}. For construction enterprises, this manifests as the availability of infrastructure projects, financial capital, and network support \citep{ma2018Unraveling}. A munificent environment provides necessary "slack resources", allowing firms to experiment with innovations and absorb failures without threatening survival. \citet{wang2024Strategic} suggested that firms in munificent environments can leverage diversified operations to capture emerging opportunities. Thus, environmental munificence dictates the "room for maneuver".

\subsubsection{Strategic orientation and sustainable competitive advantage}
While \citet{porter1997COMPETITIVE}'s generic strategies have long served as a baseline, recent CEM literature suggested that sustainable competitive advantage emerges not from a single strategic posture but from the dynamic configuration of multiple orientations that match the firm's resource endowment with environmental demands \citep{shao2025Competitive}. This study identifies \textit{diversification}, \textit{differentiation}, and \textit{cost leadership} as three strategy-related antecedents.

\textbf{Diversification.} Diversification refers to the strategic expansion into new market segments or business lines to spread risks and capture emerging opportunities. For construction enterprises facing cyclical demand and intense local competition, diversification is a vital mechanism for survival and growth. \citet{wang2024Strategic} argued that "strategic resource amalgamation" is the driver of diversified operations, enabling contractors to leverage their operational and innovation capabilities across broader markets. Thus, diversification represents a strategy of scope, allowing firms to exploit their existing resource base.

\textbf{Differentiation.} Differentiation involves creating a unique value proposition, thereby allowing for premium pricing or customer loyalty. In the construction industry, differentiation is increasingly driven by "soft power" attributes such as corporate image, technical innovation, and brand reputation \citep{anjomshoa2024Key}. \citet{budayan2014Alignment} classified this into "quality and image-related differentiation," emphasizing that firms must align their project management processes with these strategic goals. Furthermore, differentiation is often achieved through the superior implementation of technologies like BIM. \citet{shao2025Competitive} found that image-oriented and quality-oriented competitive strategies signal competence and secure legitimacy. Differentiation serves as a strategy of value, insulating firms from direct price competition.

\textbf{Cost leadership.} Cost leadership is characterized by the pursuit of the lowest operational costs to offer competitive delivery services. While often viewed as a traditional strategic orientation, modern cost leadership transcends mere cost-cutting; it involves the rigorous pursuit of efficiency through lean management and technological integration \citep{li2024Lean}. \citet{sabug2020Competitive} highlighted that in competitive markets, a hybrid approach combining cost leadership with other strategies is often required for success. Thus, cost leadership represents a strategy of efficiency, essential for maintaining the economic viability in the low-margin construction industry \citep{li2025Impact}.

\subsection{Organizational resilience and its relationship with sustainable competitive advantage}

Resilience represents the fundamental capacity of construction enterprises to survive amidst contingencies \citep{zhang2022Organizational}. In the CEM literature, organizational resilience is defined as the dynamic capability of an enterprise to anticipate, absorb, recover from, and adapt to unexpected disruptions and shocks \citep{zhang2024Deconstructing}. The relevance of organizational resilience in the construction industry stems from the inherent complexity and uncertainty of project delivery \citep{yao2025Clear}. Construction enterprises frequently face high-impact, low-probability events—ranging from supply chain ruptures to sudden policy shifts—that threaten their viability \citep{wang2026Dynamic}. Unlike traditional risk management, which focuses on identifying specific risks, resilience emphasizes a generalized capacity to cope with the "unknown unknowns" \citep{lv2024Digital}.

Sustainable competitive advantage and organizational resilience represent two distinct yet complementary dimensions of construction enterprises' long-term viability. Sustainable competitive advantage focuses on \textit{market superiority}: it emphasizes the capacity to outperform competitors and secure persistent economic returns, largely under normal market conditions \citep{adner2006Demandbased}. In contrast, organizational resilience focuses on \textit{operational robustness}: it emphasizes the capacity to withstand shocks, absorb disruptions, and recover functions during crises or high-uncertainty events \citep{zhang2022Organizational,yao2025Clear}. While sustainable competitive advantage answers the question of "how to thrive and lead," organizational resilience answers "how to survive and persist". 

Studies on competitive advantage rarely account for how high-performance configurations work under crisis. It remains unclear whether the pathways leading to sustainable competitive advantage naturally encompass the attributes required for resilience, or if they leave enterprises vulnerable to disruptions.

In summary, the research framework of this study based on the "Resource-Environment-Strategy" framework is suggested in Fig. 1.

\begin{center}[Insert Fig. 1 here]\end{center}
\section{Research Methods}
\label{sec:methods}

\subsection{The Baseline Two-Player Evolutionary Game Model}


\section{Results}
\label{sec:results}

\subsection{Necessary conditions analysis results}

Table \ref{tab:nca} presents the NCA results using both Ceiling Regression (CR) and Ceiling Envelopment (CE) techniques. The empirical data reveal that the effect sizes for all nine antecedent conditions—spanning strategic orientations, environmental characteristics, and organizational resources—fall substantially below the threshold of 0.1. Moreover, the permutation tests yield non-significant results for all conditions (p-values = 1.00), indicating that no single factor constitutes a necessary condition for achieving sustainable competitive advantage. Furthermore, the bottleneck level analysis in Table 3 corroborates these findings, demonstrating that achieving any specific percentile of the outcome (ranging from 10\% to 100\%) does not require any antecedent condition to reach a specific minimum threshold.

These findings provide a critical theoretical insight: they effectively refute "monocausal" explanations for success in the Chinese construction industry. The results demonstrate that achieving a high level of sustainable competitive advantage does not depend on any single attribute—neither a specific strategy (e.g., diversification) nor a particular resource (e.g., digitalization) individually serves as a prerequisite. This absence of universal necessary conditions validates the core premise of this study: success is not driven by isolated factors but by the synergistic orchestration of multiple ingredients. Consequently, these results provide a robust empirical foundation for the subsequent sufficiency analysis, justifying the focus on configurational pathways rather than net-effect relationships.
\section{Discussion}
\label{sec:discussion}

\subsection{Discussion on research questions}
\subsubsection{Configurations that shape construction enterprises' sustainable competitive advantage (RQ1)}
Regarding RQ1, our findings reveal that the formation of sustainable competitive advantage is shaped by complex interactions of antecedents across dimensions of organizational resource, external environment, and strategic orientation. \textbf{\textit{Firstly}}, the NCA results indicate that no single condition constitutes a necessary condition for sustainable competitive advantage. This finding expands on some CEM literature that often elevated specific factors, such as BIM adoption or human capital, to the status of prerequisites for success \citep{probojakti2025Driving}. Instead, our results resonate with the principle of "equifinality", demonstrating that construction enterprises can achieve the same outcome through multiple, distinct pathways. For instance, while digital transformation is highly emphasized \citep{simard2025Project}, our NCA results suggest it is not necessary; enterprises can still achieve sustainable advantages through specialized cost leadership without heavy digital reliance (as seen in C4a), provided they possess other resources like cost stickiness.

\textbf{\textit{Secondly}}, TSQCA results indicate eight distinct configurations. The high consistency of these configurations validates the principle of "equifinality". There is no single "best way"; rather, there are multiple "orchestrations" of resources and strategies that match specific environments, echoing the configurational view that competitive advantage stems from complex resource networks \citep{black1994Strategic}. Further, POCOV results delineate the distribution of valid configurations. Configurations in Group C3 ("Diversification-Driven Cost Leadership") and Group C2 ("Digital-Enabled Lean Scale") exhibit the highest explanatory power. From the "Resource-Environment-Strategy" perspective, these high-coverage groups share a common strategic core of cost leadership, yet they achieve it through distinct mechanisms. Group C3 represents an "externally-oriented" efficiency model where enterprises, operating in munificent environments, leverage diversification strategies to integrate supply chains and internalize transaction costs, consistent with the logic of strategic resource amalgamation \citep{wang2024Strategic}. Conversely, Group C2 represents an "internally-oriented" efficiency model, where firms rely on massive organizational size and digital transformation to achieve economies of scale and optimize management efficiency. The dominance of these two groups suggests that efficiency-based cost leadership remains the mainstream logic for Chinese construction enterprises to build sustainable competitive advantage. This finding aligns with the industry's inherent characteristics of fierce price competition and thin profit margins \citep{das2021Developing,sharma2023Construction}. Consequently, for the majority of enterprises, the primary pathway to sustainability lies in leveraging resource endowments to ensure efficiency.

In contrast, Group C1 ("Dual-Resource Driven Differentiation") and Group C4 ("Specialized Cost Leadership") demonstrate significantly lower POCOV values. This disparity indicates that these configurations represent "elite" or "niche" pathways that are fewer in number but equally effective in outcome. Group C1 reflects a "high-end" pathway where enterprises combine superior ESG performance and digital capabilities to pursue differentiation. The lower coverage of this group suggests high barriers to entry, as it requires the accumulation of scarce, high-quality resources that average firms typically lack, particularly given the challenges of transforming ethical behaviors into strategic assets \citep{locatelli2025Social}. Similarly, Group C4 reflects a "specialized" pathway where enterprises reject diversification in favor of focusing on a niche market. The presence of cost stickiness in this configuration implies a reliance on asset specificity and long-term resource commitment, which acts as a strategic investment in specialized human capital \citep{luo2019Impacts}. The low coverage here indicates that this is a focused strategy suitable for "hidden champions" in specific technical sub-sectors, rather than a general model for the mass market. 

\subsubsection{High sustainable competitive advantage configurations that simultaneously maintain high organizational resilience (RQ2)}

Regarding RQ2, we identified which sustainable competitive advantage configurations can lead to high organizational resilience. Our investigation of temporal variations of BECOV in configurations during crisis reveals the answer: the configurations that are most dominant in stable times (Groups C2 and C3) appear the most vulnerable during crises. Specifically, the coverage of C2 and C3 declined sharply during shock periods. This finding empirically corroborates the "efficiency-resilience trade-off" discussed in recent literature, suggesting that the pursuit of absolute efficiency may inadvertently erode the capacity to withstand shocks \citep{shao2024Contradiction}. We argue that the very mechanisms driving the efficiency of these groups create structural rigidities. For Group C3, the reliance on complex, integrated supply chains—while reducing transaction costs in normal times—amplifies exposure to disruption risks. When external shocks cause widespread supply chain ruptures, the tight coupling between diversified segments transmits the shock across the entire enterprise, leading to a rapid loss of advantage \citep{zhang2024Deconstructing}. Similarly, for Group C2, the pursuit of economies of scale creates organizational inertia. While massive assets provide resource buffers, they also entail high fixed costs. In the face of sudden market contractions, these "heavy" enterprises struggle to pivot quickly, highlighting that focusing solely on competitive superiority may create relative vulnerability to contingencies \citep{lv2024Digital}.

On the contrary, Group C1 and Group C4 demonstrated remarkable stability, with minimal fluctuations in coverage during crises. This suggests that the pathways to resilience differ fundamentally from those to efficiency. For Group C1, the stability suggests that superior ESG performance acts as an "insurance mechanism". During crises, intangible assets such as corporate reputation and stakeholder trust become critical buffers. Enterprises in this group leverage their commitment to social responsibility to maintain legitimacy and reduce friction with stakeholders, thereby preserving their market position even when external conditions deteriorate \citep{mattera2022Sustainable}. Furthermore, the resilience of Group C4 offers a novel theoretical insight into the role of cost stickiness. While conventional accounting wisdom often views cost stickiness as a sign of inefficiency, our findings suggest that for specialized firms, high stickiness represents the deliberate retention of slack resources—such as skilled project teams and specialized equipment \citep{potgieter2016Maximizing}. Instead of laying off staff to cut costs during downturns, these enterprises retain their core human capital. This "resource redundancy" allows them to rapidly mobilize capabilities and recover functions as soon as the shock subsides \citep{love2004Industrycentric}.

\subsection{Theoretical implications}
This study suggests four theoretical implications for the literature. First, this study contributes to the CEM literature by systematically conceptualizing and empirically investigating sustainable competitive advantage in construction enterprises. While prior studies have extensively examined static competitiveness or isolated factors, this research represents a pioneering effort to unravel the "sustainable" dimension of advantage—specifically the durability of value creation. By adopting a configurational perspective, we demonstrate that sustainable competitive advantage in the project-based construction industry is not driven by single factors but by the "orchestration" of resources, environments, and strategies. This shift provides a more holistic perspective for understanding how construction firms maintain long-term survival amidst discontinuity.

Second, this study bridges the theoretical divide between sustainable competitive advantage and organizational resilience in the CEM domain. Existing literature often treats these two constructs in isolation. By integrating them, our study uncovers a critical "efficiency-resilience trade-off" specific to the construction sector. We provide theoretical nuance by revealing that the efficiency-driven configurations that dominate stable periods are structurally fragile to exogenous shocks, whereas niche configurations exhibit superior resilience. This insight challenges the assumption that high performance automatically equates to high resilience, enriching the theoretical understanding of organizational viability in turbulent environments.

Third, this study validates and contextualizes the "Resource-Environment-Strategy" framework within the unique setting of the construction industry. While this framework was derived from general management literature, its application in CEM has been limited. We extend its applicability by demonstrating that construction enterprises are complex open systems where internal resource endowments must be dynamically aligned with external environmental conditions through strategic orientations. By confirming the explanatory power of this framework in the construction industry, we provide a robust theoretical scaffold for future research on strategic management in project-based organizations.

Fourth, this study makes a methodological contribution by integrating TSQCA with TCT. While QCA is gaining traction in CEM research, it is often criticized for remaining a "black box" regarding causal mechanisms \citep{frateur2025How}. By employing TCT, we moved beyond merely identifying "what" certain configurations work to explaining "how" they work through the detailed reconstruction of evidence chains in typical enterprises. Our mixed-method approach offers a rigorous template for future empirical studies addressing causal complexity in the construction management field.

\subsection{Practical implications}

We provide practical insights for construction enterprise managers aiming to achieve sustainable competitive advantage and maintain organizational resilience as follows.

First, managers must recognize that there is no "one-size-fits-all" formula for long-term success; rather, they should tailor their managerial patterns to align with their specific resource endowments and environmental contexts. For leaders of large-scale enterprises, the "Digital-Enabled Lean Scale" pathway (as seen in Group C2) offers a guideline. Managers in such enterprises should prioritize digital transformation not merely as a technical upgrade, but as a strategic enabler to optimize cost leadership. For example, implementing centralized digital procurement platforms can leverage massive purchasing power to negotiate lower material prices, converting sheer size into genuine efficiency. Conversely, for specialized or smaller firms, blindly imitating the diversification strategies of giants is ill-advised. Instead, they should adopt the "Specialized Cost Leadership" model (Group C4). Managers should focus resources on a specific niche—such as tunnel engineering or green building technology—to build technical barriers that generalist competitors cannot penetrate.

Second, managers are required to make a strategic choice between maximizing efficiency in stable times and ensuring resilience during crises, as our results highlight a distinct trade-off. For enterprises pursuing the "mainstream" efficiency strategies (Groups C2 and C3), managers must be acutely aware of their structural fragility. While integrated supply chains drive profits in munificent environments, they become vulnerabilities during shocks. Therefore, managers adopting these strategies should establish proactive risk-addressing mechanisms, such as maintaining flexible backup suppliers rather than relying solely on lean, just-in-time delivery, to mitigate the losses due to contingencies.

Third, for managers seeking high organizational resilience, our findings regarding "niche" configurations (Groups C1 and C4) offer advice regarding resource allocation. Specifically, managers should rethink the value of ESG and Cost Stickiness. Regarding ESG, managers should view investments in social responsibility not as compliance burdens but as intangible assets. Proactive community engagement and environmental compliance can build a reservoir of goodwill that protects the firm's legitimacy when market sentiments turn negative. Regarding cost stickiness, managers of specialized firms should resist the pressure to immediately cut costs by laying off core technical staff during industry downturns. By retaining skilled project managers and R&D teams during recessions, managers effectively preserve the organization's recovery capacity, enabling the firm to seize opportunities faster than competitors once the market rebounds.
\section{Conclusion and Future Work}
\label{sec:conclusion}

This study successfully demonstrates the application of game-theoretic approaches to multi-agent engineering management problems. Our key contributions include:

\begin{enumerate}
    \item A novel mathematical framework combining game theory with optimization
    \item Computational algorithms that efficiently solve large-scale problems
    \item Empirical validation showing significant performance improvements
\end{enumerate}

Future research directions include extending the model to dynamic environments and incorporating uncertainty in agent behaviors.

% ===== 参考文献 =====
% 使用BibTeX生成参考文献列表(ASCE author-year格式,适用于JCEM投稿)
% ASCE使用author-date系统,类似APA但有特殊格式要求

% 优化参考文献行间距以控制页数(策略1+2组合)
\setlength{\bibsep}{0pt plus 0.3ex}  % 减少条目间距
{\linespread{0.98}\selectfont  % 轻微压缩行距至0.97倍
\bibliographystyle{elsarticle-harv}
\bibliography{dj01}


\end{document}

% ===== ASCE 引用格式说明 =====
% 文内引用: \citep{key} → (Author Year), \citet{key} → Author (Year)
% 参考文献列表格式(需在.bib中确保):
%   - Authors. Year. "Title." Journal abbr. Volume (Issue): CID. DOI.
%   - 按作者姓氏字母顺序排列
%   - 期刊名需使用ASCE标准缩写
%   - 必须包含DOI
% 投稿前建议使用ASCE官方模板确保完全符合格式要求