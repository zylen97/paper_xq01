% !TEX program = pdflatex
% Cover Letter — for IJPM submission

\documentclass[12pt]{article}
\usepackage{geometry}
\geometry{left=2.5cm, right=2.5cm, top=2.5cm, bottom=2.5cm}
\usepackage{newtxtext}
\usepackage{hyperref}
\usepackage{setspace}
\onehalfspacing

\pagestyle{empty}

\begin{document}

\today

\bigskip\bigskip

Professor Martina Huemann\\
Editor-in-Chief\\
\textit{International Journal of Project Management}

\bigskip

Dear Professor Huemann,

\bigskip

We are pleased to submit our manuscript entitled ``\textbf{How digital communication tools use shapes improvisation capability in large-scale construction projects: The role of meta-knowledge}'' for consideration for publication in the \textit{International Journal of Project Management}.

This study examines how work-oriented and social-oriented digital communication tool use influence improvisation capability through the development of structural, relational, and procedural meta-knowledge in large-scale construction project teams. Drawing on distributed cognition theory and using time-lagged survey data from 355 project professionals in China, we find that different digital tool use patterns foster complementary yet distinct forms of meta-knowledge, which selectively transmit effects to improvisation capability. Structural meta-knowledge emerges as the dominant mediator linking digital collaboration to both immediacy and creativity dimensions of improvisation, while relational and procedural meta-knowledge exhibit more selective effects.

We believe this manuscript is well suited for the \textit{International Journal of Project Management} for the following reasons:

\begin{enumerate}
    \item It addresses a timely and practically relevant topic---how digital collaboration tools reshape team cognitive infrastructure in temporary project settings---that is central to the journal's scope on project organization and management.
    \item It introduces procedural meta-knowledge as a novel cognitive construct and provides a differentiated pathway framework that advances understanding of digital collaboration outcomes in project teams.
    \item The empirical findings offer actionable guidance for project managers and organizations seeking to design digital collaboration practices that support both rapid response and creative problem-solving under uncertainty.
\end{enumerate}

% TODO: Update word count after tables are added to the manuscript.
The manuscript contains approximately [X,XXX] words, including figures and tables (each counted as 300 words) and excluding references.

We confirm that this manuscript has not been published previously, is not under consideration for publication elsewhere, and its submission has been approved by all co-authors. All authors have no competing interests to declare.

We appreciate your time and consideration, and look forward to your response.

\bigskip\bigskip

Sincerely,

\bigskip\bigskip

Xueqing Gan (Corresponding Author)\\
School of Business, Jiangsu University of Science and Technology\\
Zhenjiang 212100, China\\
Email: TODO@just.edu.cn  % TODO: fill in correct email

\end{document}
