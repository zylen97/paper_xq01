% !TEX program = xelatex
% 这行告诉LaTeX编辑器使用XeLaTeX引擎编译(支持更好的字体处理)

% Elsarticle文档类声明:review格式,10pt字体(5号字),author-year引用风格
\documentclass[authoryear,preprint,review,times,10pt]{elsarticle}

% ===== 宏包导入部分 =====
% 数学相关宏包(必备!工程管理论文经常用到数学公式)
\usepackage{amsmath}      % 强大的数学环境,如align、equation等
\usepackage{amssymb}      % 数学符号,如∈、⊆、∀、∃等
\usepackage{amsfonts}     % 数学字体,如粗体向量、花体字母等
\usepackage{bm}           % 加粗数学符号

% 基础功能宏包(elsarticle已包含graphicx)
\usepackage{booktabs}     % 制作专业表格(\toprule、\midrule、\bottomrule)
\usepackage{threeparttable} % 支持表格注释
\usepackage{caption}      % caption格式控制
\usepackage{adjustbox}    % 自动调整表格大小
\usepackage{enumitem}     % 增强列表功能(itemize、enumerate)
\usepackage{lineno}       % 行号宏包(review格式需要)

% 页边距调整(草稿阶段使用,投稿时可注释掉)
\usepackage{geometry}
\geometry{
    left=2cm,    % 左边距(原约3.5cm)
    right=2cm,   % 右边距(原约3.5cm)
    top=1.2cm,     % 上边距(原约3cm)
    bottom=1.5cm,  % 下边距(调整后,页码不会太靠下)
    footskip=0.5cm % 文本底部到页脚的距离
}

\usepackage{hyperref}     % 超链接和PDF书签(应该放在最后导入)

% 移除frontmatter后的双横线
\makeatletter
\def\ps@pprintTitle{%
  \let\@oddhead\@empty
  \let\@evenhead\@empty
  \def\@oddfoot{\hfil\thepage\hfil}%  % 页脚居中显示页码
  \let\@evenfoot\@oddfoot}
% 重新定义pprintMaketitle,移除横线但保留Email显示
\def\pprintMaketitle{\clearpage
  \iflongmktitle\if@twocolumn\let\columnwidth=\textwidth\fi\fi
  \resetTitleCounters
  \def\baselinestretch{1}%
  \printFirstPageNotes  % 恢复脚注显示(包含Email和通讯作者信息)
  \begin{center}%
   \thispagestyle{pprintTitle}%
   \def\baselinestretch{1}%
    \Large\@title\par\vskip18pt
    \normalsize\elsauthors\par\vskip10pt
    \normalsize\itshape\elsaddress\par\vskip12pt  % 改为\normalsize,进一步增大affiliation字体
    % 横线被移除了,原本这里有\hrule命令
  \end{center}%
  \gdef\thefootnote{\arabic{footnote}}%
  }
\makeatother

% 指定目标期刊(根据实际投稿期刊修改)
\journal{International Journal of Project Management}

\begin{document}

% 行号设置(审稿格式)
\modulolinenumbers[1]  % 每行都显示行号(而不是默认每5行)
\runninglinenumbers  % 使用连续行号(适用于所有环境包括frontmatter)
\begin{frontmatter}

% 论文标题
\title{How digital communication tools use shapes improvisation capability in large-scale construction projects: The role of meta-knowledge}

% 作者信息
\author[labela]{Zilun\corref{cor1}}
\ead{zylenw97@usts.edu.cn}
\author[labelb]{Ying Tang}
\ead{19552089006@163.com}
\author[labelc]{Dongjie Zheng}
\ead{2531180@tongji.edu.cn}
\author[labeld]{Qinghua He}
\ead{heqinghua@tongji.edu.cn}

\cortext[cor1]{Corresponding author.}

% 机构信息
\affiliation[labela]{organization={School of Civil Engineering, Suzhou University of Science and Technology},
            city={Suzhou},
            postcode={215011},
            country={China}}

\affiliation[labelb]{organization={School of Civil Engineering, Suzhou University of Science and Technology},
            city={Suzhou},
            postcode={215011},
            country={China}}

\affiliation[labelc]{organization={School of Economics and Management, Tongji University},
            city={Shanghai},
            postcode={200092},
            country={China}}

\affiliation[labeld]{organization={School of Economics and Management, Tongji University},
            city={Shanghai},
            postcode={200092},
            country={China}}

\begin{abstract}
Improvisation capability enables project teams to respond effectively to unexpected situations. While digital communication tools are widely used in projects, limited research has examined how different patterns of digital communication tool use influence improvisation capability or the cognitive mechanisms involved. This study develops and tests a model linking work-oriented and social-oriented digital communication tool use to improvisation capability through meta-knowledge, defined as shared awareness of expertise, relationships, and procedures. Using time-lagged survey data from 355 project members in China, the results show that different digital communication tool use patterns foster complementary forms of meta-knowledge, which selectively transmit effects to improvisation capability. Structural meta-knowledge emerges as the dominant mediator, whereas relational and procedural meta-knowledge exhibit more selective effects. This research contributes by distinguishing work-oriented from social-oriented digital communication tool use, introducing procedural meta-knowledge as a novel cognitive construct, and identifying differentiated meta-knowledge pathways linking digital collaboration to improvisation capability. The findings offer practical guidance for designing digital collaboration practices that enhance team capability in uncertain project environments.
\end{abstract}

\begin{keyword}
Digital communication tools \sep Meta-knowledge \sep Improvisation capability \sep Large-scale construction projects
\end{keyword}

\end{frontmatter}

% ===== 正文各章节 =====

\section{Introduction}
Large-scale construction projects—including highways, bridges, airports, and urban infrastructure—are characterized by high capital investment, long project cycles, complex technical interdependencies, and the involvement of multiple stakeholders (Jiang et al., 2025; Miner et al., 2001). These projects frequently encounter unexpected technical problems, supply chain disruptions, design changes, and environmental uncertainties that cannot be fully addressed through pre-planned solutions. Under such conditions, project teams must improvise by creating and implementing novel solutions in real time using available resources. Improvisation capability (IC), defined as the ability to integrate design and execution by leveraging existing knowledge under time pressure (Vera & Crossan, 2005), has therefore emerged as a critical success factor in large-scale construction projects (Cao and Yu, 2019; Leonardi, 2014).

Developing improvisation capability is particularly challenging in large construction projects. Project teams are temporary, multi-organizational assemblies composed of owners, contractors, designers, suppliers, and consultants, each contributing specialized expertise but often lacking prior collaborative experience. Coordination occurs across organizational and professional boundaries, participants are geographically dispersed, and membership turnover is common across project. In such contexts, effective improvisation depends less on individual expertise than on the collective ability to rapidly locate, mobilize, and integrate distributed knowledge (Ellison et al., 2015; Nisula and Kianto, 2016). Team members must understand not only what knowledge exists, but also where it resides, through whom it can be accessed, and how it has been applied in similar situations—what we conceptualize as meta-knowledge phases (Cao and Yu, 2019; Engelbrecht et al., 2019; Sun et al., 2021).

Digital communication tools such as WeChat, and DingTalk have become central to coordination in contemporary construction projects (Zhu and Jin, 2023). However, existing research has largely treated digital tool use as a homogeneous construct, assuming that greater use leads to broadly similar collaboration outcomes. This assumption overlooks an important distinction in the literature: digital tool use can be categorized into work-oriented use (task-focused interactions for formal project coordination, technical discussion, and document sharing) and social-oriented use (relationship-focused interactions for informal exchange and interpersonal connection) (Ali-Hassan et al., 2015; Chen & Wei, 2019). In large-scale projects with diverse participants and complex coordination needs, different usage patterns may generate distinct cognitive and relational infrastructures, yet little is known about how they shape meta-knowledge development or whether they differentially enable improvisation capability.

Moreover, while prior research has examined structural and relational meta-knowledge—awareness of “who knows what” and “through whom knowledge can be accessed”—it has largely overlooked procedural meta-knowledge, or shared understanding of “how work gets done” in project settings (Leonardi, 2014; Sun et al., 2022). In large construction projects, where formal procedures are often incomplete across organizational boundaries and teams lack stable shared experience, this procedural dimension may be particularly critical for enabling rapid action when unexpected situations arise.

Drawing on distributed cognition theory, which conceptualizes cognition as emerging from interactions among individuals, technological artifacts, and the social environment (Hutchins, 2020; Leonardi, 2015), this study examines how work-oriented and social-oriented digital communication tool use influence improvisation capability through the development of structural, relational, and procedural meta-knowledge in large-scale construction projects. We test the proposed model using time-lagged survey data collected from 355 professionals involved in large infrastructure projects in China.

This study aims to answer three research questions:

(1) How do work-oriented and social-oriented digital communication tool use influence structural, relational, and procedural meta-knowledge?

(2) How do structural, relational, and procedural meta-knowledge affect improvisation capability?

(3) Through which meta-knowledge pathways do digital communication tool use patterns transmit their effects to improvisation capability?

The remainder of the paper is structured as follows. Section 2 reviews the literature on improvisation capability, digital communication tools, and meta-knowledge. Section 3 develops the research hypotheses. Section 4 describes the methodology. Section 5 presents the results. Section 6 discusses the findings and their implications. Section 7 concludes with limitations and directions for future research.

\section{Literature review}
\subsection{Improvisational competence and digital communication tools usage in project teams}
IC has been widely recognized as a critical capability for coping with complexity and uncertainty, particularly when individuals or teams must respond without relying on detailed prior planning. Improvisation was initially understood primarily as a temporal phenomenon, emphasizing the convergence of the action formulation and execution (Moorman and Miner, 1998). Over time, extant scholars argued that a purely temporal perspective is insufficient and reconceptualized improvisation as a process through which novel solutions emerge from the recombination of existing resources and knowledge under constraints. Accordingly, improvisation has been described as “the deliberate and substantive fusion of the design and execution of new action” and “the real-time convergence of conception and execution” (Miner et al., 2001; Sun et al., 2022; Kyriakopoulos, 2011). Building on this perspective, (Magni et al., 2009) conceptualized IC as the ability to respond creatively and immediately to unexpected events, emphasizing that effective improvisation depends not only on possessing knowledge, but on mobilizing and coordinating it under time pressure. This definition has since been widely adopted in subsequent research (e.g., Liu et al., 2023). Consistent with this stream of research, the present study defines IC as an individual’s ability to integrate design and execution by leveraging existing knowledge and resources to generate novel solutions in the absence of prior planning.

Although improvisation has been primarily examined at the individual level, its relevance within temporary project teams has received comparatively limited attention. In temporary project teams, improvisation is particularly critical yet difficult to achieve due to dispersed expertise and severe time constraints (Bechky, 2006; Okhuysen and Bechky, 2009). Large-scale construction projects are frequently characterized by dispersed expertise, knowledge overload, and severe time pressure, which jointly constrain individuals’ ability to search for, interpret, and apply relevant knowledge when unexpected events arise (Sumbal et al., 2025). Under such conditions, effective improvisation depends less on generating new knowledge than on the ability to rapidly mobilize existing expertise—specifically, knowing whom to contact, how to communicate, and which prior solutions or templates can be drawn upon in real time. This understanding aligns with research on transactive memory systems, which highlights that performance under uncertainty relies not only on the availability of expertise, but also on shared awareness of where that expertise resides and how it can be accessed (Lewis, 2003; Hu et al. 2024).

The widespread adoption of DCTs has substantially altered how information is shared and accessed in project teams (Ding et al., 2019; Leonardi, 2014). Research on DCTs has primarily emphasized how their technical characteristics—such as visibility, accessibility, and social affordances—facilitate external knowledge acquisition and utilization (Leonardi, 2014; Sun et al., 2022). However, DCTs do not automatically enhance IC. Their effectiveness depends on whether specific use patterns reinforce the cognitive conditions that enable rapid and creative action, such as awareness of expertise distribution and the ability to mobilize knowledge in unexpected situations (Jiang et al., 2025). Accordingly, DCT usage is often conceptualized as a goal-oriented behavior, with scholars distinguishing between work-oriented and social-orineted use patterns, showing that these patterns have distinct implications for task and relational dimensions of work outcomes (Ali-Hassan et al., 2015; Zhang et al., 2018).

In temporary project teams, DCTs function not only as critical work support systems but also as essential social infrastructures (Huang and Liu, 2017; Jia et al., 2025)). Given limited shared work history and high uncertainty, how DCTs are used may shape not only information exchange but also how team members develop shared understanding and mobilize knowledge in unexpected situations. Therefore, a key unanswered question remains as to how different patterns of DCT usage enable—or constrain—the development of IC in temporary project teams.

\subsection{Meta-knowledge in project teams}
Meta-knowledge serves as a critical cognitive resource that enables project teams to navigate dispersed expertise and respond adaptively to evolving task demands and unexpected events (Ren and Argote, 2011; Wegner, 1987; Xue and Zou, 2022). Early research on meta-knowledge focused primarily on awareness of expertise distribution, conceptualized as knowing “who knows what” within a collective (Wegner, 1987). This line of work highlighted the coordination benefits of such awareness, particularly in settings where tasks require contributions from multiple specialists (Brown & Eisenhardt, 1997).

Building on these insights, meta-knowledge is now commonly discussed as a multidimensional construct encompassing structural and relational aspects. Structural meta-knowledge refers to understanding how expertise and knowledge domains are distributed across organizational members and units, enabling individuals to locate appropriate knowledge sources efficiently (Lewis, 2003; Ren and Argote, 2011). Relational meta-knowledge concerns awareness of social ties and interaction pathways through which knowledge can be accessed, particularly in contexts where formal roles provide limited guidance for coordination (Borgatti and Cross, 2003; Palazzolo et al., 2006). These two dimensions offer important insights into how project teams identify and access distributed expertise under conditions of time pressure and task complexity.

However, temporary and action-intensive project environments pose coordination challenges that exceed existing two-dimensional conceptualizations of meta-knowledge. Coordination in such settings often occurs under time pressure and ambiguity, requiring action before complete information is available. As a result, awareness of where expertise resides or how it can be accessed through whom, although important, is not sufficient on its own. Effective action also depends on shared understanding of how knowledge is enacted in practice, including commonly adopted task-handling approaches, typical responses to recurrent problems, and the sequencing of actions in practice (Anand et al., 1998; Faraj and Sproull, 2000).

These procedural elements are particularly salient in project contexts characterized by temporariness, dispersed participation, and high-intensity action. Project teams frequently lack a stable history of joint work, and membership turnover limits opportunities for experiential learning through prolonged co-participation. As a result, coordination often relies on implicit understandings of “how things are usually done” in comparable situations, rather than on formalized procedures or individualized expertise alone. While prior research has acknowledged the importance of routines and action patterns in collective work, these procedural aspects have not been systematically theorized as a distinct dimension of meta-knowledge.

Addressing this gap, this study introduces procedural meta-knowledge as a third dimension of meta-knowledge. Procedural meta-knowledge (PM) is defined as shared understanding of how knowledge is applied in practice, including routines, problem-solving sequences, and appropriate courses of action (Anand et al., 1998; Faraj & Sproull, 2011). Procedural meta-knowledge captures project members’ awareness of how distributed expertise is typically mobilized and combined in action, rather than merely where it resides or through whom it can be accessed.

Despite growing interest in meta-knowledge, most empirical insights have been derived from relatively stable teams, where membership continuity allows knowledge structures and coordination patterns to gradually stabilize over time (Mohammadparst Tabas et al., 2024; Zhang et al., 2023). Temporary project teams present a distinct context in which meta-knowledge must develop more rapidly and adapt more frequently to changing conditions, making it a particularly critical cognitive resource for rapid coordination and improvisation in project environments.

\subsection{Mechanisms linking DCT usage and IC: A distributed cognition theory perspective}
In temporary project teams, knowledge, cognition, and action are inherently distributed across individuals and digital communication tools. Under such conditions, individual-level cognitive perspectives provide limited explanatory power for understanding IC. This study therefore adopts Distributed Cognition Theory. By conceptualizing cognition as an emergent outcome of interactions among individuals, technological artifacts, and the social environment, distributed cognition theory provides a well-suited theoretical lens for examining how DCT usage shape IC in temporary project teams (Safadi, 2024; Swart et al., 2022). 

Within distributed cognitive systems, meta-knowledge has been discussed as a form of “cognitive indexing” that helps individuals understand how knowledge resources are distributed, related, and accessed (Engelbrecht et al., 2019; Ye and Chen, 2021). Rather than emphasizing the possession of substantive expertise, this line of research highlights knowing where relevant knowledge resides, how it can be obtained, and how it may be mobilized in action. As DCTs become increasingly embedded in organizational work, interaction processes that were previously transient or implicit are more likely to be externalized and stabilized. Persistent communication traces, visible interaction histories, and routinized collaboration structures thus constitute important conditions under which meta-knowledge can form and be maintained (Palazzolo et al., 2006; Leonardi, 2014; Treem et al., 2020).
A growing body of research suggests that the development and activation of meta-knowledge is closely related to improvisational action, particularly in contexts characterized by dispersed expertise (Huang and Liu, 2017; Oostervink et al., 2016)). This mechanism is particularly salient in large-scale construction projects, where specialized knowledge spans multiple professional domains. In such settings, individuals often lack the time or capacity to acquire new expertise when unexpected events arise and instead depend on accessing and coordinating existing knowledge. At the same time, organizational knowledge is widely distributed across departments, teams, and individuals, making its coordination both essential and challenging (Gold et al., 2001)). These conditions foreground the importance of mechanisms that support the identification and mobilization of distributed expertise.

From a distributed cognition perspective, DCTs can be understood as reshaping how cognitive activities are organized and sustained in project teams (Hutchins, 2020). Rather than functioning solely as channels for information transmission, DCTs have been described as technological artifacts through which cognitive processes are embedded in everyday work practices (Leonardi 2015). Through work use, team members externalize and preserve task-related interactions, problem-solving discussions, and coordination activities that would otherwise remain ephemeral (Gupta et al., 2013). Over time, the accumulation of such communication traces may support a clearer understanding of how knowledge is distributed, where professional boundaries lie, and how different knowledge domains are interconnected. 

In contrast, social use of DCTs expands interaction beyond formal task structures, enabling more flexible connections. Informal exchanges and socially oriented interactions provide additional cues about relationship networks, communication pathways, and potential routes for accessing expertise (Nisula and Kianto, 2016; Zhang et al., 2018). 

Prior research suggests that interaction patterns shape how meta-knowledge develops in collective work (Engelbrecht et al., 2019; Ren and Argote, 2011; Yuan et al., 2010). From a distributed cognition perspective, DCTs can be understood as reshaping how cognitive activities are organized and sustained in project teams. Rather than functioning solely as channels for information transmission, these technologies serve as technological artifacts through which cognitive processes become embedded in everyday work practices (Leonardi, 2015). This perspective provides a coherent framework for explaining how different patterns of DCT usage, including work-oriented and social-oriented interaction, support the development of distinct forms of meta-knowledge that subsequently enable improvisation.

% ===== 图片占位符 =====
\begin{center}[Insert Fig. 1 here]\end{center}

\section{Hypothesis development}
\subsection{DCT usage and meta-knowledge}
Work use of DCTs (WU) refers to team members’ purposeful creation, sharing, and use of task-related information to accomplish work goals. In temporary project teams, where shared work history is limited and interaction patterns are unstable, early understanding of “who knows what” is often fragmented and shaped by formal roles rather than demonstrated expertise.

WU addresses this limitation by externalizing task-related contributions into visible and persistent digital traces, such as project logs, technical discussions, and problem-solving records (Gupta et al., 2013). Through task-focused search and routine exposure to these records, team members can observe what types of expertise are demonstrated in different contributions and infer content-based expertise domains under time pressure (Ellison et al., 2015; Leonardi, 2015). By mapping substantive knowledge to specific actors based on their task contributions, work use supports the development of structural meta-knowledge. Accordingly, we propose:

H1a: WU is positively associated with SM.

Beyond identifying expertise locations, effective coordination also requires understanding how expertise can be accessed through social pathways. In temporary project teams, such relational understanding cannot rely on gradually accumulated shared experience (Ding et al., 2019).

By making task-related interactions visible over time, WU reveals who interacts with whom, how responses are routed, and which actors repeatedly coordinate across tasks. Attention to participation patterns and response sequences allows team members to infer practical interpersonal pathways for accessing knowledge, even when formal authority offers limited guidance (Borgatti and Cross, 2003; Leonardi and Vaast, 2017). Unlike informal social ties, these patterns reflect role-based coordination structures that emerge from task execution. By exposing how work flows between people, work use supports the development of relational meta-knowledge. Accordingly, we propose:

H1b: WU is positively associated with RM.

WU also facilitates the development of procedural meta-knowledge by externalizing work processes into visible and persistent digital traces. In temporary project teams, where formal process documentation is often incomplete or unevenly understood across organizational boundaries, accumulated records of task execution, coordination sequences, and problem-handling trajectories allow members to observe how work is actually carried out in practice (Anand et al., 1998; Faraj and Sproull, 2000). By reviewing how similar tasks were previously processed, individuals develop an understanding of procedural patterns and action sequences without relying on prolonged shared experience. This procedural visibility enables members to recognize familiar action templates and respond quickly when new situations arise. Accordingly, we propose:

H1c: WU is positively associated with PM.

\subsection{Social use of DCTs and meta-knowledge}
Unlike work use, social use of DCTs (SU) emphasizes relationship building rather than task execution. In temporary project teams, socially oriented interactions provide an important channel for exposure across organizational and disciplinary boundaries that is difficult to achieve through formal work processes alone.

Through social interactions, team members expand their interpersonal networks and gain access to experiential and relational information about colleagues’ backgrounds, experiences, and areas of competence. Unlike task-focused observations, social cues reveal personal histories, informal expertise developed outside current project tasks, and tacit knowledge that may not be visible in formal work exchanges (Ma et al., 2022; Zhang et al., 2025). As such interactions accumulate, these social cues enrich and refine perceptions of expertise distribution beyond what is observable in task-focused communication, complementing content-based structural awareness derived from work use. Accordingly, we propose:

H2a: SU is positively associated with SM.

SU also contributes to relational meta-knowledge by increasing visibility into informal interactional connections within the team. Informal exchanges expose patterns of co-engagement and recurring social interaction, enabling members to infer which ties serve as effective channels for knowledge access (Huang and Liu, 2017). Unlike the formal coordination patterns revealed by work use, these ties reflect trust-based relationships and personal connections that may operate outside established reporting structures. In addition, relational familiarity lowers barriers to interaction, making network structures observable through practice rather than explicit articulation. Accordingly, we propose:

H2b: SU is positively associated with RM.

In addition, SU supports procedural meta-knowledge by facilitating informal sensemaking about how work is carried out in practice. Through experience sharing, practical advice, and situational discussion, team members gain insight into accepted action sequences and coordination norms that are rarely captured in formal procedures. Unlike WU, which reveals procedurally relevant information through formal task execution traces, social use provides narrative and judgment-based insights into practice—tacit norms, preferred ways of handling issues, and situational judgments about when deviations are acceptable (Ali-Hassan et al., 2015; Cao and Yu, 2019). Repeated exposure to such exchanges supports shared understanding of how action typically unfolds, complementing procedural understanding derived from work use. Accordingly, we propose:

H2c: SU is positively associated with PM.

\subsection{Meta-knowledge and improvisational competence}
Prior research conceptualizes improvisation as the real-time recombination of existing cognitive structures to guide action under emergent conditions (Crossan and Sorrenti, 2003). In project teams, such action depends not only on individuals’ own expertise but also on their awareness of how knowledge and expertise are distributed across the collective (Ren & Argote, 2011)(Ren and Argote, 2011). Following this stream of research, we conceptualize improvisation capability as comprising two dimensions: immediacy, defined as the ability to respond rapidly under time pressure, and creativity, defined as the ability to generate novel solutions through knowledge recombination (Ciborra, 1996; Magni et al., 2009; Vera and Crossan, 2005). This dual-dimensional conceptualization captures both the speed and novelty requirements of improvisation in complex project environments. While prior research has often emphasized either immediacy or creativity, we argue that effective improvisation in project settings requires the development of both capabilities.

Structural meta-knowledge (SM) enables individuals to identify who possesses relevant domain-specific expertise when unexpected problems arise. In temporary project teams characterized by limited shared history and fragmented expertise, individuals’ own knowledge is often insufficient to address emergent issues. By clarifying the content-based distribution of expertise across team members, SM reduces the time and cognitive effort required to locate appropriate knowledge holders and supports rapid response under time pressure. Accordingly, we propose:

Beyond speed of response, improvisation often requires integrating heterogeneous knowledge across disciplinary and organizational boundaries (Ye and Chen, 2021). Structural meta-knowledge facilitates such integration by enabling individuals to mobilize diverse knowledge domains beyond their immediate task scope. This awareness supports cross-disciplinary interaction and the recombination of distributed knowledge elements, thereby enhancing the generation of novel action plans under uncertainty. Accordingly, we propose:

H3a: SM is positively associated with the IMM.

H3b: SM is positively associated with CRE.

Relational meta-knowledge (RM) enables understanding of how expertise can be accessed through interpersonal pathways. It provides awareness of who is connected to whom, which actors serve as intermediaries, and how indirect ties can be activated in practice (Huang and Liu, 2017). In temporary project teams with weak formal authority structures, such relational awareness shortens the path between problem recognition and coordinated action by enabling pathway-based access to expertise. 

RM also expands access to heterogeneous perspectives by enabling targeted coordination through informal networks (Faraj and Sproull, 2000). By supporting selective activation of appropriate ties and intermediaries, it facilitates focused interaction among actors with different professional logics, thereby supporting the recombination of diverse knowledge inputs and the generation of creative solutions. Accordingly, we propose:

H4a: RM is positively associated with IMM.

H4b: RM is positively associated with CRE.

Procedural meta-knowledge (PM) concerns understanding how work is typically carried out in practice, including common task sequences, coordination routines, and acceptable procedural pathways. In emergent situations, delays often stem from uncertainty about how to proceed rather than lack of effort (Rulke and Rau, 2000; Sun et al., 2022). Procedural meta-knowledge provides familiar action templates that allow individuals to translate situational cues into action without prolonged deliberation. Accordingly, we propose:

PM also supports improvisational creativity by enabling individuals to adapt and recombine familiar action patterns when existing procedures prove insufficient. This shared understanding reduces uncertainty about the feasibility of alternative courses of action, making experimentation and hybrid solutions more viable under time pressure. Accordingly, we propose:

H5a: PM is positively associated with IMM.

H5b: PM is positively associated with the CRE.

\subsection{The mediating role of meta-knowledge}
DCTs have become important platforms for knowledge externalization in contemporary project teams (Ma et al., 2022; Oostervink et al., 2016). Through the use of DCTs, team members continuously generate and accumulate visible cognitive cues about the collective knowledge system, which support the development of meta-knowledge. Specifically, structural, relational, and procedural meta-knowledge provide cognitive frameworks that help members understand where knowledge is located, through whom it can be accessed, and how it can be applied in action, enabling them to rapidly locate, mobilize, and integrate dispersed knowledge resources within complex knowledge systems.

In temporary project teams, members often lack a stable history of collaboration and face rapidly changing task conditions, making formal institutions and established procedures insufficient for addressing emergent demands (Jiang et al., 2025). In this context, meta-knowledge functions as a key mediating mechanism. By reducing the costs of knowledge search and coordination, it helps members translate the information and interactions enabled by DCTs into executable improvisational actions, thereby supporting both the immediacy and creativity of responses to unexpected situations.

WU contributes to improvisational competence primarily by externalizing task-related traces that support the development of structural, relational, and procedural meta-knowledge. Through systematic documentation and visibility of task execution, WU enables team members to construct reliable maps of knowledge content distribution, formal coordination pathways, and action patterns that can be rapidly mobilized when unexpected situations arise (Zhang et al., 2018). In contrast, SU contributes to IC by facilitating informal interactions that complement task-based observations, providing relational and experiential insights that enhance meta-knowledge development through non-work channels (Nisula and Kianto, 2016). Both use patterns thus support IC through distinct yet complementary pathways centered on meta-knowledge formation. Accordingly, we propose:

H6a: SM mediates the relationship between WU and IMM.

H6b: SM mediates the relationship between WU and CRE.

H6c: RM mediates the relationship between WU and IMM

H6d: RM mediates the relationship between WU and CRE.

H6e: PM mediates the relationship between WU and IMM.

H6f: PM mediates the relationship between WU and CRE.

H6g: SM mediates the relationship between SU and IMM.

H6h: SM mediates the relationship between SU and CRE.

H6i: RM mediates the relationship between SU and IMM.

H6j: RM mediates the relationship between SU and CRE.

H6k: PM mediates the relationship between SU and IMM.

H6l: PM mediates the relationship between SU and CRE.

\section{Methodology}
\subsection{Data collection}
To enhance the reliability and validity of the survey instrument, a pilot study was conducted prior to the formal data collection. Using a snowball sampling approach, employees with experience using DCTs in project settings were recruited through the research team's professional networks. A total of 161 questionnaires were distributed, and 137 valid responses were obtained after excluding incomplete or inconsistent questionnaires, yielding an effective response rate of 85.09\%. The pilot results indicated satisfactory psychometric properties for all measurement scales (Cronbach’s ɑ > 0.70), supporting their suitability for the main study. 

The formal data collection employed a time-lagged survey design to reduce common method bias (Podsakoff et al., 2003). Data were collected from participants involved in large-scale infrastructure projects in mainland China, including residential buildings, public buildings,infrastructures, and industrial buildings. These projects typically involve multiple organizational participants, complex coordination requirements, and high levels of uncertainty, making them an appropriate context for examining DCT usage and improvisational competence. The formal survey was distributed through a Chinese construction training organization that provides government-recognized certification programs for project professionals. The organization’s trainee and alumni network include practitioners from diverse organizational roles and project types, providing access to a broad and heterogeneous sample. 

In the Chinese construction context, teams predominantly rely on three digital collaboration platforms—QQ, WeChat, and DingTalk—for daily coordination and communication (Zhu and Jin, 2023; Jia et al., 2024). The survey therefore assessed respondents’ tool use specifically within these three digital environments. Only respondents who had been involved in their current projects for at least six months and who reported regular use of digital communication tools for project collaboration were invited to participate. 

At Time 1, respondents were asked to report demographic information, DCT usage patterns, and meta-knowledge variables. A total of 978 questionnaires were distributed, and 567 valid responses were returned within one week, resulting in an effective response rate of 57.98\%. One month later, the Time 2 survey was administered to respondents who had provided valid responses at Time 1. This wave measured improvisation capability, including immediacy and creativity. To match responses across the two waves, respondents were asked to provide the last four digits of their mobile phone numbers as a unique identifier. A total of 409 valid questionnaires were collected at Time 2, yielding a response rate of 72.1\% relative to the Time 1 sample. After matching the two survey waves and excluding cases with missing or inconsistent data, the final sample comprised 355 valid responses, representing an overall retention rate of 62.6\% from Time 1. Descriptive statistics of the sample are presented in Table 1.

To ensure data quality, several measures were implemented. First, the professional survey company conducted IP address checks to prevent multiple submissions and removed responses with straight-lining patterns or insufficient response time (less than 50\% of median completion time). Second, an attention check question was embedded in the survey to verify that respondents were reading the questions carefully. Third, we tested for non-response bias by comparing early respondents (first 25\%) with late respondents (last 25\%) on key demographic variables (professional qualifications, work experience) using T-tests (Jia et al., 2025; Meyer et al., 2014). No significant differences were found (p > 0.05), suggesting that non-response bias was not a serious concern.

\subsection{Measures}
To ensure measurement rigor, all constructs in this study were operationalized using established scales adapted from existing research. To ensure contextual appropriateness and construct equivalence in the Chinese project setting, all measurement items were translated using a back-translation procedure. The translation process involved independent forward translation into Chinese, back-translation into English, and reconciliation by bilingual researchers with expertise in project management and organizational behavior. Discrepancies were discussed and resolved to ensure conceptual consistency with the original scales while maintaining relevance to large-scale project contexts in China. All items were measured using a five-point Likert scale ranging from 1 (“strongly disagree”) to 5 (“strongly agree”). 

DCT usage was measured using six items adapted from Chen and Wei (2019), including three items capturing work-oriented use and three capturing social-oriented use. Structural and relational meta-knowledge were each assessed using three-item scales adapted from Engelbrecht et al. (2019). Procedural meta-knowledge was measured using three items adapted from Faraj and Sproull (2000) and modified to reflect the project context. Improvisation capability was measured using eight items from Magni et al. (2009), comprising four items for immediacy and four for creativity. All items were adapted to the construction project context and are reported in the Appendix.

Prior research has shown that individual characteristics such as gender, age, and educational background are important determinants of IC (Koutsoupidou, 2005; Magni et al., 2009). Accordingly, gender, age, and educational background are set as control variables.


\section{Results}
\subsection{Reliability and validity}
To assess the reliability and validity of the measurement instruments, confirmatory factor analysis (CFA) was conducted using Mplus 8 (Foss et al., 2011; Pesämaa et al., 2021). The proposed measurement model demonstrates a good fit to the data (χ²(330) = 513.10, p < 0.001; CFI = 0.963; TLI = 0.957; RMSEA = 0.040, 90\% CI [0.033, 0.046]; SRMR = 0.079), exceeding commonly accepted thresholds for model adequacy. These results indicate that the hypothesized multi-construct structure is well supported by the empirical data.

The internal consistency of the constructs was examined using Cronbach’s α and composite reliability (CR). As reported in Table 2, Cronbach’s α values range from 0.823 to 0.881, and CR values range from 0.826 to 0.881, all exceeding the recommended threshold of 0.70. These results suggest satisfactory internal reliability across all constructs (Eisinga et al., 2013).

Convergent validity was assessed using the average variance extracted (AVE). As shown in Table 2, the AVE values range from 0.593 to 0.649. All values exceed or are close to the recommended cutoff of 0.50, indicating that the measurement items adequately capture the variance of their corresponding latent constructs. Discriminant validity was evaluated following the Fornell-Larcker criterion. As presented in Table 2, the square root of the AVE for each construct is greater than its correlations with all other constructs, providing evidence of satisfactory discriminant validity (Chen, 2007; Hair et al., 2019).

\subsection{Hypothesis testing}
To test the proposed hypotheses, a covariance-based structural equation modeling (CB-SEM) approach was employed to obtain robust parameter estimations for the complex research model. The results of the structural model are presented in Tables 3 and 4.

The empirical results indicate that WU has a significant positive effect on SM (β = 0.247, 95\% CI = [0.141,0.353], p < 0.001) and PM (β = 0.294, p < 0.001), thereby supporting H1a and H1c. However, the effect of WU on RM is insignificant (β = 0.060, p > 0.05), and thus H1b is not supported. Similarly, SU shows significant positive effects on SM (β = 0.204, p < 0.001) and RM (β = 0.193, p < 0.001), supporting H2a and H2b. In contrast, the effect of SU on PM is insignificant (β = 0.036, p > 0.05), not supporting H2c.

Regarding the effects of meta-knowledge on IC, SM exhibits significant positive effects on both IMM (β = 0.205, 95\% CI = [0.092,0.318], p < 0.001), and CRE (β = 0.136, 95\% CI = [0.028, 0.244], p < 0.05), supporting H3a and H3b. RM demonstrates a significant positive association with IMM (β = 0.174, 95\% CI = [0.066, 0.282], p < 0.01). However, the 95\% confidence interval for the effect of RM on CRE includes zero (β = 0.030, 95\% CI = [-0.074, 0.133], p > 0.1), indicating that H4a is supported whereas H4b is not. PM shows a significant positive effect on CRE (β = 0.256, 95\% CI = [0.156,0.357], p < 0.001) but does not significantly affect IMM (β = 0.009, 95\% CI = [-0.096, 0.113], p > 0.1), Therefore, H5b is supported, while H5a is rejected.

Bootstrapping with 5,000 resamples was conducted to assess the significance of the indirect effects, and the results are reported in Table 4. The findings indicate that the indirect effect of WU on IMM through SM is significant, as the 95\% bootstrap confidence interval does not include zero (β = 0.051, 95\% CI = [0.018, 0.089]). Similarly, the indirect effect of WU on CRE via SM is also significant (β = 0.034, 95\% CI = [0.004, 0.070]). Therefore, H6a and H6b are supported.

In contrast, the 95\% confidence interval for the indirect effect of WU on IMM and CRE through RM do include zero (WU → RM → IMM: β = 0.010, 95\% CI = [−0.013, 0.035]; WU → RM → CRE: β = 0.002, 95\% CI = [−0.007, 0.015]). Thus, H6c and H6d are not supported.

The results show that the indirect effect of WU on CRE through PM is significant (β = 0.075, 95\% CI = [0.034, 0.128]), whereas the indirect effect on IMM is insignificant because the confidence interval includes zero (β = 0.003, 95\% CI = [−0.038, 0.040]). Hence, PM mediates the relationship between WU and CRE but not the relationship between WU and IMM. With respect to SU, the results reveal significant indirect effects on IMM through both SM (β = 0.042, 95\% CI = [0.013, 0.079]) and RM (β = 0.034, 95\% CI = [0.008, 0.069]). However, the indirect effect through PM is not significant (β = 0.001, 95\% CI = [−0.009, 0.010]). Therefore, SM and RM mediate the relationship between SU and IMM, while PM does not. For CRE, SU shows a significant indirect effect through SM (β = 0.028, 95\% CI = [0.002, 0.059]), whereas the mediating effects through RM and PM are not supported, as their confidence intervals include zero (SU → RM → CRE: β = 0.006, 95\% CI = [−0.017, 0.030]; SU → PM → CRE: β = 0.009, 95\% CI = [−0.023, 0.043]).

\subsection{Endogeneity and robustness checks}
To enhance the robustness of our findings, we conducted supplementary analyses. After controlling for demographic variables, a simplified model excluding meta-knowledge variables showed that both WU and SU remained significantly positively associated with IC (IMM and CRE), indicating robust main effects.

To examine potential endogeneity, we explored reverse causality. Reverse regression showed that improvisation capability positively predicts WU (β = 0.195, p < 0.001), suggesting a mutually reinforcing relationship between digital tool use and team capability. This aligns with the co-evolution view of technology use and organizational capabilities. The Durbin–Wu–Hausman test for WU was significant (p < 0.05), indicating that a strictly unidirectional causal interpretation should be treated with caution (Becker et al., 2016).

In summary, these analyses suggest that DCT usage and IC likely exhibit a mutually reinforcing relationship. While our theoretical model and time-lagged design support the proposed pathway, the endogeneity tests imply that teams with stronger improvisation capability may also more effectively leverage DCT. The meta-knowledge mediation mechanism revealed in this study should therefore be considered as a core mechanism within this broader dynamic relationship rather than as an exclusive causal chain. Future research using panel data or instrumental variables would help clarify causal direction and strengthen the robustness of the proposed model.

\section{Discussion}
\subsection{Main findings}
As summarized in Table 5, the empirical analysis reveals that while most hypotheses are supported, several require further investigation. The following discussion examines these key findings in detail.

\subsubsection{The asymmetric effects of DCT usage on meta-knowledge}
The results reveal distinct patterns in how different DCTs usage orientations shape meta-knowledge. WU exhibits significant positive effects on SM and PM, but not on RM. In contrast, SU significantly enhances SM and RM, while showing no significant effect on PM. These asymmetric effects suggest that different DCTs usage fosters complementary yet distinct forms of meta-knowledge infrastructure. 

The differential effects of WU can be understood through its task-oriented nature. Work-focused interactions are typically structured around coordination, role clarity, and workflow execution—activities that naturally generate shared understanding of “who knows what” (SM) and “how work gets done” (PM) (Cho and Lee, 2022; Neff et al., 2010). Frequent task discussions, problem-solving, and process alignment continuously reinforce these structural and procedural knowledge components. However, the transactional and role-bound character of WU provides limited exposure to socio-emotional cues, team members’ collaboration styles, or informal relationship dynamics. Without deeper interpersonal exchanges, relational awareness remains underdeveloped, explaining the non-significant effect on RM.

SU demonstrates a theoretically complementary pattern. Social interactions extend beyond formal task boundaries and provide socio-emotional cues that enhance understanding of team members’ preferences, expectations, and informal networks (Brockhaus et al., 2023; Cho and Lee, 2022). This interpersonal familiarity directly strengthens RM. At the same time, the spontaneous and less task-focused nature of SU does not consistently produce stable or codifiable procedural knowledge, accounting for the non-significant relationship with PM. As a result, through their distinct yet complementary interaction patterns, both WU and SU tend to foster the development of SM, while exhibiting complementary effects on RM and PM.

\subsubsection{The differentiated effects of meta-knowledge on IC}
The results consistently show that SM is positively associated with both IMM and CRE (H3a–H3b supported), indicating that shared awareness of expertise distribution benefits both dimensions of improvisation capability. More nuanced patterns emerge for RM and PM, whose effects display a clear asymmetry across the two outcomes. RM is positively associated with IMM but shows no significant relationship with CRE (H4a supported; H4b not supported). In contrast, PM exerts a significant positive effect on CRE but not on IMM (H5b supported; H5a not supported). Taken together, the three meta-knowledge dimensions do not contribute to IC in a uniform manner, but exhibit distinct and complementary relationships with IMM and CRE.

The above results suggest that, although improvisation requires teams to integrate distributed knowledge resources under time pressure and uncertainty, the mechanisms differ across IMM and CRE. A knowledge integration perspective provides a coherent explanation for this asymmetric pattern (Faraj and Sproull, 2000; Perry-Smith and Mannucci, 2017). In this context, SM appears to function as a general integration infrastructure that supports both outcomes, while RM and PM correspond to more specialized integration pathways. This asymmetry reflects the efficiency–innovation tension inherent in DCT-enabled collaboration. IMM relies primarily on rapid mobilization and alignment of relevant expertise, whereas CRE depends more strongly on the recombination of heterogeneous knowledge inputs. Prior research helps explain why these integration mechanisms operate differently. The findings of (Soda et al., 2019) indicate that coordination-oriented interaction structures tend to reinforce shared understanding and streamline communication, but may simultaneously reduce exposure to diverse knowledge domains that enable novel recombination. Conversely, process-oriented knowledge provides a repertoire of routines that can be flexibly recombined, which supports creative outcomes. However, such procedural integration does not necessarily translate into faster action under urgency. In time-critical situations, procedural templates alone are insufficient if teams cannot rapidly align roles, coordinate interdependent actions, and enact accountability in real time. Okhuysen and Bechky (2009) showed that IMM depends more strongly on immediate alignment and accountability than on formalized procedures alone, providing further support for this interpretation. The knowledge integration perspective, combined with these established arguments in prior research, offers a plausible explanation for the differentiated support observed across H3a–H5b.

\subsubsection{The mediating role of meta-knowledge in DCT-IC}
Although DCT usage shows significant relationships with multiple types of meta-knowledge, only a limited set of mediation paths is supported. Specifically, SM consistently transmits the effects of both WU and SU to IC (H6a, H6b, H6g, H6h), whereas RM and PM operate as mediators only in isolated paths. This uneven pattern suggests that the cognitive consequences of DCT interaction may be more constrained and context-dependent than originally anticipated.

A team cognition perspective provides a coherent explanation for this pattern. Prior research has shown that shared cognitive structures do not emerge uniformly but develop unevenly depending on the informational cues embedded in interaction (Ren and Argote, 2011; Liu et al., 2008). DCTs are not neutral conduits of knowledge. Instead, they shape which cognitive structures are most likely to form through repeated interaction. The findings of Majchrzak et al. (2013) support this view by demonstrating that digital interaction generates persistent and highly visible traces of communication and problem solving. Such visibility makes expertise signaling and knowledge location particularly salient in digital environments. This perspective helps explain why SM emerges as the dominant and universal mediator. Persistent communication traces make it easier for team members to recognize and remember the distribution of expertise over time (Chen et al., 2020). In contrast, RM and PM rely on different types of interactional cues that are not consistently produced through DCT usage. Relational meta-knowledge depends on socio-emotional and contextual information that typically extends beyond task-focused interaction. Procedural meta-knowledge develops through repeated engagement in workflow execution and process coordination (Marion and Fixson, 2021). As a result, the cognitive consequences of digital interaction remain uneven, leading to selective mediation effects.

These findings suggest that the relationship between digital collaboration and performance is more contingent than often assumed. Digital interaction generates abundant communication, but only some of this interaction crystallizes into shared cognitive structures that can support improvisation capability (Ye and Chen, 2021; Zaverzhenets and Łobacz, 2021). 

\subsection{Theoretical implications}
\subsubsection{Reframing digital collaboration as cognitive infrastructure building}
This study reframes digital collaboration in project research by showing that digital communication tools primarily build team cognitive infrastructure rather than merely improve communication efficiency. Prior project management research has largely treated digital tool use as a homogeneous driver of coordination and information sharing outcomes (Faraj and Sproull, 2000; Leonardi, 2015; Sun et al., 2022). The present findings show that digital collaboration does not produce uniform effects; instead, its consequences depend fundamentally on how digital tools are used within project teams.

Distinguishing between work-oriented and social-oriented tool use reveals that different interaction orientations cultivate distinct forms of shared understanding. Task-focused interaction primarily strengthens awareness of expertise distribution and work processes, whereas socially oriented interaction reinforces relational awareness across team members. Digital collaboration therefore operates as a system through which project teams construct shared cognitive foundations that enable coordinated action. By shifting attention from technology adoption to interaction orientation, this study positions digital collaboration as a cognitive infrastructure-building process that underpins team capability in complex project environments (Leonardi, 2015; Majchrzak et al., 2013).

\subsubsection{Contribution to meta-knowledge and distributed cognition theory}
The findings extend distributed cognition and project knowledge research by showing that team capability depends not only on knowledge possession but also on the ability to mobilize distributed knowledge. Traditional project research has emphasized knowledge acquisition, expertise, and learning as primary drivers of performance (Hutchins, 2020; Leonardi, 2014; Oostervink et al., 2016). This perspective implicitly assumes that teams perform better when members possess more knowledge or higher levels of expertise.

The present study shows that, in digitally mediated project environments, the ability to locate, combine, and mobilize distributed knowledge becomes as important as knowledge possession and can be even more critical under time pressure (Mohammadparst et al., 2024; Moser and Deichmann, 2021; Jun et al., 2022). Team effectiveness therefore relies on shared awareness of how knowledge is distributed, how it can be accessed, and how it is enacted in practice. This perspective extends distributed cognition theory by highlighting knowledge mobilization as a central foundation of team capability.

The study further strengthens this theoretical shift by conceptualizing procedural meta-knowledge as a distinct dimension of team cognition. Existing research has primarily focused on awareness of expertise and relational connections, while shared understanding of how work is carried out in practice has remained under-theorized (Orlikowski and Scott, 2008; Zheng et al., 2023; Ren and Argote, 2011; Wegner, 1987)). The findings demonstrate that shared procedural understanding enables teams to recombine routines and respond creatively to unexpected situations. By identifying this cognitive mechanism, the study expands the conceptualization of team cognition and clarifies how distributed knowledge becomes actionable during project execution.

\subsubsection{Inconsistent DCT usage findings through differentiated pathways}
This study contributes to the digital collaboration literature by explaining why prior research has reported inconsistent and context-dependent outcomes. Previous studies have alternately emphasized the coordination benefits of digital tools and the new constraints they introduce, resulting in fragmented conclusions regarding their overall effectiveness (Majchrzak et al., 2013; Kane et al., 2014; Sun et al., 2025).

The present findings demonstrate that digital collaboration influences team capability through differentiated cognitive pathways rather than through a single uniform mechanism. Work-oriented and social-oriented interactions cultivate complementary but distinct forms of shared understanding, which in turn exert selective effects on improvisation capability. This pathway perspective clarifies how digital interaction can simultaneously enable coordination and generate uneven outcomes across teams and contexts.

By identifying structural, relational, and procedural meta-knowledge as the cognitive mechanisms through which digital interaction shapes team capability, the study provides a coherent explanation for the mixed findings in prior research. This perspective shifts the theoretical focus from whether digital collaboration improves performance to how digital collaboration reshapes team cognitive infrastructures that enable capability development.

\subsection{Practical implications}
This study provides practical insights for project owners and frontline managers seeking to use DCTs to support improvisation capability in large-scale construction projects.

Firstly, managing DCT usage as a knowledge-building process, not merely a communication channel. Our findings show that work-oriented use primarily builds structural and procedural meta-knowledge, whereas social-oriented use primarily builds structural and relational meta-knowledge. Because single-mode interaction creates meta-knowledge blind spots, project managers should design digital collaboration environments that combine both task-focused coordination and informal interaction spaces.

Secondly, work-focused DCT practices should systematically externalize expertise and work processes. In temporary project teams lacking shared history, participants rely on digital traces to understand “who knows what” and “how work gets done.” WU develops SM and PM by making task coordination, decision processes, and expertise distribution visible. Practices such as maintaining searchable project logs, documenting decision processes, and creating expertise directories enable teams to rapidly mobilize knowledge when unexpected situations arise.

Thirdly, creating dedicated spaces for informal interaction to build relational awareness. SU strengthens relational meta-knowledge, which supports rapid knowledge mobilization under time pressure—particularly critical in large scale   projects with fluid membership and pronounced organizational boundaries. Digital environments should therefore accommodate both structured work communication and relationship-building interactions, enabling teams to efficiently navigate interpersonal networks during urgent situations.

Furthermore, aligning meta-knowledge-building strategies with specific capability demands. According to the findings that RM supports improvisation immediacy, whereas PM supports improvisation creativity. In phases demanding rapid response, managers should prioritize practices that enhance expertise visibility and relationship networking. In phases requiring creative problem-solving, managers should focus on capturing and sharing procedural knowledge through lessons-learned repositories and structured reflection sessions.

Finally, recognize that digital interaction alone does not automatically generate meta-knowledge. Structural meta-knowledge emerges as the most consistent mediator linking DCT usage to improvisation, whereas RM and PM require more intentional design. Project leaders should therefore implement mechanisms that systematically externalize expertise distribution—such as skill-tagging systems and expertise directories—while creating spaces for informal exchange and relationship-building. By treating digital collaboration as designed infrastructure rather than an emergent byproduct, organizations can build the complementary knowledge foundations required for effective improvisation.

\subsection{Limitations and future research directions}
This study has several limitations should be acknowledged. A first limitation concerns the project context in which the proposed meta-knowledge pathways were examined. Large-scale construction projects are characterized by temporary,  and highly interdependent organizational structures. These features make meta-knowledge particularly critical for coordination and improvisation. However, the prominence of meta-knowledge observed in this study may be partly context-specific. In projects with more stable membership, higher task modularity, or stronger technological integration—such as software development or R&D teams—DCT usage may generate different cognitive consequences. Future research could therefore examine how project characteristics shape the relative importance of different meta-knowledge pathways and explore the boundary conditions of the proposed framework across diverse project settings.

Another limitation relates to the temporal nature of meta-knowledge development. Although the time-lagged design provides separation between predictors and outcomes, the study captures meta-knowledge as a relatively stable cognitive structure. In practice, meta-knowledge is likely to emerge, decay, and be reconfigured through ongoing digital interaction, team turnover, and shifting project demands. Understanding how these cognitive structures evolve over time represents an important next step. Longitudinal and process-based research could examine how meta-knowledge pathways unfold across project phases and how digital collaboration reshapes team cognition dynamically.

The study also relies on perceptual measures to capture meta-knowledge and digital collaboration. While appropriate for examining shared cognitive structures, this approach cannot fully capture the behavioral traces through which meta-knowledge develops in digital environments. Future research could combine survey and digital trace data to better understand how interaction patterns translate into shared cognitive infrastructures.

\section{Conclusion}
This study examines how digital communication tool use shapes large-sclae construction project teams’ improvisation capability through meta-knowledge development. Drawing on distributed cognition theory, the study develops and tests a theoretical model that distinguishes work-oriented and social-oriented DCT usage and conceptualizes structural, relational, and procedural meta-knowledge as key cognitive mechanisms linking digital interaction to team capability. The model is examined using time-lagged survey data collected from members of large-scale construction project teams.
The findings reveal that DCT usage does not influence improvisation capability uniformly but operates through differentiated meta-knowledge pathways. Work-oriented and social-oriented DCT usage foster distinct yet complementary forms of meta-knowledge, while structural meta-knowledge emerges as the most consistent mechanism linking digital collaboration to both dimensions of improvisation capability. Relational and procedural meta-knowledge exhibit more selective effects, highlighting the uneven cognitive consequences of digital interaction. Together, these results demonstrate that the impact of DCT usage depends on how digital interaction reshapes team cognitive infrastructures rather than on technology use alone.

These insights contribute to research on digital collaboration, distributed cognition, and project teamwork by clarifying the cognitive mechanisms through which digital interaction translates into team capability. The findings also offer guidance for project organizations seeking to design digital collaboration practices that support both rapid response and creative problem solving in complex project environments. Because the empirical context focuses on construction and infrastructure project teams in China, caution is warranted when generalizing the findings to other project settings and national contexts.

\section*{Data availability statement}
The data that support the findings of this study are available from the corresponding author upon reasonable request.

\section*{Acknowledgments}

% ===== 参考文献 =====
% 参考文献(Elsevier author-year 格式)
\bibliographystyle{elsarticle-harv}
\bibliography{xq01}


\end{document}

